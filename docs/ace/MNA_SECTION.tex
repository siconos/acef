
 
\section{Introduction, Modified Nodal Analysis}
This part is a short overview about the Modified Nodal Analysis(MNA). The MNA is the method used
in SPICE to obtain the circuit equation formulation. For more details, we refer the reader to \cite{Ogrodzki1994,chua1991}.

\subsection{Basics on circuit topology}

Circuits are basically composed of branches, characterized by a current and a related voltage.
Moreover, branches are connected to circuit nodes. The physical behavior of the circuit obeys
Kirchhoff's law. Circuit analysis leads to a mathematical formulation which will be describe in this
document.

\subsubsection{Circuit branches}
\begin{figure}[h]
\centerline{
 \scalebox{0.6}{
    \input{Branch.pstex_t}
 }
}
\caption{Basic branch}
\label{fig-Basic-branch}
\end{figure}
In any case, the branch is described by a pair of branch variables: the tension $U_{a}$ and the current $I_{a}$.
Moreover, the Branch Constitutive Equation can be expressed in a general implicit form:
\begin{equation}\label{BCE}F(U_{a},I_{a},...)=0\end{equation}
We will see the different forms of this relation.
\subsubsection{The Kirchhoff Current Law (KCL)}
\newtheorem{kcl}{Kcl}
\begin{kcl}
At any node in an electrical circuit where charge density is not changing in time, the sum of
currents flowing towards that node is equal to the sum of currents flowing away from that node.
\end{kcl}
The KCL law gives rise to this type of equation:\\
\begin{equation}
 \sum_{i} I_{i}=0\label{eq:KCL}
\end{equation}

\begin{figure}[h]
\centerline{
 \scalebox{0.6}{
    \input{SimpleCircuit.pstex_t}
 }
}
\caption{Circuit and a graph representing its topology}
\label{fig-Circuit-example}
\end{figure}
With the example shown in figure \ref{fig-Circuit-example}, the KCL leads to the following equations:
\[-I_{1}+I_{2}+I_{3}=0\]
\[-I_{3}+I_{4}=0\]
\[I_{1}-I_{2}-I{4}=0\]
or the matrix formulation :
\[AI=0\]
where A is known as the incidence matrix and I is the vector of branch currents.
\subsubsection{The Kirchhoff Voltage Law (KVL)}
\newtheorem{kvl}{Kvl}
\begin{kvl}
The directed sum of the electrical differences around a closed circuit must be zero.
\end{kvl}
With the example shown in figure \ref{fig-Circuit-example}, the KVL leads to the following
equations:
\[U_{1}+U_{2}=0\]
\[-U_{2}+U_{3}+U_{4}=0\]
\[U_{1}+U_{3}+U_{4}=0\]
These equations contain redundancy and the system can be written:
\[\left(\begin{array}{cccc}
  1&1&0&0\\
  0&-1&1&1\\
  \end{array}\right)U=0
  \]
  where U is the vector of tensions. The matrix is known as the loop matrix B.
  \subsubsection{General equations formulation}
  The physical behavior of an electrical circuit can be describe with the following equations:
  \[AI=0\]
  \[BU=0\]
  \[\textrm{For all branches :} \qquad F(U_{a},I_{a},...)=0 \]
\subsection{Definition}
\newtheorem{mur}{Def}
\begin{mur}
The branch is current-defined if its currents is a function of its own voltage, controlling variable
or their time--derivatives:
\begin{equation}\label{CD}I_{a}=F_{i}(U_{a},U_{b},I_{c},\frac{dU_a}{dt},\frac{dU_b}{dt},\frac{dI_{c}}{dt})\end{equation}
\end{mur}
Examples : \\
A resistor is a current-defined branch because $I_{a}=\frac{U_{a}}{R}$.\\
A capacitor is a current-defined branch because $I_{a}=C\frac{dU_{a}}{dt}$.\\
\begin{mur}
The branch is voltage-defined if its voltage is a function of its own current, controlling variable
or their derivatives:
\begin{equation}\label{VD}U_{a}=F_{u}(I_{a},U_{b},I_{c},\frac{dU_a}{dt},\frac{dU_b}{dt},\frac{dI_{c}}{dt})\end{equation}
\end{mur}
Examples : \\
A resistor is a voltage-defined branch because $U_{a}=RI_{a}$.\\
A inductor is a voltage-defined branch because $U_{a}=L\frac{dI_{a}}{dt}$.\\
\subsection{Hypothesis}
The M.N.A. assumes smooth branches are explicit functions of current or voltage. It means each branch is either Voltage Defined (V.D.) or Current Defined (C.D.)\\
\subsection{Unknowns}
The M.N.A. use the following unknowns:
\begin{enumerate}
\item Nodal voltages
\item Currents in the V.D. branches
\item Capacitor's charges and currents
\item Inductor's flux and currents
\item Currents control
\end{enumerate}
These unknowns are the state variables which are assumed to be sufficient to describe the state of the circuit.
\subsection{Equations}
The M.N.A. use following equations:


Current from current-defined branch is replaced with relation~(\ref{CD}). The result is a linear relation between system's unknowns.
\subsubsection{Law in voltage-defined branches (LVD)}
It consists in replacing $U_{a}$ with $V_{i}-V_{j}$ in the relation \ref{VD} and we obtain a linear relation between system's unknowns.
\[V_{i}-V_{j}=F_{i}(I_{a},U_{b},I_{c},\frac{dU_a}{dt},\frac{dU_b}{dt},\frac{dI_{c}}{dt})\]
\subsubsection{Capacitor laws (CAP)}
In a capacitor branch, the voltage is defined by 
\begin{equation}
 q_{a}=CU_{a} 
\end{equation}

A first relation between capacitor charge and nodal tension can be written:\\
\begin{equation}
 q_{a}=C(V_{i}-V_{j})\label{eq:CAP1}\tag{CAP1}
\end{equation}

A second dynamic relation is obtained
\begin{equation}
I_{a}=\frac{dq_{a}}{dt} \label{eq:CAP2}\tag{CAP2}
\end{equation}


After a time discretisation, these equations give to linear relations between system's unknowns.
\subsubsection{Inductor laws (IND)}
A relation between inductor flux and current is defined by
\begin{equation}
\psi _{a}=LI_{a}\label{eq:IND1}\tag{IND1}
\end{equation}
and the  dynamic relation is given by
\begin{equation}
  \label{eq:IND2}
   V_{i}-V_{j}=\frac{d\psi _{a}}{dt} \tag{IND2}
\end{equation}
After a time discretization procedure, these equations give tow linear relations between system's unknowns.
\newpage
\subsection{A simple MNA example}
\begin{figure}[h]
\centerline{
 \scalebox{0.6}{
    \frame
{
\newtheorem{mur}{A Current-Defined Branch (CD)}
\begin{mur}
The branch is current-defined if its current is a function of its own voltage, controlling variable
or their time--derivatives:
\begin{equation}\label{CD}I_{a}=F_{i}(U_{a},U_{b},I_{c},\frac{dU_a}{dt},\frac{dU_b}{dt},\frac{dI_{c}}{dt})\end{equation}
\end{mur}
\newtheorem{mur_}{A Voltage-Defined Branch (VD)}
\begin{mur_}
The branch is voltage-defined if its voltage is a function of its own current, controlling variable
or their derivatives:
\begin{equation}\label{VD}U_{a}=F_{u}(I_{a},U_{b},I_{c},\frac{dI_a}{dt},\frac{dU_b}{dt},\frac{dI_{c}}{dt})\end{equation}
\end{mur_}
Examples : \\
A resistor is a voltage-defined branch because $U_{a}=RI_{a}$.\\
A inductor is a voltage-defined branch because $U_{a}=L\frac{dI_{a}}{dt}$.\\
A capacitor is a current-defined branch because $I_{a}=C\frac{dU_{a}}{dt}$.\\

 \begin{block}{MNA Hypothesis:}
The M.N.A. assumes that smooth branches are explicit functions of current or voltage. It means each
branch is either Voltage Defined or Current Defined.
  \end{block}
}
\frame
{
\frametitle{MNA unknowns}
 \begin{block}{The vector of unknowns contains:}
\begin{itemize}
\item All node potentials $V_{i}$
\item Currents through all voltage defined branches
\end{itemize}
\end{block}

 \begin{block}{The following equations are written:}
\begin{itemize}
\item The KCL is applied to every node.
\item The Branch Constitutive Equation for all voltage defined branches.
\end{itemize}
\end{block}
Example:
  \begin{figure}[h]
   \centerline{
   \scalebox{0.5}{
    \input{LC.pstex_t}
  }
 } 
 \end{figure}

The vector of unknowns is :$(V_{1},I_{L})^{t}$
\[CV_{1}'-I_{L}=0 \qquad LI_{L}'+V_{1}=0\]

}
 }
}
\caption{Circuit for the MNA example}
\label{fig-MNA-example}
\end{figure}
\subsubsection{Topology: Branches and nodes definition}
The first step consists in deciding witch branches are voltage defined and witch are current
defined. For example, a resistor is a simple branch that can be solved either for the current I=U/R or for the
voltage U=RI. \\
After, the list of unknowns can be done. \\
Finally, the table equations is filled with the physical equations. 

 \textcolor{red}{\tt TBW}
\subsubsection{Branches analysis}

\begin{enumerate}
\item Branch 1 is voltage defined.
\item Branch 2 is looked as current defined ($I_{2}=\frac{U_{2}}{R}$).
\item Branch 3 is current defined.
\item Branch 4 is voltage defined.
\item Branch 5 is looked as voltage defined ($U_{5}=RIU_{5}$).
\end{enumerate}
\subsubsection{Unknowns}
\begin{enumerate}
\item The voltage nodes : $V_{1}$,$V_{2}$,$V_{3}$.
\item $I_{1}$,$I_{4}$,$I_{5}$, because the branches 1,4 and 5 are voltage defined.
\item $q_{3}$ and $I_{3}$, because the branch 3 is a capacitor.
\item $\psi _{4}$ from the inductor branch
\end{enumerate}
Therefore the unknowns vector is:\\
($V_{1}$,$V_{2}$,$V_{3}$,$I_{1}$,$I_{3}$,$I_{4}$,$I_{5}$,$q_{3}$,$\psi _{4}$)
\subsubsection{Table Equations}
\[\left(\begin{array}{ccccccccccc}
  V_{1}&V_{2}&V_{3}&I_{1}&I_{3}&I_{4}&I_{5}&q_{3}&\psi _{4}\\
  \hline
  \left(\begin{array}{c} KCL1 \end{array}\right)&\frac{-1}{R_{2}}&\frac{1}{R_{2}}&0&-1&0&0&0&0&0\\
  \left(\begin{array}{c} KCL2 \end{array}\right)&\frac{1}{R_{2}}&\frac{-1}{R_{2}}&0&0&-1&0&0&0&0\\
  \left(\begin{array}{c} KCL3 \end{array}\right)&0&0&0&0&0&1&-1&0&0\\
  \left(\begin{array}{c} CAP1 \end{array}\right)&0&C_{3}&0&0&0&0&0&-1&0\\
  \left(\begin{array}{c} IND1 \end{array}\right)&0&0&0&0&0&L_{4}&0&0&-1\\
  \left(\begin{array}{c} VDL5 \end{array}\right)&0&0&\frac{-1}{R_{5}}&0&0&0&1&0&0\\
  \left(\begin{array}{c} VDL1 \end{array}\right)&1&0&0&0&0&0&0&0&0\\
  \left(\begin{array}{c} CAP2 \end{array}\right)&0&0&0&0&1&0&0&\frac{-1}{h}&0\\
  \left(\begin{array}{c} IND2 \end{array}\right)&0&1&0&0&0&0&0&0&\frac{-1}{h}\\
\end{array}\right) \left(\begin{array}{c}
 V_{1}(t+h)\\
 V_{2}(t+h)\\
 V_{3}(t+h)\\
 I_{1}(t+h)\\
 I_{3}(t+h)\\
 I_{4}(t+h)\\
 I_{5}(t+h)\\
 q_{3}(t+h)\\
 \psi _{4}(t+h)
  \end{array}\right)=
\left(\begin{array}{c}
  RSH\\
  \hline
  0\\
  0\\
  0\\
  0\\
  0\\
  0\\
  0\\
  U_{1}(t+h)\\
  0\\
  \frac{-q_{3}(t)}{h}\\
  \frac{-\psi_{4}(t)}{h}\\
\end{array}\right)\]
Each time step consists in solving a system 9x9.


\subsection{Stamp method}
The stamp method is an algorithmic method used to fill the table equation from the components. It
consists in writing a sub-table for each type of component, this sub-table is the contribution of the component in the tableau equation.\\
Following, there are stamp examples.
\subsubsection{Resistor stamp}
\[\left(\begin{array}{cccc}
&V_{i}&V_{j}&RSH\\
  \hline
  KCL(i)&\frac{-1}{R}&\frac{1}{R}&\\
  KCL(j)&\frac{1}{R}&\frac{-1}{R}&\\
  \end{array}\right)
\]
Where R is the branch's resistance.
\subsubsection{Conductance stamp}
\[\left(\begin{array}{ccccc}
&V_{i}&V_{j}&I_{a}&RSH\\
  \hline
  KCL(i)&&&1&\\
  KCL(j)&&&-1&\\
  LVD&G&-G&1&\\
  \end{array}\right)
\]
Where G is the branch's conductance.
\subsubsection{Voltage source stamp}
\[\left(\begin{array}{ccccc}
&V_{i}&V_{j}&I_{a}&RSH\\
  \hline
  KCL(i)&&&1\\
  KCL(j)&&&-1\\
  LVD&1&-1&&E
  \end{array}\right)
\]
\subsubsection{Current controlled voltage source stamp}
\[\left(\begin{array}{cccccc}
&V_{i}&V_{j}&I_{a}&I_{b}&RSH\\
  \hline
  KCL(i)&&&1&\\
  KCL(j)&&&-1&\\
  LVD&1&-1&&\gamma
  \end{array}\right)
\]
With $U_{a} = \gamma I_{b}$.


%%% Local Variables: 
%%% mode: latex
%%% TeX-master: "ace"
%%% End: 
