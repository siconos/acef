\documentclass[10pt]{article}


%% Symbole de fraction
\newcommand{\Frac}[2]{{\displaystyle \frac{\displaystyle #1}{\displaystyle #2}}}
\newcommand{\Prac}[2]{\displaystyle \genfrac{(}{)}{}{}{\displaystyle #1}{\displaystyle #2}}
\newcommand{\Crac}[2]{\displaystyle \genfrac{[}{]}{}{}{\displaystyle #1}{\displaystyle #2}}

\newcommand{\norme}[1]{\|#1\|}


\newcommand{\HRule}{\rule{\linewidth}{1mm}}

% Fonction math�matiques

\newcommand{\transposee}[1]{{\vphantom{#1}}^{\text{\tiny{\textsf T}}}{#1}}
\newcommand{\argmin}{\mathop{\mathrm{argmin}}}
\newcommand{\argminn}{\mathop{\mathrm{argmin}}}
\newcommand{\lexicomin}{\mathop{\mathrm{lexicomin}}}
%\newcommand{\arg}{\mathop{\mathrm{arg}}}



\DeclareMathOperator{\rot}{rot}
\DeclareMathOperator{\sh}{sh}
\DeclareMathOperator{\ch}{ch}
%\DeclareMathOperator{\th}{th}
\DeclareMathOperator{\arcsh}{arcsh}
\DeclareMathOperator{\argth}{argth}
\DeclareMathOperator{\sign}{sign}


%%The Principal Value Integral symbol
\def\Xint#1{\mathchoice
   {\XXint\displaystyle\textstyle{#1}}%
   {\XXint\textstyle\scriptstyle{#1}}%
   {\XXint\scriptstyle\scriptscriptstyle{#1}}%
   {\XXint\scriptscriptstyle\scriptscriptstyle{#1}}%
   \!\int}
\def\XXint#1#2#3{{\setbox0=\hbox{$#1{#2#3}{\int}$}
     \vcenter{\hbox{$#2#3$}}\kern-.5\wd0}}
\def\ddashint{\Xint=}
\def\dashint{\Xint-}



% macro pour les symbols d'ensemble
%\nbOne
\def\nbOne{{\mathchoice{\rm 1\mskip-4mu l}{\rm 1\mskip-4mu l} {\rm 1 \mskip-4.5mu l}{\rm 1\mskip-5mu l}}}
%
%%  Les ensembles de nombres  C. Fiorio (fiorio�at�math.tu-berlin.de) 
%
\def\nbR{\ensuremath{\mathrm{I\!R}}} % IR
\def\nbN{\ensuremath{\mathrm{I\!N}}} % IN
\def\nbF{\ensuremath{\mathrm{I\!F}}} % IF
\def\nbH{\ensuremath{\mathrm{I\!H}}} % IH
\def\nbK{\ensuremath{\mathrm{I\!K}}} % IK
\def\nbL{\ensuremath{\mathrm{I\!L}}} % IL
\def\nbM{\ensuremath{\mathrm{I\!M}}} % IM
\def\nbP{\ensuremath{\mathrm{I\!P}}} % IP
%
% \nbOne : 1I : symbol one
\def\nbOne{{\mathchoice {\rm 1\mskip-4mu l} {\rm 1\mskip-4mu l}
{\rm 1\mskip-4.5mu l} {\rm 1\mskip-5mu l}}}
%
% \nbC   :  Nombres Complexes
\def\nbC{{\mathchoice {\setbox0=\hbox{$\displaystyle\rm C$}%
\hbox{\hbox to0pt{\kern0.4\wd0\vrule height0.9\ht0\hss}\box0}}
{\setbox0=\hbox{$\textstyle\rm C$}\hbox{\hbox
to0pt{\kern0.4\wd0\vrule height0.9\ht0\hss}\box0}}
{\setbox0=\hbox{$\scriptstyle\rm C$}\hbox{\hbox
to0pt{\kern0.4\wd0\vrule height0.9\ht0\hss}\box0}}
{\setbox0=\hbox{$\scriptscriptstyle\rm C$}\hbox{\hbox
to0pt{\kern0.4\wd0\vrule height0.9\ht0\hss}\box0}}}}
%
% \nbQ   : Nombres Rationnels Q
\def\nbQ{{\mathchoice {\setbox0=\hbox{$\displaystyle\rm
Q$}\hbox{\raise
0.15\ht0\hbox to0pt{\kern0.4\wd0\vrule height0.8\ht0\hss}\box0}}
{\setbox0=\hbox{$\textstyle\rm Q$}\hbox{\raise
0.15\ht0\hbox to0pt{\kern0.4\wd0\vrule height0.8\ht0\hss}\box0}}
{\setbox0=\hbox{$\scriptstyle\rm Q$}\hbox{\raise
0.15\ht0\hbox to0pt{\kern0.4\wd0\vrule height0.7\ht0\hss}\box0}}
{\setbox0=\hbox{$\scriptscriptstyle\rm Q$}\hbox{\raise
0.15\ht0\hbox to0pt{\kern0.4\wd0\vrule height0.7\ht0\hss}\box0}}}}
%
% \nbT   : T
\def\nbT{{\mathchoice {\setbox0=\hbox{$\displaystyle\rm
T$}\hbox{\hbox to0pt{\kern0.3\wd0\vrule height0.9\ht0\hss}\box0}}
{\setbox0=\hbox{$\textstyle\rm T$}\hbox{\hbox
to0pt{\kern0.3\wd0\vrule height0.9\ht0\hss}\box0}}
{\setbox0=\hbox{$\scriptstyle\rm T$}\hbox{\hbox
to0pt{\kern0.3\wd0\vrule height0.9\ht0\hss}\box0}}
{\setbox0=\hbox{$\scriptscriptstyle\rm T$}\hbox{\hbox
to0pt{\kern0.3\wd0\vrule height0.9\ht0\hss}\box0}}}}
%
% \nbS   : S
\def\nbS{{\mathchoice
{\setbox0=\hbox{$\displaystyle     \rm S$}\hbox{\raise0.5\ht0%
\hbox to0pt{\kern0.35\wd0\vrule height0.45\ht0\hss}\hbox
to0pt{\kern0.55\wd0\vrule height0.5\ht0\hss}\box0}}
{\setbox0=\hbox{$\textstyle        \rm S$}\hbox{\raise0.5\ht0%
\hbox to0pt{\kern0.35\wd0\vrule height0.45\ht0\hss}\hbox
to0pt{\kern0.55\wd0\vrule height0.5\ht0\hss}\box0}}
{\setbox0=\hbox{$\scriptstyle      \rm S$}\hbox{\raise0.5\ht0%
\hboxto0pt{\kern0.35\wd0\vrule height0.45\ht0\hss}\raise0.05\ht0%
\hbox to0pt{\kern0.5\wd0\vrule height0.45\ht0\hss}\box0}}
{\setbox0=\hbox{$\scriptscriptstyle\rm S$}\hbox{\raise0.5\ht0%
\hboxto0pt{\kern0.4\wd0\vrule height0.45\ht0\hss}\raise0.05\ht0%
\hbox to0pt{\kern0.55\wd0\vrule height0.45\ht0\hss}\box0}}}}
%
% \nbZ   : Entiers Relatifs Z
\def\nbZ{{\mathchoice {\hbox{$\sf\textstyle Z\kern-0.4em Z$}}
{\hbox{$\sf\textstyle Z\kern-0.4em Z$}}
{\hbox{$\sf\scriptstyle Z\kern-0.3em Z$}}
{\hbox{$\sf\scriptscriptstyle Z\kern-0.2em Z$}}}}
%%%% fin macro %%%%



\newcommand{\putidx}[1]{\index{#1}\textit{#1}}
% macro pour r�f�rencer les �quations

\newcommand{\refeq}[1]{(\ref{#1})}
\newcommand{\reffig}[1]{({\it cf} figure : \ref{#1})}
\newcommand{\refann}[1]{({\it cf} Annexe : \ref{#1})}


%\definecolor{darkgray}{gray}{.25}
%\definecolor{gray}{gray}{.5}
%\definecolor{lightgray}{gray}{.75}
%\definecolor{gradbegin}{rgb}{0,1,1}
%\definecolor{gradend}{rgb}{0,.1,.95}
%\newcommand{\newtexte}[1]{\textcolor{darkgray} {#1}}
\newcommand{\newtexte}[1]{{#1}}% macro pour les varibales favorites
% normal tangent
\def\n{{\hbox{\tiny{N}}}}
\def\t{{\hbox{\tiny{T}}}}
\def\ss{{\hbox{\tiny{S}}}}
\def\nt{\hbox{\tiny{NT}}}
\def\nsf{\hbox{\tiny{\textsf N}}}
\def\tsf{\hbox{\tiny{\textsf T}}}
\def\sigman{\sigma_{\n}}
\def\sigmat{\sigma_{\t}}
\def\sigmant{\sigma_{\nt}}
\def\epsn{\epsilon_{\n}}
\def\epst{\epsilon_{\t}}
\def\epsnt{\epsilon_{\nt}}
\def\eps{\epsilon}
\def\veps{\varepsilon}
\def\sig{\sigma}
\def\Rn{R_{\n}}
\def\Rt{R_{\t}}
\def\cn{c_{\n}}
\def\Cn{C_{\n}}
\def\ct{c_{\t}}
\def\Ct{C_{\t}}
\def\un{u_{\n}}
\def\ut{\buu_{\t}}
\def\uut{u_{\t}}
\def\unc{u_{\n}^c}
\def\utc{\buu_{\t}^c}
\def\vn{v_{\n}}
\def\vt{v_{\t}}
\def\rr{\hbox{\tiny{\textsf R}}}
\def\irr{\hbox{\tiny{\textsf{IR}}}}
\def\rn{r_{\n}}
\def\rt{\brr_{\t}}
\def\rnc{r_{\n}^c}
\def\rtc{\brr_{\t}^c}
\def\trn{\Tilde{r}_{\n}}
\def\trt{\Tilde{\brr}_{\t}}
\def\tr{\Tilde{\brr}}
\def\tv{\Tilde{\bvv}}
\def\vn{v_{\n}}
\def\vt{\bvv_{\t}}
\def\adh{\mathsf{adh}}
\def\adj{\hbox{\tiny{\textsf{adj}}}}
\def\adjc{\hbox{\tiny{\textsf{adjC}}}}
\def\adja{\hbox{\tiny{\textsf{adjA}}}}
\def\cc{\hbox{\tiny{\textsf C}}}
\def\ca{\hbox{\tiny{\textsf A}}}
%%    Unit�e
\def\mm{\,\mathsf{mm}}
\def\cm{\,\mathsf{cm}}
\def\m{\,\mathsf{m}}
\def\ms{\,\mathsf{m.s^{-1}}}
\def\mms{\,\mathsf{mm.s^{-1}}}
\def\Mpa{\,\mathsf{MPa}}
\def\Gpa{\,\mathsf{GPa}}
\def\Kg{\,\mathsf{Kg}}
\def\Hz{\,\mathsf{Hz}}
\def\kHz{\,\mathsf{kHz}}
\def\N{\,\mathsf{N}}
\def\kN{\,\mathsf{kN}}
\def\Nmmm{\,\mathsf{N.m^{-3}}}
\def\ds{d_{\hbox{\tiny{S}}}}
% domaines et frontieres
\def\om{\Omega}
\def\oma{\Omega^{\alpha}}
\def\omu{\Omega^1\cup \Omega^2}
\def\gc{\Gamma_c}
\def\omt{\omu \cup \gc}
% derivee partielle et gradient et divergence
\def\p{\partial}
\def\grad{\nabla}
\def\div{\mathop{\rm div}\nolimits}
%

%\DeclareTextSymbol{\deg}{T1}{6}
%\def\degre{\mathdegree}
%\newcommand{\degre}{\mathdegree}

\def\etc{\textit{etc}\ldots}
\newcommand{\mdegre}{\hbox{\text{\degre}}}

%\def\nscd{\textsf{\bfseries NSCD}}
%\def\nscd{\textsf{NSCD}}
\newcommand{\nscd}{\textsf{NSCD}}
%\Pisymbol{psy}{212} ou encore \Pisymbol{psy}{228}




%----------------------------------------------------------------------
%             Des chiffres avec des ronds autour
%----------------------------------------------------------------------
\def\nombrecercle#1{\def\taille{0.3}
                \put(0,0){#1}
                \put(0.08,0.08){\circle{\taille}}}



\def\ae#1{\stackrel{\mbox{\scriptsize a.e.}}{#1}}
\def\argmin{\mathop{\rm argmin}}
\def\eqref#1{{\rm (\ref{#1})\/}}
\def\indicfon{\mathord{\rm i}}       %indicator function
\def\p{\mathord{\rm proj}}
\def\N{\mathord{\rm N}}
% \def\prosca#1#2{#1\cdot#2}
\def\prosca#1#2{\langle #1,#2\rangle}
\def\qedtext{\mbox{}\hfill$\Box$}
\def\qedmath{\eqno\Box}

\def\s{{$\mathcal{S}$}}
\def\somme{\mathop{\textstyle\sum}}
\def\somme{\mathop{\textstyle\sum}}
\def\submoins{_{\scriptscriptstyle-}}
\def\subplus{_{\scriptscriptstyle+}}
\def\T{\mathord{\rm T}}

%----------------------------------------------------------------------
%             Macro M Jean 
%----------------------------------------------------------------------

\def\Real{\mbox{I\hspace{-.15em}R}}
\def\Integer{\mbox{I\hspace{-.15em}N}}
\def\Bunit{\mbox{I\hspace{-.15em}B}}
\def\real{\mbox{\scriptsize I\hspace{-.15em}R}}
\def\bunit{\mbox{\scriptsize I\hspace{-.15em}B}}
\def\IL{\mbox{\scriptsize I\hspace{-.15em}L}}
\def\Indic{\mbox{\large $\psi$}}
\def\bfxi{\mbox{$\xi$ \hspace{-1.1em} $\xi$}}
%\def\bfXi{\mbox{$\Xi$ \hspace{-1.1em} $\Xi$}}
\def\RunR{\mathcal R}
\def\RunRN{\mathcal R_{N}}
\def\RunRT{\mathcal R_{T}}
\def\RunS{\mathcal S}
\def\RunSN{\mathcal S_{N}}
\def\RunST{\mathcal S_{T}}
\def\RunU{\mathcal U}
\def\RunUN{\mathcal U_{N}}
\def\RunUT{\mathcal U_{T}}
\def\RunUP{\mathcal U'}
\def\RunUPN{\mathcal U'_{N}}
\def\RunUPT{\mathcal U'_{T}}
\def\RunJ{\mathcal J}
\def\RunW{\mathcal W}
\def\RunF{f}
\def\RunFa{f_{1}}
\def\RunFb{f_{2}}
\def\RunFP{f'}
\def\RunV{v}
\def\RunVP{v'}
\def\EspF{\mathcal F}
\def\EspV{\mathcal V}
%%%%
\catcode`\�=13
\def�{\'e}
\catcode`\�=13
\def�{\`e}
\catcode`\�=13
\def�{\`a}
\catcode`\�=13
\def�{\c c}
\def\N{\mbox{I\hspace{ -.15em}N}}
\def\Z{\mbox{Z\hspace{ -.3em}Z}}
\def\Q{\mbox{l\hspace{ -.47em}Q}}
\def\R{\mbox{l\hspace{ -.15em}R}}
\def\F{\mbox{l\hspace{ -.15em}F}}
\def\E{\mbox{l\hspace{ -.15em}E}}
\def\LMGC90{{\small \it LMGC90 }}
\def\NSCD{{\small \it NSCD }}
\def\CHIC{{\small \it CHIC }}
\def\half{{\frac{_{1}}{^{2}}}}
\def\12T{{\frac{_{1}}{^{2T}}}}

\def\geq{\geqslant}
\def\leq{\leqslant}
\def\ge{\geqslant}
\def\le{\leqslant}


\begingroup
\count0=\time \divide\count0by60 % Hour
\count2=\count0 \multiply\count2by-60 \advance\count2by\time
% Min
\def\2#1{\ifnum#1<10 0\fi\the#1}
\xdef\isodayandtime{\the\year-\2\month-\2\day\space\2{\count0}:%
\2{\count2}}
\endgroup

%---------------------------------------------------------------------
%             Redaction note environnement B. Brogliato
%----------------------------------------------------------------------
\makeatletter

{\newtheorem{ndr1bb}{\textbf{\textsc{Redaction note B.B.}}}[section]}

\newenvironment{ndrbb}%
{%
\noindent\begin{ndr1bb}\hrule\vspace{1em}%
\ttfamily\small
}%
{%
\begin{flushright}%
%\vspace{-1.5em}\ding{111}
\end{flushright}%
\vspace{-1.5em}\hrule
\end{ndr1bb}%
}

%---------------------------------------------------------------------
%             Redaction note environnement O. Bonnefon
%----------------------------------------------------------------------

{\newtheorem{ndr1ob}{\textbf{\textsc{Redaction note O.B.}}}[section]}

\newenvironment{ndrob}%
{%
\noindent\begin{ndr1ob}\hrule\vspace{1em}%
\ttfamily\small
}%
{%
\begin{flushright}%
%\vspace{-1.5em}\ding{111}
\end{flushright}%
\vspace{-1.5em}\hrule
\end{ndr1ob}%
}

%----------------------------------------------------------------------
%             Redaction note environnement V.ACARY
%----------------------------------------------------------------------
% Faut etre fou pour s'amuser a pondre des notes pareilles

{\newtheorem{ndr1va}{\textbf{\textsc{Redaction note V.A.}}}[section]}

\newenvironment{ndrva}%
{%
\noindent\begin{ndr1va}\hrule\vspace{1em}%
\ttfamily\small \  \\
\indent}%
{%
\begin{flushright}%
\  \\
%\vspace{-1.5em}\ding{111}
\end{flushright}%
\vspace{-1.5em}\hrule
\end{ndr1va}%
}
\makeatother






% ----------------DEFINITIONS-----------------
% 

 \def\II{\mathop{{\rm I}\mskip-3.0mu{\rm I}}\nolimits}




% -----------------------------------
 \def\c{\mathop{{\rm 1}\mskip-10.0mu{\rm C}}\nolimits}
 \def\C{\mathop{{\rm 1}\mskip-10.0mu{\rm C}}\nolimits}
 \def\ZZ{\mathaccent23Z}
% 
 \def\abstract{
 \footnotesize\quotation \noindent {\bf Abstract.}}
% 

\newcommand{\ie}{{\textit{i.e.}}}


%\def\sgn{\mbox{\rm sgn}}
\DeclareMathOperator{\sgn}{sgn}
\DeclareMathOperator{\proj}{proj}

\newcommand{\RR}{\mbox{\rm $I\!\!R$}}
\newcommand{\NN}{\mbox{\rm $I\!\!N$}}



% ---------------- MMC -----------------
% 

\newcommand{\contract}{{\,:\,}}

\newcommand{\scontract}{{\,{\Bar\otimes}\,}}
\newcommand{\tcontract}{{\,{\Bar{\Bar{\Bar\otimes}}}\,}}


\newcommand{\DP}[2]{\displaystyle \frac{\partial {#1}}{\partial {#2}}}


\usepackage{pifont}
\makeatletter
\def\cqfd{\ifmmode\sqw\else{\ifhmode\unskip\fi\nobreak\hfil
\penalty50\hskip1em\null\nobreak\hfil\ding{111}
\parfillskip=0pt\finalhyphendemerits=0\endgraf}\fi}
\makeatother
\def\off{{\hbox{\tiny{\textsf{off}}}}}
\def\on{{\hbox{\tiny{\textsf{on}}}}}
\def\pwl{{\hbox{\tiny{\textsf{pwl}}}}}

%%% Local Variables: 
%%% mode: latex
%%% TeX-master: "book"
%%% x-symbol-coding: iso-8859-2
%%% End: 

\usepackage{psfrag}
\usepackage{fancyhdr}
\usepackage{subfigure}
%\renewcommand{\baselinestretch}{1.2}
\textheight 23cm
\textwidth 16cm
\topmargin 0cm
%\evensidemargin 0cm
\oddsidemargin 0cm
\evensidemargin 0cm
\usepackage{layout}
\usepackage{mathpple}
\makeatletter
\renewcommand\bibsection{\paragraph{References
     \@mkboth{\MakeUppercase{\bibname}}{\MakeUppercase{\bibname}}}}
\makeatother
%% style des entetes et des pieds de page
\fancyhf{} % nettoie le entetes et les pieds
\fancyhead[L]{Template 6 : Electrical oscillator with 4 diodes bridge full-wave rectifier - Pascal Denoyelle}
%\fancyhead[C]{}%
\fancyhead[R]{\thepage}
%\fancyfoot[L]{\resizebox{!}{0.7cm}{\includegraphics[clip]{logoesm2.eps}}}%
\fancyfoot[C]{}%
 \begin{document}
 \title{Automatic circuit equations formulation}

\maketitle

\newpage
\tableofcontents
 \newpage
 \section{Notations}
\begin{itemize}
\item[--] C.D. : Current Defined
\item[--] V.D. : Voltage Defined
\item[--] M.N.A. : Modify Nodal Analysis
\end{itemize}
 \begin{itemize}
\item[--] $N_{n}$ : Number of nodes.
\item[--] $N_{b}$ : Number of branches.
\item[--] $I_{j}$ : Current in the branch number j.
\item[--] $V_{i}$ : Voltage on node number i. 
\end{itemize}

 \section{Non smooth model}
Each non smooth electrical component has a physical model:
\[\beta=B\lambda + A\alpha +a\]
\[y=C\alpha + D\lambda +e\]
The complementary condition:
\[0 \leq y \, \perp \, \lambda \geq 0\]
$\lambda$ and y are intermediate variables describing the branch's piecewise linear model.
$\lambda$ and y are tow vectors which ave the same dimension.\\
\newline
$\beta$ and $\alpha$ are the physical vectors variables of the component.\\
\newpage
%***********************************MNA ANALYSIS
\section{Modify Nodal Analysis and complementary conditions}
\subsection{Hypothesis\\}
The M.N.A. assumes smooth branches are explicit functions of current or voltage. It means each smooth
branch is V.D. or C.D.\\

With the same assumption, all branches can be divided into following classes:\\
\begin{enumerate}
\item The  \underline{C.D. branches} where\underline{ I is not a controlling current}: Indexes form 1 to $n_{c}$\\
  Let $N_{ci}$=$n_{c}$
\item The  \underline{C.D.  branches} where\underline{ I is a controlling current} (in smooth or non smooth branches!): Indexes form $N_{ci}$+1 to $N_{ci}$+$n_{ci}$\\
  Let $N_{v}$=$N_{ci}$+$n_{ci}$
\item The  \underline{V.D. branches}: Indexes form $N_{v}$+1 to $N_{v}$+$n_{v}$ \\
  Let $N_{r}$=$N_{v}$+$n_{v}$
 \item The  \underline{non smooth branches}: Indexes form $N_{r}$+1 to $N_{r}$+$n_{r}$\\
  Let $N_{u}$=$N_{r}$+$n_{r}$
 \item The \underline{dynamical capacitor branches}: Indexes form $N_{u}$+1 to $N_{u}$+$n_{u}$\\
  Let $N_{i}$=$N_{u}$+$n_{u}$
 \item The \underline{dynamical inductors branches}: Indexes form $N_{i}$+1 to $N_{b}$ \\
\end{enumerate}


\newpage
\subsection{Unknowns\\}
\begin{enumerate}
\item Nodal voltages\\
$V_{0}$ is the reference (=0).\\
$V_{1}$ to $V_{N_{n}}$ : $N_{n}$ - 1 unknowns.
\item Currents in the V.D. branches\\
  $I_{N_{v}+1}$ to  $I_{N_{v} + n_{v}}$ : $n_{v}$ unknowns.
\item Currents in the non smooth branches(see NB3 )\\
  $I_{N_{r}+1}$ to  $I_{N_{r} + n_{r}}$ : $n_{r}$ unknowns.
\item Currents in the inductor branches\\
  $I_{N_{i}+1}$ to  $I_{N_{b} }$ : $n_{i}$ unknowns.

\end{enumerate}
Number of unknowns = $N_{I} = N_{n} - 1 + N_{b} - n_{ci} - n_{u}$\\
Let X = $^{t}(V_{1}...V_{N_{n}},I_{N_{ci}+1}...I_{N_{ci} + n_{ci}},I_{N_{v}+1}...I_{N_{v} + n_{v}},I_{N_{r}+1}...I_{N_{r} + n_{r}},I_{N_{i}+1}...I_{N_{b}}$)\\
Let J = \{$N_{ci}+1...N_{ci} + n_{ci},N_{v}+1...N_{v} + n_{v},N_{r}+1...N_{r} +n_{r},N_{i}+1...N_{b}$\}\\
\underline{NB1:} $\forall a \notin J, I_{a} = \sum_{i}^{}a_{i}V_{i} + \sum_{j\in J}^{}b_{j}I_{j}
+\sum_{i}^{}c_{i}V'_{i}$\\
\underline{NB2:} Controlling currents are optional.\\
\underline{NB3:} Optional if the non smooth model give $I_{a} = \sum_{i}^{}a_{i}V_{i} + \sum_{j\in J}^{}b_{j}I_{j} +\sum_{i}^{}c_{i}\lambda_{i} $
\newpage
\subsection{Smooth equations\\}
\begin{enumerate}
  \item The Kirchhoff Current Law (KCL) : $N_{n}$ - 1 equations.\\
    NB : The current in the capacitor branches is written with :
    \[I = C*\frac{d(V_{i} - V_{j})}{dt}\]
  \item The Controlling Current Equation (CCE) :

     \[I=\sum_{i}^{}a_{i}V_{i}\ + \sum_{j \in J}^{}b_{j}I_{j}  +  \sum_{i}^{}c_{i}V'_{i} + source\]
     This equation comes from the branch constitution, not from Kirchoff. (ex : I = $\frac{U}{R}$),
     it uses to write the KCL laws.
  \item The V.D. equation (VDE) : $n_{v}$ equations.\\
    A V.D. branch give an equation :
    \[V_{i}-V_{j} = \sum_{i}^{}a_{i}V_{i} + \sum_{j\in J}^{}b_{j}I_{j} +  \sum_{i}^{}c_{i}V'_{i} +
    source\]
    This equation comes from the branch constitution, not from Kirchoff. (ex : U = R*I)
  \item The inductor law (IL) : $n_{i}$ equations.
     \[V_{i} - V_{j} = L*\frac{dI}{dt}\]

\end{enumerate}
Number of equations = $N_{E}$ =$N_{I}$ - $n_{r}$
\subsection{Non smooth branches}

For each non smooth branch, we write a complementarity condition:\\
\[\left(\begin{array}{c}
H_{i}X = B_{i}l+A_{i}X + a_{i}\\
y=C_{i}X+D_{i}l+e_{i}\\
0 \leq y \, \perp \, l \geq 0
\end{array}\right)\]

\newpage
\subsection{Matrices formulation}
We obtain the Circuit Equation :
\[CX'+AX = s\]
or with x the dynamic variables extracted from X, and z the non dynamic variables from X:
\[\left(\begin{array}{c}
x'=Bx+Cz+s1\\
0 =Dx+Ez+s2
\end{array}\right)\]
For each non smooth branch:
\[\left(\begin{array}{c}
H_{i}X = B_{i}l+A_{i}X + a_{i}\\
y=C_{i}X+D_{i}l+e_{i}\\
0 \leq y \, \perp \, l \geq 0
\end{array}\right)\]
There are $n_{r}$ complementary conditions.
\newline
With A :\\
\[
\left( \begin{array}{ccccc}
V_{1}...V_{N_{n}} & I_{N_{ci}+1}...I_{N_{ci} + n_{ci}} & I_{N_{v}+1}...I_{N_{v} +
  n_{v}}&I_{N_{r}+1}...I_{N_{r} + n_{r}} & I_{N_{i}+1}...I_{N_{i} + n_{i}} \\
\hline
KCL & 0& KCL & KCL&KCL\\
CCE & I & CCE & CCE&CCE\\
VDE & VDE & VDE & VDE&VDE\\
IL&0&0&0&0
\end{array} \right)
\left( \begin{array}{c}
right hand\\
\hline
s(current)\\
s\\
s(tension)\\
0

  \end{array}\right)
\]
And C :
\[
\left( \begin{array}{ccccc}
V_{1}...V_{N_{n}} & I_{N_{ci}+1}...I_{N_{ci} + n_{ci}} & I_{N_{v}+1}...I_{N_{v} +
  n_{v}}&I_{N_{r}+1}...I_{N_{r} + n_{r}} & I_{N_{i}+1}...I_{N_{i} + n_{i}} \\
\hline
KCL & 0& 0 & 0&0\\
CCE & 0 & 0 & 0&0\\
VDE & 0 & 0 & 0&0\\
0&0&0&0&IL
\end{array} \right)
\]

\subsection{conclusion}
The number of unknowns is equal at the number of constraints(linear constraints or complementary
constraints), so it must be possible to solve it???
\newpage
\section{Substitution}

This section describe how modify the previous analysis to be able to solve the system.\\

\subsection{Added unknowns}

It consists to build a $Z_{ns}$ vector contains all variables useful to describe non smooth
branches. X contains all useful currents (see X's def) but some tensions could be add. 
$Z_{ns}$ is composed with new unknown tensions and some currents \underline{extracted} from X. The $Z_{ns}$ vector comes from $\beta$ only.\\
  let dim($Z_{ns}$) = $n_{ns}$\\
 We add the voltage Kirchoff laws for the added unkwown tensions : $U=V_{j}-V_{i}$

\subsection{Non smooth model instance}

\paragraph{Local formulation}
For each non smooth branch, we write a complementarity condition:\\
\[\left(\begin{array}{c}
\beta = z_{i} = B_{i}l+A_{i}X + a_{i}\\
y=C_{i}X+D_{i}l+e_{i}\\
0 \leq y \, \perp \, l \geq 0
\end{array}\right)\]
Where $z_{i}$ is a voltage and currents vector.





\paragraph{Global formulation}
The global complementary condition is written from local conditions :\\
Let $Z_{ns}$ = $^{t}(z_{1},...,z_{n_{r}}$)\\
Let Y = $^{t}(y_{1},...,y_{n_{r}}$)\\
Let $\lambda$ =$^{t}(l_{1},...,l_{n_{r}}$)\\
\[\left(\begin{array}{c}
Z_{ns} = B\lambda +AX + a\\
Y=CX+D\lambda +e\\
0 \leq Y \, \perp \, \lambda \geq 0
\end{array}\right)\]

\subsection{Matrices formulation}
Before go head, we rename the variables.\\
x contains only the dynamic variables(currents in inductor and tensions from capacitor branches)\\
$Z_{s}$ a vector extract from X, $Z_{s}$ contains only the non dynamic variables(Voltage nodes,... ).\\

We replace X by x,$Z_{s}$,$Z_{ns}$:
\[\left(\begin{array}{c}
x'=A_{1x}x +A_{1s}Z_{s} + A_{1ns}Z_{ns}\\
0=B_{1x}x+B_{1s}Z_{s} + B_{1ns}Z_{ns}\\
Z_{ns}= C_{1x}x+C_{1s}Z_{s}+C_{1\lambda}\lambda \\
Y=D_{1x}x +D_{1s}Z_{s}+D_{1ns}Z_{ns}+D_{1\lambda}\lambda\\
0 \leq Y \, \perp \, \lambda \geq 0
\end{array}\right)\]
Substitute $Z_{ns}$:
\[\left(\begin{array}{cc}
R=A_{1ns}C_{\lambda}\lambda &(eq1)\\
x'=(A_{1x}+A_{1ns}C_{x})x +(A_{1s}+A_{1ns}C_{1s})Z_{s} +R=A_{2x}x +A_{2s}Z_{s} +R&(eq2)\\
0=(B_{1x}+B_{1ns}C_{1x})x+(B_{1s}+B_{1ns}C_{1s})Z_{s} + B_{1ns}C_{1\lambda}\lambda = B_{2x}x+B_{2s}Z_{s} + C_{2\lambda}\lambda&(eq3)\\
Y=(D_{1x}+D_{1ns}C_{1x})x+(D_{1s}+D_{1ns}C_{1s})Z_{s}+(D_{1\lambda}+D_{1ns}C_{1\lambda})\lambda = D_{2x}x+D_{2s}Z_{s}+D_{2\lambda}\lambda &(eq4)\\
0 \leq Y \, \perp \, \lambda \geq 0&(\perp)\\
\end{array}\right)\]
\newpage

\subsection{A time discretisation}

 eq1 discretisation:
\[x(t_{i+1}) - x(t_{i})=h\theta A_{2x}x(t_{i+1})+h(1-\theta)A_{2x}x(t_{i}) +h\theta A_{2s}Z_{s}(t_{i+1}) +h(1-\theta)A_{2s}Z_s(t_{i}) +hR(t_{i+1})\]
We assume $(I-h\theta A_{2x})$ is regular,$W(I-h\theta A_{2x}) = I.$
\[x(t_{i+1})=Wx_{free}+h\theta WA_{2s}Z_{s}(t_{i+1})+hWR(t_{i+1}) = Wx_{free}+h\theta
WA_{2s}Z_{s}(t_{i+1}) 
+hWA_{ns}C_{\lambda}\lambda(t_{i+1}) (eq5)\]
eq3 discretisation:
\[0 = B_{2x}x(t_{i+1})+B_{2s}Z_{s}(t_{i+1}) + C_{2\lambda}\lambda(t_{i+1})\]
With eq5:
\[0 = B_{2x}Wx_{free}+(h\theta B_{2x} WA_{2s}+B_{2s})
Z_{s}(t_{i+1})+(hB_{2x}WA_{ns}C_{\lambda}+C_{2\lambda})\lambda(t_{i+1})\]
Rename the matrices:
\[0 = q_{free}+B_{3s} Z_{s}(t_{i+1})+B_{3\lambda}\lambda(t_{i+1})\]

eq4 discretisation:
\[Y(t_{i+1})=D_{2x}x(t_{i+1})+D_{2s}Z_{s}(t_{i+1}) +D_{2\lambda}\lambda(t_{i+1})\]
With eq5 and rename the matrices:
\[Y(t_{i+1})=D_{2x}Wx_{free}+D_{3s}Z_{s}(t_{i+1})+D_{3\lambda}\lambda(t_{i+1})\]
\[0 \leq Y(t_{i+1}) \, \perp \, \lambda(t_{i+1}) \geq 0\]

\paragraph{MLCP}
description:\\

\[W=\left(\begin{array}{c}W_{1}\\0\end{array}\right)\]
\[Z=\left(\begin{array}{c}Z_{1}\\Z_{2}\end{array}\right)\]
\[W=MZ+q\]
\[0 \leq W_{1} \, \perp \, Z_{1} \geq 0\]
\paragraph{MLCP instance}
We identify a MLCP:\\
\[W_{1} = Y(t_{i+1})\]
\[Z_{1} = \lambda(t_{i+1})\]
\[Z_{2} = Z_{s}(t_{i+1})\]
\[M = \left(\begin{array}{cc}
  WD_{2\lambda}&WD_{2s}\\
B_{3\lambda}&B_{3s}
\end{array}\right)\]
\newpage
\section{Non smooth formulation}
example with 5 non smooth components:\\


\[\left(\begin{array}{c}
  Z_{ns}\\
  \hline
  \left(\begin{array}{c} Z_{ns1} \end{array}\right)\\
  \left(\begin{array}{c} Z_{ns2} \end{array}\right)\\
  \left(\begin{array}{c} Z_{ns3} \end{array}\right)\\
  \left(\begin{array}{c} Z_{ns4} \end{array}\right)\\
  \left(\begin{array}{c} Z_{ns5} \end{array}\right)\\
\end{array}\right) =
\left(\begin{array}{c}
  C_{1x}\\
  \hline
  \left(\begin{array}{c}1\end{array}\right)\\
  \left(\begin{array}{c}2\end{array}\right)\\
  \left(\begin{array}{c}3\end{array}\right)\\
  \left(\begin{array}{c}4\end{array}\right)\\
  \left(\begin{array}{c}5\end{array}\right)\\
\end{array}\right)*x +
\left(\begin{array}{c}
  C_{1s}\\
  \hline
  \left(\begin{array}{c}1\end{array}\right)\\
  \left(\begin{array}{c}2\end{array}\right)\\
  \left(\begin{array}{c}3\end{array}\right)\\
  \left(\begin{array}{c}4\end{array}\right)\\
  \left(\begin{array}{c}5\end{array}\right)\\
\end{array}\right)*Z_{s} +
\left(\begin{array}{ccccc}
  C1_{\lambda}\\
  \hline
  B1&0&0&0&0\\
  0&B2&0&0&0\\
  0&0&B3&0&0\\
  0&0&0&B4&0\\
  0&0&0&0&B5\\
\end{array}\right)*
\left(\begin{array}{c}
  \lambda\\
  \hline
  \left(\begin{array}{c} \lambda _{1} \end{array}\right)\\
  \left(\begin{array}{c} \lambda _{2}\end{array}\right)\\
  \left(\begin{array}{c} \lambda _{3}\end{array}\right)\\
  \left(\begin{array}{c} \lambda _{4}\end{array}\right)\\
  \left(\begin{array}{c} \lambda _{5}\end{array}\right)\\
\end{array}\right)+
\left(\begin{array}{c}
  cst\\
  \hline
  \left(\begin{array}{c} cst_{1} \end{array}\right)\\
  \left(\begin{array}{c} cst_{2}\end{array}\right)\\
  \left(\begin{array}{c} cst_{3}\end{array}\right)\\
  \left(\begin{array}{c} cst_{4}\end{array}\right)\\
  \left(\begin{array}{c} cst_{5}\end{array}\right)\\
\end{array}\right)
\]

\[\left(\begin{array}{c}
  Y\\
  \hline
  \left(\begin{array}{c} Y_{1} \end{array}\right)\\
  \left(\begin{array}{c} Y_{2} \end{array}\right)\\
  \left(\begin{array}{c} Y_{3} \end{array}\right)\\
  \left(\begin{array}{c} Y_{4} \end{array}\right)\\
  \left(\begin{array}{c} Y_{5} \end{array}\right)\\
\end{array}\right) =
\left(\begin{array}{c}
  D_{1x}\\
  \hline
  \left(\begin{array}{c}1\end{array}\right)\\
  \left(\begin{array}{c}2\end{array}\right)\\
  \left(\begin{array}{c}3\end{array}\right)\\
  \left(\begin{array}{c}4\end{array}\right)\\
  \left(\begin{array}{c}5\end{array}\right)\\
\end{array}\right)*x +
\left(\begin{array}{c}
  D_{1s}\\
  \hline
  \left(\begin{array}{c}1\end{array}\right)\\
  \left(\begin{array}{c}2\end{array}\right)\\
  \left(\begin{array}{c}3\end{array}\right)\\
  \left(\begin{array}{c}4\end{array}\right)\\
  \left(\begin{array}{c}5\end{array}\right)\\
\end{array}\right)*Z_{s} +
\left(\begin{array}{ccccc}
  D1_{\lambda}\\
  \hline
  B1&0&0&0&0\\
  0&B2&0&0&0\\
  0&0&B3&0&0\\
  0&0&0&B4&0\\
  0&0&0&0&B5\\
\end{array}\right)*
\left(\begin{array}{c}
  \lambda\\
  \hline
  \left(\begin{array}{c} \lambda _{1} \end{array}\right)\\
  \left(\begin{array}{c} \lambda _{2}\end{array}\right)\\
  \left(\begin{array}{c} \lambda _{3}\end{array}\right)\\
  \left(\begin{array}{c} \lambda _{4}\end{array}\right)\\
  \left(\begin{array}{c} \lambda _{5}\end{array}\right)\\
\end{array}\right)+
\left(\begin{array}{c}
  cst\\
  \hline
  \left(\begin{array}{c} cst_{1} \end{array}\right)\\
  \left(\begin{array}{c} cst_{2}\end{array}\right)\\
  \left(\begin{array}{c} cst_{3}\end{array}\right)\\
  \left(\begin{array}{c} cst_{4}\end{array}\right)\\
  \left(\begin{array}{c} cst_{5}\end{array}\right)\\
\end{array}\right)
\]

\newpage
\section{How get $x' = Ax+BZ_{s} + CZ_{ns}$?}
\subsection{Example 1}
\begin{figure}[h]
\centerline{
 \scalebox{0.5}{
    \input{cir1.pstex_t}
 }
}
\end{figure}
\paragraph{First matrices formulation}
X=$^{t}(V_{1},V_{2},V_{3})$\\
\[\left(\begin{array}{c}
  \\
  KCL(1)\\  KCL(2)\\  KCL(3)
  \end{array}\right)
\left(\begin{array}{ccc}
  V_{1}&V_{2}&V_{3}\\
  \hline
  \frac{1}{R}-\frac{1}{R}&  \frac{-1}{R}&0\\
  \frac{1}{R}&  \frac{-1}{R}&0\\
  0&0&\frac{1}{R}
\end{array}\right)X+
\underline{
\left(\begin{array}{ccc}
   V_{1}'&V_{2}'&V_{3}'\\
  \hline
0&0&0\\
  0&C&-C\\
  0&-C&C
\end{array}\right)}X'=
\left(\begin{array}{c}
  \\
  I\\
  0\\
  0
  \end{array}\right)
\]
But, we can't extract x from X to get x'=...\\
\newline
\paragraph{Add current and tension from the capacitor}
x=$(U_{32})$
$Z_{s}=^{t}(V_{1},V_{2},V_{3},I_{32})$\\
\[x'=CI_{32}\]
\[\left(\begin{array}{c}
  \\
  KCL(1)\\
  KCL(2)\\
  KCL(3)\\
  U_{32}
  \end{array}\right)
\left(\begin{array}{cccc}
  V_{1}&V_{2}&V_{3}&I_{32}\\
  \hline
  \frac{1}{R}-\frac{1}{R}&  \frac{-1}{R}&0&0\\
  \frac{1}{R}&  \frac{-1}{R}&0&1\\
  0&0&\frac{1}{R}&-1\\
  0&-1&1&0
\end{array}\right)Z_{s}+
\left(\begin{array}{c}
  U_{32}\\
  \hline
  0\\
  0\\
  0\\
  1
  \end{array}\right)x
=
\left(\begin{array}{c}
  \\
  I\\
  0\\
  0\\
  0
  \end{array}\right)
\]
We obtain the matrices system\\
\[x'=BZ_{s}\]
\[Ax+CZ_{s}=s\]
$N_{I}=N_{E}=5$\\
The following examples, show we don't need to add the capacitor current. Sometime, current in
capacitor can be replace with a KCL law.\\

\paragraph{Add only tension from the capacitor}
x=$(U_{32})$,$Z_{s}=^{t}(V_{1},V_{2},V_{3})$\\
First get x' with \underline{KCL(3)}:
\[x'=\frac{V_{3}}{RC}\]
Second, the current in capacitor is $\frac{V_{3}}{R}$, and use it to write other KCL law:
\[\left(\begin{array}{c}
  \\
  KCL(1)\\
  KCL(2)\\
  U_{32}
  \end{array}\right)
\left(\begin{array}{ccc}
V_{1}&V_{2}&V_{3}\\
  \hline
  \frac{1}{R}+\frac{1}{R}&  \frac{-1}{R}&0\\
  \frac{-1}{R}&  \frac{1}{R}&\frac{1}{R}\\
  0&-1&1
\end{array}\right)Z_{s}+
\left(\begin{array}{c}
U_{32}\\
  \hline
  0\\
  0\\
  0\\
  1
  \end{array}\right)x
=
\left(\begin{array}{c}
  \\
  I\\
  0\\
  0\\
  0
  \end{array}\right)
\]
$N_{I}=N_{E}=4$
\newpage
\subsection{Example 2}
\begin{figure}[h]
\centerline{
 \scalebox{0.9}{
    \input{cir2.pstex_t}
 }
}
\end{figure}
\paragraph{Add currents and tensions from the capacitor}
$x=^{t}(I_{42},U_{43},U_{31},U_{50})$,
$Z_{s}=^{t}(V_{1},V_{2},V_{3},V_{4},V_{5},I_{43},I_{31},I_{50})$
We obtain following equation:
\[\left(\begin{array}{cccc}
  I_{42}'&U_{43}'&U_{31}'&U_{50}'\\
  \hline
L&0&0&0\\
0&C&0&0\\
0&0&C&0\\
0&0&0&C
\end{array}\right)x'=
0x+
\left(\begin{array}{cccccccc}
  V_{1}&V_{2}&V_{3}&V_{4}&V_{5}&I_{43}&I_{31}&I_{50}\\
  \hline
  0&1&0&-1&0&0&0&0\\
  0&0&0&0&0&-1&0&0\\
  0&0&0&0&0&0&1&0\\
  0&0&0&0&0&0&0&1\\
\end{array}\right)Z_{s}
\]
\[\left(\begin{array}{c}
\\KCL(1)\\KCL(2)\\KCL(3)\\KCL(4)\\KCL(5)\\U_{43}\\U_{31}\\U_{50}
\end{array}\right)
\left(\begin{array}{cccccccc}
  V_{1}&V_{2}&V_{3}&V_{4}&V_{5}&I_{43}&I_{31}&I_{50}\\
  \hline
  -\frac{1}{R}&\frac{1}{R}&0&0&0&0&1&0\\
  \frac{1}{R}&-\frac{1}{R}&0&0&0&0&0&0\\
  0&0&0&0&0&-1&1&0\\
  0&0&0&-\frac{1}{R}&\frac{1}{R}&1&0&0\\
  0&0&0&\frac{1}{R}&-\frac{1}{R}&0&0&1\\
  0&0&-1&1&0&0&0&0\\
  -1&0&1&0&0&0&0&0\\
  0&0&0&0&1&0&0&0\\
\end{array}\right)Z_{s}+
\left(\begin{array}{cccc}
  I_{42}&U_{43}&U_{31}&U_{50}\\
  \hline
  0&0&0&0\\
  1&0&0&0\\
  0&0&0&0\\
  0&0&0&0\\
  1&0&0&0\\
  0&1&0&0\\
  0&0&1&0\\
  0&0&0&1\\
\end{array}\right)x=
\left(\begin{array}{c}
  \\ I\\  0\\  0\\  0\\  0\\  0\\  0\\  0
  \end{array}\right)
\]

\[x'=BZ_{s}\]
\[Cx+BZ_{s}=s\]
$N_{I}=N_{E}=12$
\paragraph{Add only tensions from the capacitor}
$x=^{t}(I_{42},U_{43},U_{31},U_{50})$,
$Z_{s}=^{t}(V_{1},V_{2},V_{3},V_{4},V_{5})$\\
We obtain following equation:
\[
\left(\begin{array}{c}
  \\  KCL(4)\\  KCL(3)\\  KCL(5)
\end{array}\right)
\underline{
\left(\begin{array}{cccc}
  I_{42}'&U_{43}'&U_{31}'&U_{50}'\\
  \hline
L&0&0&0\\
0&C&0&0\\
0&C&C&0\\
0&0&0&C
\end{array}\right)}x'=
\left(\begin{array}{cccc}
  I_{42}&U_{43}&U_{31}&U_{50}\\
  \hline
0&0&0&0\\
1&0&0&0\\
0&0&0&0\\
0&0&0&0\\ 
\end{array}\right)x+
\left(\begin{array}{ccccc}
V_{1}&V_{2}&V_{3}&V_{4}&V_{5}\\
  \hline
  0&1&0&-1&0\\
  0&0&0&\frac{1}{R}&-\frac{1}{R}\\
  0&0&0&0&0\\
  0&0&0&-\frac{1}{R}&\frac{1}{R}
\end{array}\right)Z_{s}
\]
So, we get $x'=Ax+BZ_{s}$. Therefore, all currents in the capacitor branch are known.\\
$I_{43}= I_{42}+\frac{V_{4}}{R}-\frac{V_{5}}{R}$,
$I_{31}= I_{42}+\frac{V_{4}}{R}-\frac{V_{5}}{R}$,
$I_{50}= \frac{V_{4}}{R}-\frac{V_{5}}{R}$\\
Use these equations to fill following matrices:
\[\left(\begin{array}{c}
KCL(1)\\KCL(2)\\U_{43}\\U_{31}\\U_{50}
\end{array}\right)
\left(\begin{array}{ccccc}
V_{1}&V_{2}&V_{3}&V_{4}&V_{5}\\
  \hline
  -\frac{1}{R}&\frac{1}{R}&0&\underline{\frac{1}{R}}&\underline{-\frac{1}{R}}\\
  \frac{1}{R}&-\frac{1}{R}&0&0&0\\
  0&0&-1&1&0\\
  -1&0&1&0&0\\
  0&0&0&0&1\\
\end{array}\right)Z_{s}+
\left(\begin{array}{cccc}
  I_{42}&U_{43}&U_{31}&U_{50}\\
  \hline
  \underline{1}&0&0&0\\
  1&0&0&0\\0&1&0&0\\0&0&1&0\\0&0&0&1\\
\end{array}\right)x=
\left(\begin{array}{c}
  I\\0\\0\\0\\0
  \end{array}\right)
\]
$N_{I}=N_{E}=9$
\subsection{Example 3}
\begin{figure}[h]
\centerline{
 \scalebox{0.8}{
    \input{cir3.pstex_t}
 }
}
\end{figure}

x=$(U_{12})$\\
$Z_{s}=^{t}(V_{1},V_{2},I_{10})$\\
\[(KCL(1))=>(C1+C2+C3)x'=I_{10}\]
So, all capacitor's currents are known:\\
$I_{c1}=\frac{C1}{C1+C2+C3}I_{10}$\\
$I_{c2}=\frac{C2}{C1+C2+C3}I_{10}$\\
$I_{c3}=\frac{C3}{C1+C2+C3}I_{10}$\\
Use these equations to fill following matrices:\\
\[\left(\begin{array}{c}
  \\
KCL(2)\\U_{21}\\VD
\end{array}\right)
\left(\begin{array}{c}
U_{21}\\
\hline
0\\
1\\
0
\end{array}\right)x+
\left(\begin{array}{ccc}
V_{1}&V_{2}&I_{10}\\
\hline
0&\frac{1}{R}&\frac{C1}{C1+C2+C3}+\frac{C2}{C1+C2+C3}+\frac{C3}{C1+C2+C3}\\
1&-1&0\\
1&0&0
\end{array}\right)Z_{s}=
\left(\begin{array}{c}
\\0\\0\\E
\end{array}\right)
\]
$N_{I}=N_{E}=4$
\newpage
\subsection{Example 4}
This example shows that is not always possible to use the KCL law to get x'=...\\
\begin{figure}[h]
\centerline{
 \scalebox{0.9}{
    \input{cir4.pstex_t}
 }
}\end{figure}\\
$x=^{t}(U_{21},U_{23},U_{34})$
$Z_{s}=^{t}(V_{1},V_{2},V_{3},V_{4})$
\paragraph{a problem}
Start to fill the x' matrices. We use KCL(2) for $U_{21}$, and KCL(3) for $U_{34}$.
\[\left(\begin{array}{c}
  \\
KCL(2)\\KCL(3)\\??
\end{array}\right)
\left(\begin{array}{ccc}
  U_{21}'&U_{23}'&U_{34}'\\
  \hline
  C&-C&0\\
  0&C&C\\
  ?&?&?
\end{array}\right)x'=0x+
\left(\begin{array}{cccc}
  V_{1}&V_{2}&V_{3}&V_{4}\\
  \hline
  0&0&0&0\\
  0&0&0&0\\
  ?&?&?&?\\
\end{array}\right)Z_{s}
 \]
 We can't use a other KCL law to get $U_{23}$. A solution could be to add an unknown, \underline{$I_{23}$} in
 $Z_{s}$. The system becomes:\\
 \[\left(\begin{array}{c}
  \\
KCL(2)\\KCL(3)\\I_{23}
\end{array}\right)
\left(\begin{array}{ccc}
  U_{21}'&U_{23}'&U_{34}'\\
  \hline
  C&-C&0\\
  0&C&C\\
  0&C&0
\end{array}\right)x'=0x+
\left(\begin{array}{ccccc}
  V_{1}&V_{2}&V_{3}&V_{4}&\underline{I_{23}}\\
  \hline
  0&0&0&0&0\\
  0&0&0&0&0\\
  0&0&0&0&1\\
\end{array}\right)Z_{s}
 \]
 So, we obtain $x'=AZ_{s}$. All capacitor's currents are known:
 $I_{21}=\underline{I_{23}}=I_{34}$\\
\[\left(\begin{array}{c}
  \\
KCL(4)\\KCL(1)\\U_{21}\\U_{23}\\U_{34}
\end{array}\right)
\left(\begin{array}{ccc}
  U_{21}&U_{23}&U_{34}\\
  \hline
  0&0&0\\
  0&0&0\\
  1&0&0\\
  0&1&0\\
  0&0&1
\end{array}\right)x+
\left(\begin{array}{cccc}
  V_{1}&V_{2}&V_{3}&\underline{I_{23}}\\
  \hline
  0&0&0&1\\
  \frac{1}{R}&0&0&1\\
  -1&1&0&0\\
  0&-1&1&0\\
  0&0&-1&1\\
\end{array}\right)Z_{s}=
\left(\begin{array}{c}
\\I\\0\\0\\0\\0
\end{array}\right)
\]
 $N_{I}=N_{E}=8$\\
\paragraph{a good choice}
\[\left(\begin{array}{c}
  \\
KCL(1)\\KCL(2)\\KCL(3)
\end{array}\right)
\left(\begin{array}{ccc}
  U_{21}'&U_{23}'&U_{34}'\\
  \hline
  C&0&0\\
  C&C&0\\
  0&-C&C
\end{array}\right)x'=0x+
\left(\begin{array}{cccc}
  V_{1}&V_{2}&V_{3}&V_{4}\\
  \hline
  \frac{1}{R}&-\frac{1}{R}&0&0\\
  0&0&0&0\\
  0&0&0&0\\
\end{array}\right)Z_{s}
 \]
 So, we obtain $x'=BZ_{s}$. Therefore all capacitor's currents are known: $I_{12}=I_{23}=I_{34}=\frac{V1}{R1}$\\
 \[\left(\begin{array}{c}
  \\
KCL(4)\\U_{21}\\U_{23}\\U_{34}
\end{array}\right)
 \left(\begin{array}{ccc}
  U_{21}&U_{23}&U_{24}\\
  \hline
  0&0&0\\
  1&0&0\\
  0&1&0\\
  0&0&1\\
\end{array}\right)x+
 \left(\begin{array}{cccc}
  V_{1}&V_{2}&V_{3}&V_{4}\\
  \hline
  \frac{1}{R}&0&0&0\\
  -1&1&0&0\\
  0&-1&1&0\\
  0&0&-1&1\\
\end{array}\right)Z_{s}=
 \left(\begin{array}{c}
  \\I\\0\\0\\0
\end{array}\right)
 \]
$N_{I}=N_{E}=7$\\
 \newpage
\subsection{Example 5}
This example shows we need the Minimum Spanning Tree of the capacitor' tension graph.\\
\begin{figure}[h]
\centerline{
 \scalebox{0.6}{
    \input{cir5.pstex_t}
 }
}\end{figure}\\
$x=^{t}(U_{12},U_{23},U_{34},U_{41})$$Z_{s}=(V_{1},V_{2},V_{3},V_{4},V_{5},I_{50})$\\
Start to write Ax'=...\
\[\left(\begin{array}{c}
  \\
KCL(1)\\KCL(2)\\KCL(3)\\KCL(4)
\end{array}\right)
\left(\begin{array}{cccc}
  U_{12}'&U_{23}'&U_{34}'&U_{41}'\\
  \hline
  C&0&0&-C\\
  -C&C&0&0\\
  0&-C&C&0\\
  0&0&-C&C\\  
\end{array}\right)x'=
\]
This matrix is not regular because of the cycle \{1-2,2-3,3-4,4-1\}. A solution could be to use the Minimum
Spanning Tree \{1-2,2-3,3-4\} to write the KCL law. About the last tension, $U_{41}$, there are tow
ways:
\begin{enumerate}
\item add a unknown $I_{41} in Z_{S}$ and write CU'=I
\item Find the linear relation $U_{41}'= \sum_{jk}^{}a_{jk}U_{kj}', and replace U_{ki}'$.
\end{enumerate}
The matrices become:\\
Start to write Ax'=...\
\[\left(\begin{array}{c}
  \\
KCL(1)\\KCL(2)\\KCL(3)\\I_{41}
\end{array}\right)
\left(\begin{array}{cccc}
  U_{12}'&U_{23}'&U_{34}'&U_{41}'\\
  \hline
  C&0&0&-C\\
  -C&C&0&0\\
  0&-C&C&0\\
  0&0&0&C\\  
\end{array}\right)x'=0x+
\left(\begin{array}{ccccccc}
  V_{1}&V_{2}&V_{3}&V_{4}&V_{5}&I_{50}&I_{41}\\
  \hline
  -\frac{1}{R}&0&0&0&0&0&0\\
  0&0&0&0&0&0&0\\
  0&0&0&\frac{1}{R}&-\frac{1}{R}&0&0\\
  0&0&0&0&0&0&1\\
\end{array}\right)Z_{s}
\]
So, we obtain $x'=BZ_{s}$. Therefore all capacitor's currents are known:$I_{12}=I_{23}=I_{41}+\frac{V1}{R},I_{43}=I_{41}+\frac{V1}{R}+\frac{V5}{R}-\frac{V3}{R}$\\
 \[\left(\begin{array}{c}
  \\
KCL(4)\\KCL(5)\\U_{12}\\U_{23}\\U_{34}\\U_{41}\\VD_{50}
\end{array}\right)
 \left(\begin{array}{cccc}
  U_{12}&U_{23}&U_{34}&U_{41}\\
  \hline
  0&0&0&0\\
  0&0&0&0\\
  1&0&0&0\\
  0&1&0&0\\
  0&0&1&0\\
  0&0&0&1\\
  0&0&0&0\\

\end{array}\right)x+
 \left(\begin{array}{ccccccc}
  V_{1}&V_{2}&V_{3}&V_{4}&V_{5}&I_{50}&I_{41}\\
  \hline
  \frac{1}{R}&0&-\frac{1}{R}&0&\frac{1}{R}&0&\underline{-1+1}\\
  0&0&\frac{1}{R}&0&-\frac{1}{R}&-1&0\\
  -1&1&0&0&0&0&0\\
  0&-1&1&0&0&0&0\\
  0&0&-1&1&0&0&0\\
  0&0&0&-1&1&0&0\\
  0&0&0&0&1&0&0\\
\end{array}\right)Z_{s}=
 \left(\begin{array}{c}
  \\0\\0\\0\\0\\0\\0\\E
\end{array}\right)
 \]
$N_{I}=N_{E}=11$\\
 \newpage
\subsection{conclusion}
x contains inductor's currents and capacitor's tensions. Derivate inductor's current is equal to a
nodal voltage difference.\\
About the capacitor's tensions, we can use the following algorithm:\\


\begin{algorithm}
\caption{fill the matrices : $x'=A_{1x}x+A_{1s}Z_{s}+A_{1ns}Z_{ns}$ }
\begin{algorithmic}
\REQUIRE Init\_I\_in\_x : initialize internal data structure to get all I from x.
\REQUIRE Next\_I\_in\_x : return the next available I from x. If there are not available x, return 0.\\
\REQUIRE Minimum Spanning Tree of the capacitor' tension graph
\REQUIRE Init\_MST : initialize internal data structure to get all u from x.
\REQUIRE Next\_u\_in\_MST : return a available U's neighbour from MST if possible. If there are not
available u in MST, return 0.\\
\REQUIRE Next\_U\_in\_x : return the next available U from x. If there are not available x, return 0.\\

0\\
\COMMENT{About current}\\
\COMMENT{get first current}
\STATE Init\_I\_in\_x()
\STATE $I_{kj}$ = Next\_I\_in\_x()
\WHILE{$I_{kj}$}
\STATE use $LI'=V_{j}-V_{k}$ to fill $I_{kj}$'s line.
\ENDWHILE\\
\COMMENT{About tension}\\
\COMMENT{get a first capacitor tension}
\STATE Init\_MST()
\STATE $U_{kj}$ = Next\_u\_in\_MST()
\WHILE{$U_{kj}$}
\STATE l=j or k with KCL(k) available.
\STATE Use CU'=I and KCL(k) to fill $U_{ki}$'s line.\\
\STATE enable KCL(k)

\STATE $U_{kj}$= Next\_u\_in\_MST ()
\ENDWHILE
\STATE $U_{kj}$ = Next\_U\_in\_x()
\WHILE {$U_{kj}$}
\STATE Add an unknown $I_{kj}$, and use it to fill the matrices. Write I=CU'.
\STATE $U_{kj}$ = Next\_U\_in\_x()
\ENDWHILE
\COMMENT{reverse A : $Ax'=Bx+CZ_{s}+DZ_{ns}$\\}
\STATE $A^{-1}$=Inv(A)
\STATE $x'=A_{1x}x+A_{1s}Z_{s}+A_{1ns}Z_{ns}$
\end{algorithmic}
\end{algorithm}
I don't prove A is regular. But each A's line comes from a independent physical laws, so I assume A
is regular.
\newpage

\section{Buck converter}
\begin{figure}[h]
\centerline{
 \scalebox{1.0}{
    \input{buck.pstex_t}
 }
}\end{figure}
$x=^{t}(U_{45},U_{67},U_{08},I_{18}) Z_{s}=^{t}(V_{1},V_{2},V_{3},V_{4},V_{5},V_{6},V_{7},V_{8},V_{9},V_{10},I_{90},I_{20},I_{30},I_{100},I_{40})
Z_{ns}=^{t}(I_{1},I_{2},I_{3},I_{4},U_{2})$\\
\underline{$Ax'= Bx+CZ_{s}$}
\[\left(\begin{array}{c}
  \\
KCL(4)\\KCL(7)\\KCL(8)\\I_{18}
\end{array}\right)
\left(\begin{array}{cccc}
  U_{45}'&U_{67}'&U_{08}'&I_{18}'\\
  \hline
  C21&0&0&0\\
  0&-C11&0&0\\
  0&0&C&0\\
  0&0&0&L\\  
\end{array}\right)x'=
\left(\begin{array}{cccc}
  U_{45}&U_{67}&U_{08}&I_{18}\\
  \hline
  0&0&0&0\\
  0&0&0&0\\
  0&0&0&1\\
  0&0&0&0\\
\end{array}\right)x+\]
\[
\left(\begin{array}{ccccccccccccccc}
  V_{1}&V_{2}&V_{3}&V_{4}&V_{5}&V_{6}&V_{7}&V_{8}&V_{9}&V_{10}&I_{90}&I_{20}&I_{30}&I_{100}&I_{40}\\
  \hline
  0&0&0&0&0&0&0&0&0&0&0&0&0&0&-1\\
  0&0&0&0&0&0&-\frac{1}{R11}&\frac{1}{R11}&0&0&0&0&0&0&0\\
  0&0&0&0&0&\frac{1}{R12}&\frac{1}{R11}&\frac{-1}{Rload}-\frac{1}{R11}-\frac{1}{R12}&0&0&0&0&0&0&0\\
  1&0&0&0&0&0&0&0&0&0&0&0&0&0&0
\end{array}\right)Z_{s}+0Z_{ns}\]
\underline{$Ex+FZ_{s}+GZ_{ns}=s$}
\[
\left(\begin{array}{c}
  \\
KCL(1)\\KCL(2)\\KCL(3)\\KCL(5)\\KCL(6)\\KCL(9)\\KCL(10)\\VD_{100}\\VD_{2}\\VD_{30}\\VD_{90}\\VD_{40}\\U_{45}\\U_{67}\\U_{08}
\end{array}\right)
\left(\begin{array}{cccc}
  U_{45}&U_{67}&U_{08}&I_{18}\\
  \hline
  0&0&0&-1\\
  0&0&0&0\\
  0&0&0&0\\
  0&0&0&0\\
  0&0&0&0\\
  0&0&0&0\\
  0&0&0&0\\
  0&0&0&0\\
  0&0&0&0\\
  0&0&0&0\\
  0&0&0&0\\
  0&0&0&0\\
  1&0&0&0\\
  0&1&0&0\\
  0&0&1&0\\
\end{array}\right)x+\]
\[
\left(\begin{array}{ccccccccccccccc}
  V_{1}&V_{2}&V_{3}&V_{4}&V_{5}&V_{6}&V_{7}&V_{8}&V_{9}&V_{10}&I_{90}&I_{20}&I_{30}&I_{100}&I_{40}\\
  \hline
  0&0&0&0&0&0&0&0&0&0&0&0&0&0&0\\
  0&0&0&0&0&0&0&0&0&0&0&1&0&0&0\\
  0&0&0&0&0&0&0&0&0&0&0&0&1&0&0\\
  0&0&0&0&-\frac{1}{R21}&\frac{1}{R21}&0&0&0&0&0&0&0&0&-1\\
  0&0&0&0&\frac{1}{R21}&-\frac{1}{R12}-\frac{1}{R21}&\frac{1}{R11}&\frac{1}{R12}-\frac{1}{R11}&0&0&0&0&0&0&0\\
  0&0&0&0&0&0&0&0&0&0&1&0&0&0&0\\
  0&0&0&0&0&0&0&0&0&0&0&0&0&-1&0\\
  0&0&0&0&0&0&0&0&0&1&0&0&0&0&0\\
  0&1&0&0&0&0&0&0&0&0&0&0&0&0&0\\
  0&0&1&0&0&0&0&0&0&0&0&0&0&0&0\\
  0&0&0&0&0&0&0&0&1&0&0&0&0&0&0\\
  0&0&0&1&0&-G2&0&0&G2&0&0&0&0&0&0\\
  0&0&0&-1&1&0&0&0&0&0&0&0&0&0&0\\
  0&0&0&0&0&-1&1&0&0&0&0&0&0&0&0\\
  0&0&0&0&0&0&0&1&0&0&0&0&0&0&0\\
\end{array}\right)Z_{s}+\]
\[
\left(\begin{array}{ccccc}
  I_{1}&I_{2}&I_{3}&I_{4}&U_{2}\\
  \hline
  1&-1&-1&1&0\\
  0&0&0&0&0\\
  0&0&0&0&0\\
  0&0&0&0&0\\
  0&0&0&0&0\\
  0&0&0&0&0\\
  0&1&1&0&0\\
  0&0&0&0&0\\
  0&0&0&0&-1\\
  0&0&0&0&0\\
  0&0&0&0&0\\
  0&0&0&0&0\\
  0&0&0&0&0\\
  0&0&0&0&0\\
  0&0&0&0&0\\
\end{array}\right)Z_{ns}=
\left(\begin{array}{c}
  \\
0\\0\\0\\0\\0\\0\\0\\V1\\0\\Vramp\\Vref\\0\\0\\0\\0
\end{array}\right)\]
$N_{I}=N_{E}-4=24$
\newpage
And the non smooth equations:dim(Y)=dim($\lambda$ )=10.\\
\underline{$Z_{ns}=B\lambda$}\\
With B=
\tiny
\[
\left(\begin{array}{cccccccccccccccccccccccc}
l1&l2&l3&l4&l5&l6&l7&l8&l9&l10&l11&l12&l13&l14&l15&l16&l17&l18&l19&l20&l21&l22&l23&l24\\
\hline
0.09&0.22&0.46&1.16&2.89&-0.09&-0.22&-0.46&-1.16&-2.89&0&0&0&0&0&0&0&0&0&0&0&0&0&0\\
0&0&0&0&0&0&0&0&0&0&1&0&0&0&0&0&0&0&0&0&0&0&0&0\\
0&0&0&0&0&0&0&0&0&0&0&1&0&0&0&0&0&0&0&0&0&0&0&0\\
0&0&0&0&0&0&0&0&0&0&0&0&0.09&0.22&0.46&1.16&2.89&-0.09&-0.22&-0.46&-1.16&-2.89&0&0\\
0&0&0&0&0&0&0&0&0&0&0&0&0&0&0&0&0&0&0&0&0&0&25&-25
\end{array}\right)
\]
\small
\underline{$Y=D\lambda$+Ex+s}
\[\left(\begin{array}{c}
  \\Y1\\Y2\\Y3\\Y4\\Y5\\Y6\\Y7\\Y8\\Y9\\Y10\\Y11\\Y12\\Y13\\Y14\\Y15\\Y16\\Y17\\Y18\\Y19\\Y20\\Y21\\Y22\\Y23\\Y24
 \end{array}\right)=
\left(\begin{array}{cccccccccccccccccccccccc}
  \\
  \hline
1&0&0&0&0&0&0&0&0&0&0&0&0&0&0&0&0&0&0&0&0&0&0&0\\
0&1&0&0&0&0&0&0&0&0&0&0&0&0&0&0&0&0&0&0&0&0&0&0\\
0&0&1&0&0&0&0&0&0&0&0&0&0&0&0&0&0&0&0&0&0&0&0&0\\
0&0&0&1&0&0&0&0&0&0&0&0&0&0&0&0&0&0&0&0&0&0&0&0\\
0&0&0&0&1&0&0&0&0&0&0&0&0&0&0&0&0&0&0&0&0&0&0&0\\
0&0&0&0&0&1&0&0&0&0&0&0&0&0&0&0&0&0&0&0&0&0&0&0\\
0&0&0&0&0&0&1&0&0&0&0&0&0&0&0&0&0&0&0&0&0&0&0&0\\
0&0&0&0&0&0&0&1&0&0&0&0&0&0&0&0&0&0&0&0&0&0&0&0\\
0&0&0&0&0&0&0&0&1&0&0&0&0&0&0&0&0&0&0&0&0&0&0&0\\
0&0&0&0&0&0&0&0&0&1&0&0&0&0&0&0&0&0&0&0&0&0&0&0\\
0&0&0&0&0&0&0&0&0&0&0&0&0&0&0&0&0&0&0&0&0&0&0&0\\
0&0&0&0&0&0&0&0&0&0&0&0&0&0&0&0&0&0&0&0&0&0&0&0\\
0&0&0&0&0&0&0&0&0&0&0&0&1&0&0&0&0&0&0&0&0&0&0&0\\
0&0&0&0&0&0&0&0&0&0&0&0&0&1&0&0&0&0&0&0&0&0&0&0\\
0&0&0&0&0&0&0&0&0&0&0&0&0&0&1&0&0&0&0&0&0&0&0&0\\
0&0&0&0&0&0&0&0&0&0&0&0&0&0&0&1&0&0&0&0&0&0&0&0\\
0&0&0&0&0&0&0&0&0&0&0&0&0&0&0&0&1&0&0&0&0&0&0&0\\
0&0&0&0&0&0&0&0&0&0&0&0&0&0&0&0&0&1&0&0&0&0&0&0\\
0&0&0&0&0&0&0&0&0&0&0&0&0&0&0&0&0&0&1&0&0&0&0&0\\
0&0&0&0&0&0&0&0&0&0&0&0&0&0&0&0&0&0&0&1&0&0&0&0\\
0&0&0&0&0&0&0&0&0&0&0&0&0&0&0&0&0&0&0&0&1&0&0&0\\
0&0&0&0&0&0&0&0&0&0&0&0&0&0&0&0&0&0&0&0&0&1&0&0\\
0&0&0&0&0&0&0&0&0&0&0&0&0&0&0&0&0&0&0&0&0&0&1&0\\
0&0&0&0&0&0&0&0&0&0&0&0&0&0&0&0&0&0&0&0&0&0&0&1
\end{array}\right)\lambda+\]
\[
\left(\begin{array}{ccccccccccccccc}
  V_{1}&V_{2}&V_{3}&V_{4}&V_{5}&V_{6}&V_{7}&V_{8}&V_{9}&V_{10}&I_{90}&I_{20}&I_{30}&I_{100}&I_{40}\\
  \hline
  0&-b&0&0&0&0&0&0&0&3&0&0&0&0&0\\
  0&-b&0&0&0&0&0&0&0&3&0&0&0&0&0\\
  0&-b&0&0&0&0&0&0&0&3&0&0&0&0&0\\
  0&-b&0&0&0&0&0&0&0&3&0&0&0&0&0\\
  0&-b&0&0&0&0&0&0&0&3&0&0&0&0&0\\
  b&-b&0&0&0&0&0&0&0&0&0&0&0&0&0\\
  b&-b&0&0&0&0&0&0&0&0&0&0&0&0&0\\
  b&-b&0&0&0&0&0&0&0&0&0&0&0&0&0\\
  b&-b&0&0&0&0&0&0&0&0&0&0&0&0&0\\
  b&-b&0&0&0&0&0&0&0&0&0&0&0&0&0\\
  1&0&0&0&0&0&0&0&0&0&0&0&0&0&0\\
  -1&0&0&0&0&0&0&0&0&1&0&0&0&0&0\\
  0&-b&0&0&0&0&0&0&0&3&0&0&0&0&0\\
  0&-b&0&0&0&0&0&0&0&3&0&0&0&0&0\\
  0&-b&0&0&0&0&0&0&0&3&0&0&0&0&0\\
  0&-b&0&0&0&0&0&0&0&3&0&0&0&0&0\\
  0&-b&0&0&0&0&0&0&0&3&0&0&0&0&0\\
  b&-b&0&0&0&0&0&0&0&0&0&0&0&0&0\\
  b&-b&0&0&0&0&0&0&0&0&0&0&0&0&0\\
  b&-b&0&0&0&0&0&0&0&0&0&0&0&0&0\\
  b&-b&0&0&0&0&0&0&0&0&0&0&0&0&0\\
  b&-b&0&0&0&0&0&0&0&0&0&0&0&0&0\\
  0&0&-1&1&0&0&0&0&0&0&0&0&0&0&0\\
  0&0&-1&1&0&0&0&0&0&0&0&0&0&0&0\\
\end{array}\right)Z_{s}=
\left(\begin{array}{c}
  \\h1\\h2\\h3\\h4\\h5\\h6\\h7\\h8\\h9\\h10\\0\\0\\h1\\h2\\h3\\h4\\h5\\h6\\h7\\h8\\h9\\h10\\-0.1\\0.1
  \end{array}\right)
  \]
\normalsize
\newpage
%*************************************OTHER ANALYSIS

%EXAMPLE SECTION************************************************************************************
\section{Diodes bridge}

\begin{figure}[h]
\centerline{
 \scalebox{0.7}{
    \input{Bridge.pstex_t}
 }
}\end{figure}

\subsection{Smooth Analyze}
$N_{b} = 7 $
$N_{n} = 4$\\
$n_{c}$=1\\
$n_{ci}$=0\\
$n_{v}$=0\\
$n_{r}$=4\\
$n_{u}$ =1\\
$n_{i}$=1\\




\subsection{Unknowns}

x = $^{t}(U_{1},I_{7})$,
$Z_{ns}=^{t}(I_{2},I_{3},I_{4},I_{5}$),
$Z_{s} = ^{t}(V_{2},V_{3})$
\subsection{Diode non smooth model instance}

\[ \beta = z_{i} = I_{i}\]
\[ \alpha =U_{i}\]
\[z_{i}=1l_{i}+0\alpha\]
\[y_{i}=1\alpha+0l_{i}\]
\[0 \leq y_{i} \, \perp \, l_{i} \geq 0\]


\subsection{Global formulation}

\[ \lambda =(l_{1},l_{2},l_{3},l_{4})\]
\[Z_{ns}=0X+0Z_{s}+Id\lambda\]
\[Y=\left(\begin{array}{cc}
0&0\\
0&0\\
1&0\\
1&0\end{array}\right) x+
\left(\begin{array}{cc}
1&0\\
0&1\\
0&-1\\
-1&0\end{array}
\right) Z_{s} + 0Z_{ns} +0\lambda\]

\[0 \leq Y \, \perp \, \lambda \geq 0\]



\subsection{Matrices formulation}
\[
\left(\begin{array}{c}
  
x'=\left(\begin{array}{cc}
0 &\frac{-1}{C}\\
\frac{1}{L}&0\end{array} \right)x
+\left(\begin{array}{cc}
0&\frac{-1}{C}\\
0&0\end{array} \right)Z_{s}
+\left(\begin{array}{cccc}
0&0&\frac{1}{C}&\frac{-1}{C}\\
0&0&0&0\end{array} \right)Z_{ns}
\\
0=\left(\begin{array}{cc}
0 &0\\
0 &0\\
0 &1\end{array} \right)x
+\left(\begin{array}{cc}
\frac{-1}{R}&\frac{1}{R}\\
\frac{-1}{R}&\frac{1}{R}\\
1&0\end{array} \right)Z_{s}
+\left(\begin{array}{cccc}
1&0&0&1\\
0&+1&1&0\\
0&0&0&0\end{array} \right)Z_{ns}
\\
Z_{ns}=Dx+0Z_{s}+Id\lambda\\
Y=\left(\begin{array}{cc}
0&0\\
0&0\\
1&0\\
1&0\end{array}\right) x+
\left(\begin{array}{cc}
1&0\\
0&1\\
0&-1\\
-1&0\end{array}
\right) Z_{s} + 0Z_{ns} +0\lambda\\

0 \leq Y \, \perp \, \lambda \geq 0

\end{array}
\right)
\]

\subsection{AN}
C=0.01 F\\
L=1/120 H\\
R=33 ohm

\[
\left(\begin{array}{c}
  
x'=\left(\begin{array}{cc}
0 &-100\\
120&0\end{array} \right)x
+\left(\begin{array}{cc}
0&-100\\
0&0\end{array} \right)Z_{s}
+\left(\begin{array}{cccc}
0&0&-100&-100\\
0&0&0&0\end{array} \right)Z_{ns}
\\
0=\left(\begin{array}{cc}
0 &0\\
0 &0\\
0 &1\end{array} \right)x
+\left(\begin{array}{cc}
-0.03&0.03\\
-0.03&0.03\\
1 &0\end{array} \right)Z_{s}
+\left(\begin{array}{cccc}
1&0&0&1\\
0&+1&1&0\\
0&0&0&0\end{array} \right)Z_{ns}
\\
Z_{ns}=0x+0Z_{s}+Id\lambda\\
Y=\left(\begin{array}{cc}
0&0\\
0&0\\
1&0\\
1&0\end{array}\right) x+
\left(\begin{array}{cc}
1&0\\
0&1\\
0&-1\\
-1&0\end{array}
\right) Z_{s} + 0Z_{ns} +0\lambda\\

0 \leq Y \, \perp \, \lambda \geq 0

\end{array}
\right)
\]
ie
\[
\left(\begin{array}{c}
R=\left(\begin{array}{cccc}
0&0&-100&-100\\
0&0&0&0\end{array} \right) \lambda\\
  
x'=\left(\begin{array}{cc}
0 &-100\\
120&0\end{array} \right)x
+\left(\begin{array}{cc}
0&-100\\
0&0\end{array} \right)Z_{s}+R\\
0=\left(\begin{array}{cc}
0 &0\\
0 &0\\
0 &1\\\end{array} \right)x
+\left(\begin{array}{cc}
-0.03&0.03\\
-0.03&0.03\\
1&0\end{array} \right)Z_{s}
+\left(\begin{array}{cccc}
1&0&0&1\\
0&-1&-1&0\\
0&0&0&0\end{array} \right)\lambda\\
Y=\left(\begin{array}{cc}
0&0\\
0&0\\
1&0\\
1&0\end{array}\right) x+
\left(\begin{array}{cc}
1&0\\
0&1\\
0&-1\\
-1&0\end{array}
\right) Z_{s} +0\lambda\\

0 \leq Y \, \perp \, \lambda \geq 0

\end{array}
\right)
\]
\newpage
\section{Algorithm}
We assume circuit topology is known.
\subsection{Unknown vectors construction}
\paragraph{Get x}:

\begin{algorithm}
\caption{x construction}
\begin{algorithmic}
\REQUIRE unknown structure
\COMMENT{About current\\}
\FORALL {inductor}
\STATE u=new unknown(I,j,k,inductor)
\STATE x.append(u)
\ENDFOR
\COMMENT{About tension\\}
\FORALL {capacitor}
\IF {$U_{jk}$ is not an unknown}
\STATE u=new unknown(U,j,k,capacitor)
\STATE x.append(u)
\ENDIF
\ENDFOR
\end{algorithmic}
\end{algorithm}
\paragraph{Get $Z_{ns}$}:

\begin{algorithm}
\caption{$Z_{ns}$ construction}
\begin{algorithmic}
\REQUIRE unknown structure
\FORALL {non smooth component}
\STATE getBetaUnknown from the model
\STATE u=new unknown(type,j,k,l,m,NS component)
\STATE $Z_{ns}$.append(u)
\ENDFOR
\end{algorithmic}
\end{algorithm}

\paragraph{Get $Z_{s}$}:

\begin{algorithm}
\caption{$Z_{s}$ construction}
\begin{algorithmic}
\REQUIRE unknown structure
\COMMENT{Add $V_{j}$\\}
\FORALL {node j}
\STATE u=new unknown(V,j)
\STATE $Z_{s}$.append(u)
\ENDFOR
\COMMENT{non smooth current\\}
\FORALL {non smooth component}
\FORALL{component connexion j}
\STATE u=new unknown(I,j,NS component)
\STATE $Z_{s}$.append(u)
\ENDFOR
\ENDFOR
\COMMENT{Voltage Defined}
\FORALL {Voltage Defined component}
\STATE u=new unknown(I,j,k,component)
\STATE $Z_{s}$.append(u)
\ENDFOR

\end{algorithmic}
\end{algorithm}
\newpage
\section{Parser API}
The library cirParser.so is a C library contains the .cir parser. it exports the parser API.\\

\subsection{cirParser API}

\begin{enumerate}
  \item bool readFile(char * file) : read a .cir file\\
  \item int initComponentList(char *typeName); :initialize internal data structure to get component of type \textit{typeName}.\\
  \item int nextComponent(void * data): fill the data structure \textit{data}.\\
\end{enumerate}

Possible values for \textit{typeName} : Resitor,Capacitor,Inductor,Vsource,Isource,Diode,...

\subsection{Ddata structures}
Each \textit{typeName} has a data structure contains all useful informations about the component.\\
Call nextComponent to fill the data structure.\\

\subsection{Algorithm}
A adapted stamp method is used to fill matrices.


 \end{document}
