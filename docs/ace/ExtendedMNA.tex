\section{The linear non smooth components}
\subsection{Topology. Choice of smooth and nonsmooth components.}
The stamp method is used for the linear components. About the non smooth components, a piecewise
linear modeling describes the component's behavior. This geometry is written with linear
complementarity conditions. 
\subsection{Unknowns}
Before go head, the unknowns vector X is subdivided:
\begin{enumerate}
\item[--] x contains only the dynamic unknowns(currents in inductor and tensions from capacitor branches)
\item[--] $Z_{s}$ contains only the non dynamic unknowns(Voltage nodes,... ).
\item[--] $Z_{ns}$ contains the useful currents and tensions for the non smooth components.
\end{enumerate}
\subsection{Linear local formulation}
For each non smooth component, we assume that the behavior can be written as the following the mixed linear  complementarity condition, that is:
\begin{equation}\left(\begin{array}{c}
\beta = Z_{nsi} = A_{i}x+B_{i}\lambda_{i}+C_{i}Z_{s} + a_{i}\\
y_{i}=D_{i}x+E_{i}Z_{s}+F_{i}\lambda_{i}+G_{i}Z_{nsi}+e_{i}\\
0 \leq y_{i} \, \perp \, \lambda_{i} \geq 0
\end{array}\right)
\end{equation}
where $Z_{nsi}$ is a voltage and currents vector. $a_{i},e_{i},\lambda_{i}$ and $y_{i}$ are some constants vectors which characterizes the components. At the end
of this document, we will see how ideal diode and piecewise linear models of transistor fits into this formulation.


\subsection{Example of a linear global formulation}
It consists in writing all the complementary conditions in the equations of the MNA. An example with 5 non smooth components is given by:

\[\left(\begin{array}{c}
  Z_{ns}\\
  \hline
  Z_{ns1} \\
  Z_{ns2} \\
  Z_{ns3} \\
  Z_{ns4} \\
  Z_{ns5} \\
\end{array}\right) =
\left(\begin{array}{c}
  C_{1x}\\
  \hline
  A1\\
  A2\\
  A3\\
  A4\\
  A5\\
\end{array}\right)x +
\left(\begin{array}{c}
  C_{1s}\\
  \hline
  C1\\
  C2\\
  C3\\
  C4\\
  C5\\
\end{array}\right)Z_{s} +
\left(\begin{array}{ccccc}
  C1_{\lambda}\\
  \hline
  B1&0&0&0&0\\
  0&B2&0&0&0\\
  0&0&B3&0&0\\
  0&0&0&B4&0\\
  0&0&0&0&B5\\
\end{array}\right)\left(\begin{array}{c}
  \lambda\\
  \hline
  \lambda _{1} \\
  \lambda _{2}\\
  \lambda _{3}\\
  \lambda _{4}\\
  \lambda _{5}\\
\end{array}\right)+
\left(\begin{array}{c}
  cst\\
  \hline
   cst_{1} \\
   cst_{2}\\
   cst_{3}\\
   cst_{4}\\
   cst_{5}\\
\end{array}\right)
\]

\[\left(\begin{array}{c}
  Y\\
  \hline
   Y_{1} \\
   Y_{2} \\
   Y_{3} \\
   Y_{4} \\
   Y_{5} \\
\end{array}\right) =
\left(\begin{array}{c}
  D_{1x}\\
  \hline
  D1\\
  D2\\
  D3\\
  D4\\
  D5\\
\end{array}\right)x +
\left(\begin{array}{c}
  D_{1s}\\
  \hline
  E1\\
  E2\\
  E3\\
  E4\\
  E5\\
\end{array}\right)Z_{s} +
\left(\begin{array}{ccccc}
  D1_{\lambda}\\
  \hline
  F1&0&0&0&0\\
  0&F2&0&0&0\\
  0&0&F3&0&0\\
  0&0&0&F4&0\\
  0&0&0&0&F5\\
\end{array}\right)
\left(\begin{array}{c}
  \lambda\\
  \hline
   \lambda _{1} \\
   \lambda _{2}\\
   \lambda _{3}\\
   \lambda _{4}\\
   \lambda _{5}\\
\end{array}\right)+\]
\[
\left(\begin{array}{ccccc}
  D1_{ns}\\
  \hline
  G1&0&0&0&0\\
  0&G2&0&0&0\\
  0&0&G3&0&0\\
  0&0&0&G4&0\\
  0&0&0&0&G5\\
\end{array}\right)
\left(\begin{array}{c}
  Zns\\
  \hline
   Z_{ns1} \\
   Z_{ns2}\\
   Z_{ns3}\\
   Z_{ns4}\\
   Z_{ns5}\
\end{array}\right)+
\left(\begin{array}{c}
  cst\\
  \hline
   cst_{1} \\
   cst_{2}\\
   cst_{3}\\
   cst_{4}\\
   cst_{5}\\
\end{array}\right)
\]
Finally, the global formulation form is : 

\[\left(\begin{array}{c}
Z_{ns}= C_{1x}x+C_{1zs}Z_{s}+C_{1\lambda}\lambda +C_{1s}\\
Y=D_{1x}x +D_{1zs}Z_{s}+D_{1ns}Z_{ns}+D_{1\lambda}\lambda+D_{1s}\\
0 \leq Y \, \perp \, \lambda \geq 0
\end{array}\right)\]
\newpage
\section{Extended MNA for nonsmooth linear components}
This part describes the automatic formulation into a Mixed Linear Complementarity System (MLCS) of a circuit. It consists in adapting the MNA to manage non-smooth model.

\subsection{MLCS and MLCP definitions}
\begin{definition}\index{Complementarity problem ! mixed linear} \index{MLCP}
  Given the matrices  ${A} \in \RR^{n \times n}$, ${B} \in \RR^{m \times m}$, ${C} \in \RR^{n \times m}$, ${D} \in \RR^{m \times n}$, and the vectors  $ {a} \in \RR^n, {b} \in \RR^m$, the MLCP denoted by $\mathrm{MLCP}(A,B,C,D,a,b)$ consists in finding two vectors $ {u} \in \RR^n$ and  $ {v} \in \RR^m$ such that
\begin{equation}\label{eq:mlcp1} 
  \begin{cases}
    A u + C v + a =0 \\  \\
    {0} \le {v} \perp     Du +B v +b   \ge {0}
  \end{cases}.
\end{equation}
The  MLCP can be defined equivalently in the following form denoted by $\mathrm{MLCP}(M,q,\mathcal E,\mathcal I)$
\begin{equation}
  \label{eq:mlcp2}
  \begin{cases}
    w = M z +q \\
    w_i=0,\forall  i \in \mathcal E \\
    {0} \le z_i \perp w_i \ge {0}, \forall  i \in \mathcal I 
 \end{cases}
\end{equation}
where  $\mathcal E$ and $\mathcal I$ are finite sets of indices such that $\mathrm{card}(\mathcal E \cup \mathcal I  ) = n$ and $\mathcal E \cap \mathcal I  = \emptyset$.\end{definition}
The MLCP is a mixture between an LCP and a system of linear equations. The former definition~(\ref{eq:mlcp1}) can be casted into the second one~(\ref{eq:mlcp2}) by introducing
\begin{equation}
  \label{eq:mlcp-m}
  M = \left[
  \begin{array}{cc}
   A & C \\
   D & B
  \end{array}\right],\quad   q = \left[
  \begin{array}{c}
    a \\
    b
  \end{array}\right],\quad   z = \left[
  \begin{array}{c}
    u \\
    v
  \end{array}\right],\quad   w = \left[
  \begin{array}{c}
    0\\
    w_i, \forall  i \in \mathcal I 
  \end{array}\right]
\end{equation}




Equivalently, the MLCS can be defined by an implicit system

\begin{definition}\index{Complementarity system ! mixed linear} \index{MLCS}
  Given the matrices  ${M} \in \RR^{n \times n}$, ${A} \in \RR^{n \times n}$, ${B} \in \RR^{m \times
  m}$, ${C} \in \RR^{n \times m}$, ${D} \in \RR^{m \times n}$, and the vectors  $ {a} \in \RR^n, {b}
  \in \RR^m$, the implicit MLCS denoted by $\mathrm{IMLCS}(M,A,B,C,D,a,b)$ consists in finding two vectors $ {x} \in \RR^n$ and  $ {v} \in \RR^m$ such that
\begin{equation}\label{eq:mlcs1} 
  \begin{cases}
   M x' = A x + C v + a  \\  \\
   {0} \le {v} \perp     Dx +B v +b   \ge {0}
  \end{cases}.
\end{equation}
\end{definition}

or an explicit system
\begin{definition}\index{Complementarity system ! mixed linear} \index{MLCS}
  Given the matrices   ${A_{x}} \in \RR^{n \times n}$, ${A_{z}} \in \RR^{n \times
  p}$, ${A_{v}} \in \RR^{n \times m}$, ${B_{x}} \in \RR^{p \times n}$, ${B_{z}} \in \RR^{p \times
  p}$, ${B_{v}} \in \RR^{p \times m}$, ${C_{x}} \in \RR^{m \times n}$, ${C_{z}} \in \RR^{m \times p}$,${C_{v}} \in \RR^{m \times m}$, and
  the vectors  $ {a} \in \RR^n$,$ {b}  \in \RR^p$,$ {c}  \in \RR^m$, the explicit MLCS denoted by
  $\mathrm{EMLCS}(A_{x},A_{z},A_{v},B_{x},B_{z},B_{v},C_{x},C_{z},C_{v},a,b,c)$ consists in finding three vectors $ {x}
  \in \RR^n$, $ {z} \in \RR^p$ and  $ {v} \in \RR^m$ such that
\begin{equation}\label{eq:mlcs2} 
  \begin{cases}
   x' = A_{x} x +A_{z} z +A_{v} v + a  \\
   0 = B_{x} x +B_{z} z + B_{v} v +b \\ \\
   {0} \le {v} \perp     C_{x} x+ C_{z}z +C_{v} v +c   \ge {0}
  \end{cases}.
\end{equation}
\end{definition}
\subsection{Hypothesis}
The MNA assumes smooth branches are explicit functions of current or voltage. It means each smooth
branch is Voltage Defined (V.D.) or Current Defined (C.D.)\\

With the same assumption, all branches can be divided into following classes:\\
\begin{enumerate}
\item The  \underline{C.D. branches} 
\item The  \underline{V.D. branches}
 \item The  \underline{non smooth branches}
 \item The \underline{dynamical capacitor branches}
 \item The \underline{dynamical inductors branches}
\end{enumerate}




\subsection{Nonsmooth equations and MLCS}
With the linear complementary condition, the system becomes an explicit MLCS of the form :
\[x' = A_{1x}x +A_{1zs}Z_{s} + A_{1ns}Z_{ns}+A_{1s}\]
\[0  = B_{1x}x+B_{1zs}Z_{s} + B_{1ns}Z_{ns}+B_{1s}\]
\[Z_{ns}= C_{1x}x+C_{1zs}Z_{s}+C_{1\lambda}\lambda +C_{1s}\]
\[Y=D_{1x}x +D_{1zs}Z_{s}+D_{1ns}Z_{ns}+D_{1\lambda}\lambda+D_{1s}\]
\[0 \leq Y \, \perp \, \lambda \geq 0\]
 

\section{To get an explicit MLCS}

This section illustrates the rules how get $x' = A_{1x}x +A_{1zs}Z_{s} + A_{1ns}Z_{ns}+A_{1s}$? To get this system, a good set of unknowns must be done. Only the necessary unknowns are added. The
following examples show how this choice could be done.
\subsection{Example 1}
\begin{figure}[h]
\centerline{
 \scalebox{0.5}{
    \input{../ace/cir1.pstex_t}
 }
}
\end{figure}
\paragraph{First matrices formulation}
X=$^{t}(V_{1},V_{2},V_{3})$\\
\[\left(\begin{array}{c}
  \\
  KCL(1)\\  KCL(2)\\  KCL(3)
  \end{array}\right)
\left(\begin{array}{ccc}
  V_{1}&V_{2}&V_{3}\\
  \hline
  \frac{1}{R}-\frac{1}{R}&  \frac{-1}{R}&0\\
  \frac{1}{R}&  \frac{-1}{R}&0\\
  0&0&\frac{1}{R}
\end{array}\right)X+
\underline{
\left(\begin{array}{ccc}
   V_{1}'&V_{2}'&V_{3}'\\
  \hline
0&0&0\\
  0&C&-C\\
  0&-C&C
\end{array}\right)}X'=
\left(\begin{array}{c}
  \\
  I\\
  0\\
  0
  \end{array}\right)
\]
But, we can't extract x from X to get x'=...\\
\newline
\paragraph{Add current and tension from the capacitor}
x=$(U_{32})$
$Z_{s}=^{t}(V_{1},V_{2},V_{3},I_{32})$\\
\[x'=CI_{32}\]
\[\left(\begin{array}{c}
  \\
  KCL(1)\\
  KCL(2)\\
  KCL(3)\\
  U_{32}
  \end{array}\right)
\left(\begin{array}{cccc}
  V_{1}&V_{2}&V_{3}&I_{32}\\
  \hline
  \frac{1}{R}-\frac{1}{R}&  \frac{-1}{R}&0&0\\
  \frac{1}{R}&  \frac{-1}{R}&0&1\\
  0&0&\frac{1}{R}&-1\\
  0&-1&1&0
\end{array}\right)Z_{s}+
\left(\begin{array}{c}
  U_{32}\\
  \hline
  0\\
  0\\
  0\\
  1
  \end{array}\right)x
=
\left(\begin{array}{c}
  \\
  I\\
  0\\
  0\\
  0
  \end{array}\right)
\]
We obtain the matrices system\\
\[x'=A1_{zs}Z_{s}\]
\[Ax+CZ_{s}=s\]
$N_{I}=N_{E}=5$\\
The following examples, show we don't need to add the capacitor current. Sometime, current in
capacitor can be replace with a KCL law.\\

\paragraph{Add only tension from the capacitor}
x=$(U_{32})$,$Z_{s}=^{t}(V_{1},V_{2},V_{3})$\\
First get x' with \underline{KCL(3)}:
\[x'=\frac{V_{3}}{RC}\]
Second, the current in capacitor is $\frac{V_{3}}{R}$, and use it to write other KCL law:
\[\left(\begin{array}{c}
  \\
  KCL(1)\\
  KCL(2)\\
  U_{32}
  \end{array}\right)
\left(\begin{array}{ccc}
V_{1}&V_{2}&V_{3}\\
  \hline
  \frac{1}{R}+\frac{1}{R}&  \frac{-1}{R}&0\\
  \frac{-1}{R}&  \frac{1}{R}&\frac{1}{R}\\
  0&-1&1
\end{array}\right)Z_{s}+
\left(\begin{array}{c}
U_{32}\\
  \hline
  0\\
  0\\
  0\\
  1
  \end{array}\right)x
=
\left(\begin{array}{c}
  \\
  I\\
  0\\
  0\\
  0
  \end{array}\right)
\]
$N_{I}=N_{E}=4$
\newpage

 
\subsubsection{Example 2}
\begin{figure}[h]
\centerline{
 \scalebox{0.9}{
    \input{cir2.pstex_t}
 }
}
\end{figure}
\paragraph{Add currents and tensions from the capacitor}
$x=^{t}(I_{42},U_{43},U_{31},U_{50})$,
$Z_{s}=^{t}(V_{1},V_{2},V_{3},V_{4},V_{5},I_{43},I_{31},I_{50})$
We obtain following equation:
\[\left(\begin{array}{cccc}
  I_{42}'&U_{43}'&U_{31}'&U_{50}'\\
  \hline
L&0&0&0\\
0&C&0&0\\
0&0&C&0\\
0&0&0&C
\end{array}\right)x'=
0x+
\left(\begin{array}{cccccccc}
  V_{1}&V_{2}&V_{3}&V_{4}&V_{5}&I_{43}&I_{31}&I_{50}\\
  \hline
  0&1&0&-1&0&0&0&0\\
  0&0&0&0&0&-1&0&0\\
  0&0&0&0&0&0&1&0\\
  0&0&0&0&0&0&0&1\\
\end{array}\right)Z_{s}
\]
\[\left(\begin{array}{c}
\\KCL(1)\\KCL(2)\\KCL(3)\\KCL(4)\\KCL(5)\\U_{43}\\U_{31}\\U_{50}
\end{array}\right)
\left(\begin{array}{cccccccc}
  V_{1}&V_{2}&V_{3}&V_{4}&V_{5}&I_{43}&I_{31}&I_{50}\\
  \hline
  -\frac{1}{R}&\frac{1}{R}&0&0&0&0&1&0\\
  \frac{1}{R}&-\frac{1}{R}&0&0&0&0&0&0\\
  0&0&0&0&0&-1&1&0\\
  0&0&0&-\frac{1}{R}&\frac{1}{R}&1&0&0\\
  0&0&0&\frac{1}{R}&-\frac{1}{R}&0&0&1\\
  0&0&-1&1&0&0&0&0\\
  -1&0&1&0&0&0&0&0\\
  0&0&0&0&1&0&0&0\\
\end{array}\right)Z_{s}+
\left(\begin{array}{cccc}
  I_{42}&U_{43}&U_{31}&U_{50}\\
  \hline
  0&0&0&0\\
  1&0&0&0\\
  0&0&0&0\\
  0&0&0&0\\
  1&0&0&0\\
  0&1&0&0\\
  0&0&1&0\\
  0&0&0&1\\
\end{array}\right)x=
\left(\begin{array}{c}
  \\ I\\  0\\  0\\  0\\  0\\  0\\  0\\  0
  \end{array}\right)
\]

\[x'=A1_{zs}Z_{s}\]
\[Cx+BZ_{s}=s\]
$N_{I}=N_{E}=12$
\paragraph{Add only tensions from the capacitor}
$x=^{t}(I_{42},U_{43},U_{31},U_{50})$,
$Z_{s}=^{t}(V_{1},V_{2},V_{3},V_{4},V_{5})$\\
We obtain following equation:
\[
\left(\begin{array}{c}
  \\  KCL(4)\\  KCL(3)\\  KCL(5)
\end{array}\right)
\underline{
\left(\begin{array}{cccc}
  I_{42}'&U_{43}'&U_{31}'&U_{50}'\\
  \hline
L&0&0&0\\
0&C&0&0\\
0&C&C&0\\
0&0&0&C
\end{array}\right)}x'=
\left(\begin{array}{cccc}
  I_{42}&U_{43}&U_{31}&U_{50}\\
  \hline
0&0&0&0\\
1&0&0&0\\
0&0&0&0\\
0&0&0&0\\ 
\end{array}\right)x+
\left(\begin{array}{ccccc}
V_{1}&V_{2}&V_{3}&V_{4}&V_{5}\\
  \hline
  0&1&0&-1&0\\
  0&0&0&\frac{1}{R}&-\frac{1}{R}\\
  0&0&0&0&0\\
  0&0&0&-\frac{1}{R}&\frac{1}{R}
\end{array}\right)Z_{s}
\]
So, we get $x'=A1_{x}x+A1_{zs}Z_{s}$. Therefore, all currents in the capacitor branch are known.\\
$I_{43}= I_{42}+\frac{V_{4}}{R}-\frac{V_{5}}{R}$,
$I_{31}= I_{42}+\frac{V_{4}}{R}-\frac{V_{5}}{R}$,
$I_{50}= \frac{V_{4}}{R}-\frac{V_{5}}{R}$\\
Use these equations to fill following matrices:
\[\left(\begin{array}{c}
KCL(1)\\KCL(2)\\U_{43}\\U_{31}\\U_{50}
\end{array}\right)
\left(\begin{array}{ccccc}
V_{1}&V_{2}&V_{3}&V_{4}&V_{5}\\
  \hline
  -\frac{1}{R}&\frac{1}{R}&0&\underline{\frac{1}{R}}&\underline{-\frac{1}{R}}\\
  \frac{1}{R}&-\frac{1}{R}&0&0&0\\
  0&0&-1&1&0\\
  -1&0&1&0&0\\
  0&0&0&0&1\\
\end{array}\right)Z_{s}+
\left(\begin{array}{cccc}
  I_{42}&U_{43}&U_{31}&U_{50}\\
  \hline
  \underline{1}&0&0&0\\
  1&0&0&0\\0&1&0&0\\0&0&1&0\\0&0&0&1\\
\end{array}\right)x=
\left(\begin{array}{c}
  I\\0\\0\\0\\0
  \end{array}\right)
\]
$N_{I}=N_{E}=9$


\subsection{Example 3}
\begin{figure}[h]
\centerline{
 \scalebox{0.8}{
    \input{cir3.pstex_t}
 }
}
\end{figure}

x=$(U_{12})$\\
$Z_{s}=^{t}(V_{1},V_{2},I_{10})$\\
\[(KCL(1))=>(C1+C2+C3)x'=I_{10}\]
So, all capacitor's currents are known:\\
$I_{c1}=\frac{C1}{C1+C2+C3}I_{10}$\\
$I_{c2}=\frac{C2}{C1+C2+C3}I_{10}$\\
$I_{c3}=\frac{C3}{C1+C2+C3}I_{10}$\\
Use these equations to fill following matrices:\\
\[\left(\begin{array}{c}
  \\
KCL(2)\\U_{21}\\VD
\end{array}\right)
\left(\begin{array}{c}
U_{21}\\
\hline
0\\
1\\
0
\end{array}\right)x+
\left(\begin{array}{ccc}
V_{1}&V_{2}&I_{10}\\
\hline
0&\frac{1}{R}&\frac{C1}{C1+C2+C3}+\frac{C2}{C1+C2+C3}+\frac{C3}{C1+C2+C3}\\
1&-1&0\\
1&0&0
\end{array}\right)Z_{s}=
\left(\begin{array}{c}
\\0\\0\\E
\end{array}\right)
\]
$N_{I}=N_{E}=4$
\newpage

\subsection{Example 4}
This example shows that is not always possible to use the KCL law to get x'=...\\
\begin{figure}[h]
\centerline{
 \scalebox{0.9}{
    \input{cir4.pstex_t}
 }
}\end{figure}\\
$x=^{t}(U_{21},U_{23},U_{34})$
$Z_{s}=^{t}(V_{1},V_{2},V_{3},V_{4})$
\paragraph{a problem:}
Start to fill the x' matrices. We use KCL(2) for $U_{21}$, and KCL(3) for $U_{34}$.
\[\left(\begin{array}{c}
  \\
KCL(2)\\KCL(3)\\??
\end{array}\right)
\left(\begin{array}{ccc}
  U_{21}'&U_{23}'&U_{34}'\\
  \hline
  C&-C&0\\
  0&C&C\\
  ?&?&?
\end{array}\right)x'=0x+
\left(\begin{array}{cccc}
  V_{1}&V_{2}&V_{3}&V_{4}\\
  \hline
  0&0&0&0\\
  0&0&0&0\\
  ?&?&?&?\\
\end{array}\right)Z_{s}
 \]
 We can't use a other KCL law to get $U_{23}$. A solution could be to add an unknown, \underline{$I_{23}$} in
 $Z_{s}$. The system becomes:\\
 \[\left(\begin{array}{c}
  \\
KCL(2)\\KCL(3)\\I_{23}
\end{array}\right)
\left(\begin{array}{ccc}
  U_{21}'&U_{23}'&U_{34}'\\
  \hline
  C&-C&0\\
  0&C&C\\
  0&C&0
\end{array}\right)x'=0x+
\left(\begin{array}{ccccc}
  V_{1}&V_{2}&V_{3}&V_{4}&\underline{I_{23}}\\
  \hline
  0&0&0&0&0\\
  0&0&0&0&0\\
  0&0&0&0&1\\
\end{array}\right)Z_{s}
 \]
 So, we obtain $x'=A1_{zs}Z_{s}$. All capacitor's currents are known:
 $I_{21}=\underline{I_{23}}=I_{34}$\\
\[\left(\begin{array}{c}
  \\
KCL(4)\\KCL(1)\\U_{21}\\U_{23}\\U_{34}
\end{array}\right)
\left(\begin{array}{ccc}
  U_{21}&U_{23}&U_{34}\\
  \hline
  0&0&0\\
  0&0&0\\
  1&0&0\\
  0&1&0\\
  0&0&1
\end{array}\right)x+
\left(\begin{array}{cccc}
  V_{1}&V_{2}&V_{3}&\underline{I_{23}}\\
  \hline
  0&0&0&1\\
  \frac{1}{R}&0&0&1\\
  -1&1&0&0\\
  0&-1&1&0\\
  0&0&-1&1\\
\end{array}\right)Z_{s}=
\left(\begin{array}{c}
\\I\\0\\0\\0\\0
\end{array}\right)
\]
 $N_{I}=N_{E}=8$\\
\paragraph{a good choice:}
\[\left(\begin{array}{c}
  \\
KCL(1)\\KCL(2)\\KCL(3)
\end{array}\right)
\left(\begin{array}{ccc}
  U_{21}'&U_{23}'&U_{34}'\\
  \hline
  C&0&0\\
  C&C&0\\
  0&-C&C
\end{array}\right)x'=0x+
\left(\begin{array}{cccc}
  V_{1}&V_{2}&V_{3}&V_{4}\\
  \hline
  \frac{1}{R}&-\frac{1}{R}&0&0\\
  0&0&0&0\\
  0&0&0&0\\
\end{array}\right)Z_{s}
 \]
 So, we obtain $x'=A1_{zs}Z_{s}$. Therefore all capacitor's currents are known: $I_{12}=I_{23}=I_{34}=\frac{V1}{R1}$\\
 \[\left(\begin{array}{c}
  \\
KCL(4)\\U_{21}\\U_{23}\\U_{34}
\end{array}\right)
 \left(\begin{array}{ccc}
  U_{21}&U_{23}&U_{24}\\
  \hline
  0&0&0\\
  1&0&0\\
  0&1&0\\
  0&0&1\\
\end{array}\right)x+
 \left(\begin{array}{cccc}
  V_{1}&V_{2}&V_{3}&V_{4}\\
  \hline
  \frac{1}{R}&0&0&0\\
  -1&1&0&0\\
  0&-1&1&0\\
  0&0&-1&1\\
\end{array}\right)Z_{s}=
 \left(\begin{array}{c}
  \\I\\0\\0\\0
\end{array}\right)
 \]
$N_{I}=N_{E}=7$\\






 
\subsection{One  example with  a capacitor  loop}

This example shows we have to manage the capacitor cycle.\\
\begin{figure}[h]
\centerline{
 \scalebox{0.6}{
    \input{../ace/cir5.pstex_t}
 }
}\end{figure}\\


\paragraph{Standard MNA Algorithm}
 \texttt{texte ::} $x=^{t}(U_{12},U_{23},U_{34},U_{41})$$z=(V_{1},V_{2},V_{3},V_{4},V_{5},I_{50})$\\
Start to write Mx'=...\
\[\left(\begin{array}{c}
  \\
KCL(1)\\KCL(2)\\KCL(3)\\KCL(4)
\end{array}\right)
\left(\begin{array}{cccc}
  U_{12}'&U_{23}'&U_{34}'&U_{41}'\\
  \hline
  C&0&0&-C\\
  -C&C&0&0\\
  0&-C&C&0\\
  0&0&-C&C\\  
\end{array}\right)x'= RHS
\]
where the Right-hand-side  $RHS$ is not here detailed.
This matrix is not regular because of the cycle \{1-2,2-3,3-4,4-1\}. 



\paragraph{The proposed Algorithm~\ref{Algo:ACEF1}}

A solution could be to use the Minimum
Spanning Tree \{1-2,2-3,3-4\} to write the KCL law. About the last tension, $U_{41}$, there are tow
ways:
\begin{enumerate}
\item add a unknown $I_{41} in z$ and write CU'=I
\item Find the linear relation $U_{41}'= \sum_{jk}^{}a_{jk}U_{kj}'$, and replace $U_{ki}'$.
\end{enumerate}
The matrices become:\\
Start to write Mx'=...\
\[\left(\begin{array}{c}
  \\
KCL(1)\\KCL(2)\\KCL(3)\\I_{41}
\end{array}\right)
\left(\begin{array}{cccc}
  U_{12}'&U_{23}'&U_{34}'&U_{41}'\\
  \hline
  C&0&0&-C\\
  -C&C&0&0\\
  0&-C&C&0\\
  0&0&0&C\\  
\end{array}\right)x'=0x+
\left(\begin{array}{ccccccc}
  V_{1}&V_{2}&V_{3}&V_{4}&V_{5}&I_{50}&I_{41}\\
  \hline
  -\frac{1}{R}&0&0&0&0&0&0\\
  0&0&0&0&0&0&0\\
  0&0&0&\frac{1}{R}&-\frac{1}{R}&0&0\\
  0&0&0&0&0&0&1\\
\end{array}\right)z
\]
So, we obtain $x'=A1_{z}z$. Therefore all capacitor's currents are
known:$I_{12}=I_{23}=I_{41}+\frac{V1}{R},I_{43}=I_{41}+\frac{V1}{R}+\frac{V5}{R}-\frac{V3}{R}$\\
The last step consists in writing the missing equations:
 \[\left(\begin{array}{c}
  \\
KCL(4)\\KCL(5)\\U_{12}\\U_{23}\\U_{34}\\U_{41}\\VD_{50}
\end{array}\right)
 \left(\begin{array}{cccc}
  U_{12}&U_{23}&U_{34}&U_{41}\\
  \hline
  0&0&0&0\\
  0&0&0&0\\
  1&0&0&0\\
  0&1&0&0\\
  0&0&1&0\\
  0&0&0&1\\
  0&0&0&0\\

\end{array}\right)x+
 \left(\begin{array}{ccccccc}
  V_{1}&V_{2}&V_{3}&V_{4}&V_{5}&I_{50}&I_{41}\\
  \hline
  \frac{1}{R}&0&-\frac{1}{R}&0&\frac{1}{R}&0&\underline{-1+1}\\
  0&0&\frac{1}{R}&0&-\frac{1}{R}&-1&0\\
  -1&1&0&0&0&0&0\\
  0&-1&1&0&0&0&0\\
  0&0&-1&1&0&0&0\\
  0&0&0&-1&1&0&0\\
  0&0&0&0&1&0&0\\
\end{array}\right)z=
 \left(\begin{array}{c}
  \\0\\0\\0\\0\\0\\0\\E
\end{array}\right)
 \]
$N_{I}=N_{E}=11$\\


\subsection{Automatic circuit equation formulation Algorithm}

\begin{algorithm}
\caption{fill the matrices : $x'=A_{1x}x+A_{1s}Z_{s}+A_{1ns}Z_{ns}$ }
\begin{algorithmic}
\REQUIRE Init\_I\_in\_x : initialize internal data structure to get all I from x.
\REQUIRE Next\_I\_in\_x : return the next available I from x. If there are not available x, return 0.\\
\REQUIRE Minimum Spanning Tree of the capacitor' tension graph
\REQUIRE Init\_MST : initialize internal data structure to get all u from x.
\REQUIRE Next\_u\_in\_MST : return a available U's neighbour from MST if possible. If there are not
available u in MST, return 0.\\
\REQUIRE Next\_U\_in\_x : return the next available U from x. If there are not available x, return 0.\\

0\\
\COMMENT{About current}\\
\COMMENT{get first current}
\STATE Init\_I\_in\_x()
\STATE $I_{kj}$ = Next\_I\_in\_x()
\WHILE{$I_{kj}$}
\STATE use $LI'=V_{j}-V_{k}$ to fill $I_{kj}$'s line.
\ENDWHILE\\
\COMMENT{About tension}\\
\COMMENT{get a first capacitor tension}
\STATE Init\_MST()
\STATE $U_{kj}$ = Next\_u\_in\_MST()
\WHILE{$U_{kj}$}
\STATE l=j or k with KCL(k) available.
\STATE Use CU'=I and KCL(k) to fill $U_{ki}$'s line.\\
\STATE enable KCL(k)

\STATE $U_{kj}$= Next\_u\_in\_MST ()
\ENDWHILE
\STATE $U_{kj}$ = Next\_U\_in\_x()
\WHILE {$U_{kj}$}
\STATE Add an unknown $I_{kj}$, and use it to fill the matrices. Write I=CU'.
\STATE $U_{kj}$ = Next\_U\_in\_x()
\ENDWHILE
\COMMENT{reverse A : $Ax'=Bx+CZ_{s}+DZ_{ns}$\\}
\STATE $A^{-1}$=Inv(A)
\STATE $x'=A_{1x}x+A_{1s}Z_{s}+A_{1ns}Z_{ns}$
\end{algorithmic}
\end{algorithm}



\subsection{conclusion}
The vector x contains inductor's currents and capacitor's tensions. Derivate inductor's current is equal to a
nodal voltage difference.\\
About the capacitor's tensions, we use the Minimum Spanning Tree of the capacitor' tension to avoid cycle.\\


%%% Local Variables: 
%%% mode: latex
%%% TeX-master: "ace"
%%% End: 


\newpage

\section{Summary of  the Matrix formulation}

The previous section describes how get the following system:
\[\left(\begin{array}{c}
x'=A_{1x}x +A_{1zs}Z_{s} + A_{1ns}Z_{ns}+A_{1s}\\
0=B_{1x}x+B_{1zs}Z_{s} + B_{1ns}Z_{ns}+B_{1s}\\
Z_{ns}= C_{1x}x+C_{1zs}Z_{s}+C_{1\lambda}\lambda +C_{1s}\\
Y=D_{1x}x +D_{1zs}Z_{s}+D_{1ns}Z_{ns}+D_{1\lambda}\lambda+D_{1s}\\
0 \leq Y \, \perp \, \lambda \geq 0
\end{array}\right)\]
Substitute $Z_{ns}$:
\[\left(\begin{array}{c}
R=A_{1ns}C_{1\lambda}\\
x'=(A_{1x}+A_{1ns}C_{1x})x +(A_{1zs}+A_{1ns}C_{1zs})Z_{s} +R\lambda+A_{1s} + A_{1ns}C_{1s}\\
x'=A_{2x}x +A_{2zs}Z_{s} +R \lambda +A_{2s}\\
0=(B_{1x}+B_{1ns}C_{1x})x+(B_{1zs}+B_{1ns}C_{1zs})Z_{s} + B_{1ns}C_{1\lambda}\lambda +B_{1s} + B_{1ns}C_{1s} \\
0=B_{2x}x+B_{2zs}Z_{s} + B_{2\lambda}\lambda + B_{2s}\\
Y=(D_{1x}+D_{1ns}C_{1x})x+(D_{1zs}+D_{1ns}C_{1zs})Z_{s}+(D_{1\lambda}+D_{1ns}C_{1\lambda})\lambda +D_{1s}+D_{1ns}C_{1s}\\
Y=D_{2x}x+D_{2zs}Z_{s}+D_{2\lambda}\lambda + D_{2s} \\
0 \leq Y \, \perp \, \lambda \geq 0\\
\end{array}\right)\]
$A_{2s}, B_{2s}$ and $D_{2s}$ are vectors.



%%% Local Variables: 
%%% mode: latex
%%% TeX-master: "ace"
%%% End: 
