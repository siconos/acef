
\frame
{
\frametitle{A solver using the linear programming.\footnote{April 1983. An Enumerative Method for the Solution
of LCP By J.J. Judice and G. Mitra}}
\begin{block}{ Given a MLCP}
 

\begin{equation}\label{eq:mlcp1}
 \begin{cases}
M \left(\begin{array}{c}
   u\\
   \lambda
   \end{array}\right) + q = \left(\begin{array}{c}
   0\\
   y
   \end{array}\right) \\
      0 \le \lambda \perp     y   \ge 0

      \end{cases}
\end{equation}


\end{block}

\begin{block}{An interesting linear programming problem}
Let I and J, tow subsets of \{1,..,m\} with  $I \cap J = \emptyset$

\begin{equation}\label{eq:mlcp1}
\begin{cases}
M \left(\begin{array}{c}
   u\\
   \lambda
   \end{array}\right) + q = \left(\begin{array}{c}
   0\\
   y
   \end{array}\right)\\
\lambda_i = 0 \textrm{ for } i \in I \qquad \lambda_i \ge 0 \textrm{ for } i \notin I\\
y_j = 0 \textrm{ for } j \in J \qquad y_j \ge 0 \qquad j \notin J\\
\min{\sum_{i\notin I\cup J} \lambda_i + \sum_{j\notin I\cup J} y_j}
\end{cases}
\end{equation}
Tow cases:\\
\begin{itemize}
 \item[--] The minimization doesn't success, it means the constraints are not compatible, $\lambda_i =0 \textrm{ for } i \in I \qquad y_j=0 \textrm{ for } j\in J$ must be eliminated.\\
\item[--]Success, may the optimal point is a solution.
\end{itemize}

\end{block}

}

\frame
{
\frametitle{A branch and prune algorithm}
\[\left(\begin{array}{cccc}
4&0&0.2&0\\
0&2&0&0\\
1&0&0.2&0.5\\
0&0&0.5&0.2
 \end{array}\right) \left(\begin{array}{c}
 u_1\\
 u_2\\
 \lambda_1\\
 \lambda_2
 \end{array}\right) + \left(\begin{array}{c}
 -6\\
 -4\\
 -3\\
 2
 \end{array}\right) =  \left(\begin{array}{c}
 0\\
 0\\
 y_1\\
 y_2
 \end{array}\right) 
 \]
\begin{figure}[h]
\centerline{
 \scalebox{0.5}{
    \input{LP_step0.pstex_t}
 }
}
\end{figure}

}
\frame
{
\frametitle{A branch and prune algorithm}
\[\left(\begin{array}{cccc}
4&0&0.2&0\\
0&2&0&0\\
1&0&0.2&0.5\\
0&0&0.5&0.2
 \end{array}\right) \left(\begin{array}{c}
 u_1\\
 u_2\\
 \lambda_1\\
 \lambda_2
 \end{array}\right) + \left(\begin{array}{c}
 -6\\
 -4\\
 -3\\
 2
 \end{array}\right) =  \left(\begin{array}{c}
 0\\
 0\\
 y_1\\
 y_2
 \end{array}\right) 
 \]
\begin{figure}[h]
\centerline{
 \scalebox{0.5}{
    \input{LP_step1.pstex_t}
 }
}
\end{figure}

}
\frame
{
\frametitle{A branch and prune algorithm}
\[\left(\begin{array}{cccc}
4&0&0.2&0\\
0&2&0&0\\
1&0&0.2&0.5\\
0&0&0.5&0.2
 \end{array}\right) \left(\begin{array}{c}
 u_1\\
 u_2\\
 \lambda_1\\
 \lambda_2
 \end{array}\right) + \left(\begin{array}{c}
 -6\\
 -4\\
 -3\\
 2
 \end{array}\right) =  \left(\begin{array}{c}
 0\\
 0\\
 y_1\\
 y_2
 \end{array}\right) 
 \]
\begin{figure}[h]
\centerline{
 \scalebox{0.5}{
    \input{LP_step2.pstex_t}
 }
}
\end{figure}

}
\frame
{
\frametitle{A branch and prune algorithm}
\[\left(\begin{array}{cccc}
4&0&0.2&0\\
0&2&0&0\\
1&0&0.2&0.5\\
0&0&0.5&0.2
 \end{array}\right) \left(\begin{array}{c}
 u_1\\
 u_2\\
 \lambda_1\\
 \lambda_2
 \end{array}\right) + \left(\begin{array}{c}
 -6\\
 -4\\
 -3\\
 2
 \end{array}\right) =  \left(\begin{array}{c}
 0\\
 0\\
 y_1\\
 y_2
 \end{array}\right) 
 \]
\begin{figure}[h]
\centerline{
 \scalebox{0.5}{
    \input{LP_step3.pstex_t}
 }
}
\end{figure}

}