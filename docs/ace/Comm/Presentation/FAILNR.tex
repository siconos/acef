
\frame
{

\frametitle{A switched circuit. \footnote{IEEE 2006. Closed-Loop Switched Circuits. P. Mafeezzonni, L. Codecasa, D. D'Amore}}
  \begin{figure}[!h]
   \centerline{
   \scalebox{0.9}{
    \input{CS.pstex_t}
    }
 } 
 \end{figure}



 
}

\frame
{

\frametitle{A switched circuit.}
  \begin{figure}[!h]
   \centerline{
   \scalebox{0.9}{
    \input{CSRES.pstex_t}
    }
 } 
 \end{figure}
\begin{block}{Model}
\begin{equation}
\begin{array}{cc}
R_s= \begin{cases}
R_{on} \qquad \text{if $V_3-V_2 \geq 0$}\\
R_{off} \qquad \text{if $V_3-V_2 < 0$}\\
\end{cases}&
R_d= \begin{cases}
R_{on} \qquad \text{if $V_1 < 0$}\\
R_{off} \qquad \text{if $V_1 \geq 0$}\\
\end{cases}
\end{array}
\end{equation}
\end{block}

 }

 \frame
{

\frametitle{t=0, $I_L=0$}
  \begin{figure}[!h]
   \centerline{
   \scalebox{0.9}{
    \input{CST0.pstex_t}
    }
 } 
 \end{figure}
\begin{block}{An equivalent linear circuit}
While the diode and the switch stay, in the same state, the Newton-Raphson algorithm converges in
one iteration. It is a linear circuit.
\end{block}


 }

 \frame
{

\frametitle{t=$t_1$, the switch change of state.}

  \begin{figure}[!h]
   \centerline{
   \scalebox{0.9}{
    \input{CSTstep0.pstex_t}
    }
 } 
 \end{figure}

 \begin{block}{Newton-Raphson iterations}
\begin{equation}
\begin{array}{|c|c|c|c|c|c|c|}
\hline
&$k=0$&$k=1$&$k=2$&$k=3$&$k=4$&...\\
\hline
S&ON&&&&&\\
\hline
D&OFF&&&&&\\
\hline
\end{array}
\end{equation}
\end{block}


 }

\frame
{

\frametitle{t=$t_1$, the switch change of state.}
  \begin{figure}[!h]
   \centerline{
   \scalebox{0.9}{
    \input{CSTstep1.pstex_t}
    }
 } 
 \end{figure}
  \begin{block}{Newton-Raphson iterations}
\begin{equation}
\begin{array}{|c|c|c|c|c|c|c|}
\hline
&$k=0$&$k=1$&$k=2$&$k=3$&$k=4$&...\\
\hline
S&ON&OFF&&&&\\
\hline
D&OFF&OFF&&&&\\
\hline
\end{array}
\end{equation}
\end{block}

 }
  \frame
{

\frametitle{t:$t_1 \to t_1+h$, first Newton-Raphson iteration.}
  \begin{figure}[!h]
   \centerline{
   \scalebox{0.9}{
    \input{CSTstep2.pstex_t}
    }
 } 
 \end{figure}
  \begin{block}{Newton-Raphson iterations}
\begin{equation}
\begin{array}{|c|c|c|c|c|c|c|}
\hline
&$k=0$&$k=1$&$k=2$&$k=3$&$k=4$&...\\
\hline
S&ON&OFF&ON&&&\\
\hline
D&OFF&OFF&ON&&&\\
\hline
\end{array}
\end{equation}
\end{block}


 }
  \frame
{

\frametitle{t:$t_1 \to t_1+h$, second Newton-Raphson iteration.}
  \begin{figure}[!h]
   \centerline{
   \scalebox{0.9}{
    \input{CSTstep1.pstex_t}
    }
 } 
 \end{figure}
   \begin{block}{Newton-Raphson iterations}
\begin{equation}
\begin{array}{|c|c|c|c|c|c|c|}
\hline
&$k=0$&$k=1$&$k=2$&$k=3$&$k=4$&...\\
\hline
S&ON&OFF&ON&OFF&ON&...\\
\hline
D&OFF&OFF&ON&OFF&ON&...\\
\hline
\end{array}
\end{equation}
\end{block}

 }


 \frame
{
\frametitle{Complementarity formulation.}

\begin{figure}
   \centerline{
   \scalebox{0.8}{
    \input{CSsmall.pstex_t}
    }
 } 
 \end{figure}

  
 \begin{eqnarray*}
 &{\color{Green}L\dot I_L -(V_1-V_2)=0}&V_3-e(t)=0\\
 &I_L-I_S-I_D=0&RI_L-V_2=0\\
 &{\color{Red}V_1+I_D(\lambda _4 +R_{on})}&{\color{Blue}V_1-20+I_S(\lambda _2 - R_{on})=0}\\
 &{\color{Red}y_3=R_{off}-\lambda _4 - R_{on}}&{\color{Blue}y_1=R_{off}-\lambda _2-R_{on}}\\
 &{\color{Red}y_4=-V_1+\lambda _3}&{\color{Blue}y_2=V_3-V_2+\lambda _1}\\
 \end{eqnarray*}
 \[0 \leq y_i \perp \lambda _i \geq 0\]
 }

\frame
{

\frametitle{Complementarity formulation.}
 \begin{figure}
 \input{CS}  
 \end{figure}

 }
