
\frame
{
\frametitle{A reformulation using the Fischer-Burmeister function}
\begin{block}{ The Fischer-Burmeister function}
 
\[\phi (a,b) = a+b-\sqrt{a^2+b^2} \textrm{ and the merit function: } \psi (a,b) = \| \psi(a,b)\|^2 \]
Note that:
\[  0 \le a \perp     b   \ge 0 \Longleftrightarrow \phi (a,b) =0 \Longleftrightarrow (a,b) \textrm{is a
global minimum of } \psi \]
\end{block}

 Properties of theses functions: $\phi$ is strongly semi-smooth, $\psi$ is continuously differentiable.

\begin{figure}[h]
\centerline{
 \scalebox{0.35}{
    \input{FischerB.pstex_t}
    \input{Psi.pstex_t}
 }
}
\end{figure}

 

}

\frame
{
\frametitle{A reformulation using the Fischer-Burmeister function}


\begin{block}{ MLCP reformulation}

\begin{equation}
  \begin{array}{cc}
   \begin{cases}
    A u + C \lambda + a =0 \\
    y=Du +B \lambda +b\\
    {0} \le \lambda \perp     y   \ge {0}
    \end{cases}&
   \begin{cases}
   \Phi :\mathbb{R}^{n+m} \longrightarrow \mathbb{R}^{n+m}\\
   \Phi(z)=\left(\begin{array}{c}
   \left(\begin{array}{cc}
   A&C
   \end{array}\right)z+a\\
   \phi(z_{n+1},(D_{1.}B_{1.})z+b_1)\\
   .\\
   .\\
   \phi(z_{n+m},(D_{m.}B_{m.})z+b_n)
   \end{array}\right)=0
    \end{cases}
  \end{array}
\end{equation}
This problem consists now in finding the global minimum of the merit function $\Psi(z)=\| \Phi(z)\|^2$.
 
\end{block}


}


%\frame
%{

%\begin{equation}
%  \begin{array}{c}
%   \begin{cases}
%    M1 \left(\begin{array}{c} u\\ v\end{array}\right)  + a =0 \\
%    {0} \le {v} \perp     M2\left(\begin{array}{c} u\\ v\end{array}\right) +b   \ge {0}\\
%    \textrm{ changement de variable : }
%    N \left(\begin{array}{c} \tilde{u}\\ \tilde{v}\end{array}\right)=  \left(\begin{array}{c} u\\ v\end{array}\right)\\
%    M1N \left(\begin{array}{c} \tilde{u}\\ \tilde{v}\end{array}\right)  + a =0 \\
%    {0} \le {\left(N \left(\begin{array}{c} \tilde{u}\\ \tilde{v}\end{array}\right)\right)_{n..n+m}} \perp     M2N\left(\begin{array}{c} \tilde{u}\\ \tilde{v}\end{array}\right) +b   \ge {0}\\
    
%    \end{cases}\\\\
%   \begin{cases}
%   \Phi :\mathbb{R}^{n+m} \longrightarrow \mathbb{R}^{n+m}\\
%   \Phi \left(\begin{array}{c} \tilde{u}\\ \tilde{v}\end{array}\right)=\left(\begin{array}{c}
  
%   M1N
%   \left(\begin{array}{c} \tilde{u}\\ \tilde{v}\end{array}\right)_i\\
%   \phi((N \left(\begin{array}{c} \tilde{u}\\ \tilde{v}\end{array}\right))_i,(M2N\left(\begin{array}{c} \tilde{u}\\ \tilde{v}\end{array}\right)_i)\\
%   .\\
%   .
%   \end{array}\right)=0
%    \end{cases}
%  \end{array}
%\end{equation}
%  reste a choisir N pour condiionner $\left(\begin{array}{cc} 0&I\\A&C\\D&B\end{array}\right)$


 
 


%}
