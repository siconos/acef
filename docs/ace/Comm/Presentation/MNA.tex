\frame
{
\newtheorem{mur}{A Current-Defined Branch (CD)}
\begin{mur}
The branch is current-defined if its current is a function of its own voltage, controlling variable
or their time--derivatives:
\begin{equation}\label{CD}I_{a}=F_{i}(U_{a},U_{b},I_{c},\frac{dU_a}{dt},\frac{dU_b}{dt},\frac{dI_{c}}{dt})\end{equation}
\end{mur}
\newtheorem{mur_}{A Voltage-Defined Branch (VD)}
\begin{mur_}
The branch is voltage-defined if its voltage is a function of its own current, controlling variable
or their derivatives:
\begin{equation}\label{VD}U_{a}=F_{u}(I_{a},U_{b},I_{c},\frac{dI_a}{dt},\frac{dU_b}{dt},\frac{dI_{c}}{dt})\end{equation}
\end{mur_}
Examples : \\
A resistor is a voltage-defined branch because $U_{a}=RI_{a}$.\\
A inductor is a voltage-defined branch because $U_{a}=L\frac{dI_{a}}{dt}$.\\
A capacitor is a current-defined branch because $I_{a}=C\frac{dU_{a}}{dt}$.\\

 \begin{block}{MNA Hypothesis:}
The M.N.A. assumes that smooth branches are explicit functions of current or voltage. It means each
branch is either Voltage Defined or Current Defined.
  \end{block}
}
\frame
{
\frametitle{MNA unknowns}
 \begin{block}{The vector of unknowns contains:}
\begin{itemize}
\item All node potentials $V_{i}$
\item Currents through all voltage defined branches
\end{itemize}
\end{block}

 \begin{block}{The following equations are written:}
\begin{itemize}
\item The KCL is applied to every node.
\item The Branch Constitutive Equation for all voltage defined branches.
\end{itemize}
\end{block}
Example:
  \begin{figure}[h]
   \centerline{
   \scalebox{0.5}{
    \input{LC.pstex_t}
  }
 } 
 \end{figure}

The vector of unknowns is :$(V_{1},I_{L})^{t}$
\[CV_{1}'-I_{L}=0 \qquad LI_{L}'+V_{1}=0\]

}