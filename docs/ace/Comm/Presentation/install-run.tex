\section{Install, run and doc}
\frame
{
  \frametitle{Installation of the platform}

  
  \begin{itemize}
  \item<1-> Autoconf, automake and libtool utilities for each package (Numerics, Kernel and Front-End): 
    configure.ac and Makefile.am files distributed with the software. 

  \item<2-> one "simple" way to install the platform: configure ; make ; make install. \\

  \item<3->Required or optional external libraries: lapack++, libxml2, cppunit ...
  \end{itemize}

}


\frame{
\frametitle{Running a simulation}

  \begin{itemize}
  \item direct c++ program writing
  \item mixed c++ program writing and xml input  
  \item python or scilab interface
  \end{itemize}

  $\Rightarrow$ input files: one main program (.cpp, .py ...), one plugin file (yourPlugin.cpp) and one xml input file. 
}


\frame{
\frametitle{Help and Documentation}

  \begin{itemize}
  \item<1-> Doxygen tools for automatic documentation in Numerics and Kernel
  \item<2-> Users, developers and theoretical manuals (in process ...)
  \item<3-> web pages, Bug tracker, forum ... on Gforge.  
  \end{itemize}
}

   



   

%%% Local Variables: 
%%% mode: latex
%%% TeX-master: t
%%% End: 
