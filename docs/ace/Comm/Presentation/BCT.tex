\frame
{
  \frametitle{Circuit branch}
\begin{figure}[h]
\centerline{
 \scalebox{0.4}{
    \input{Branch.pstex_t}
 }
}
\caption{Basic branch}
\end{figure}
A branch is described by a pair of branch variables: the tension $U_{a}$ and the current $I_{a}$.
Moreover, the Branch Constitutive Equation can be expressed in a general implicit form:
\begin{equation}\label{BCE}F(U_{a},I_{a},\frac{dU_{a}}{dt},\frac{dI_{a}}{dt},...)=0\end{equation}
Some examples with a resistor, a capacitor and a current source:
\[U_{a}=RI_{a} \qquad I_{a}=C\frac{dU_{a}}{dt} \qquad I_{a}=\alpha U_{b}\]
We will see the different forms of this relation.
  }

\frame
{
\newtheorem{kcl}{The Kirchhoff Current Law (KCL)}
\begin{kcl}
At any node in an electrical circuit where charge density is not changing in time, the sum of
currents flowing towards that node is equal to the sum of currents flowing away from that node.
\end{kcl}
\begin{figure}[h]
\centerline{
 \scalebox{0.35}{
    \input{SimpleCircuit.pstex_t}
 }
}
\end{figure}
With this example 
\[-I_{1}+I_{2}+I_{3}=0 \qquad (KCL1)\]
\[-I_{3}+I_{4}=0 \qquad (KCL2)\]
\[I_{1}-I_{2}-I_{4}=0 \qquad (KCL0)\]
or the matrix formulation :
\[AI=0\]
where A is known as the incidence matrix and I is the vector of branch currents.

}
\frame
{
\newtheorem{kvl}{The Kirchhoff Voltage Law (KVL)}
\begin{kvl}
The directed sum of the electrical differences around a closed circuit must be zero.
\end{kvl}
\begin{figure}[h]
\centerline{
 \scalebox{0.35}{
    \input{SimpleCircuit.pstex_t}
 }
}
\end{figure}
With this example 
\[U_{1}+U_{2}=0\]
\[ -U_{2}+U_{3}+U_{4}=0\]
\[ U_{1}+U_{3}+U_{4}=0\]
or the matrix formulation :
\[BU=0\]
where B is known as the loop matrix and U is the vector of branch tensions.
}

\frame
{
\begin{block}{Relation between A and B}
\[BA^{t}=0\]
\end{block}
\pause
Proof:
\[B=\left(\begin{array}{c}b_{1}\\.\\.\\b_{n_{l}}\end{array}\right)
\qquad A=\left(\begin{array}{c}KCL_{1}\\.\\.\\KCL_{n_{n}}\end{array}\right)\]
Let p $\in \lbrace 1,n_{l} \rbrace $ and q $\in \lbrace 1,n_{n} \rbrace $. We proof that
$b_{p}.KCL_{q}^{t}=0$.
\begin{figure}[h]
\centerline{
 \scalebox{0.5}{
    \input{loop.pstex_t}
 }
}
\end{figure}

\begin{enumerate}
\item The node $_{q}$ is not on the loop $_{p}$, then $b_{p}$ and $KCL_{q}$ have no common non null coordinate.
\item The node $_{q}$ is on the loop $_{p}$:
\[b_{p}=(...1...-1...)\qquad KCL_{q}=(...1...1...)\]
The other coordinates are not simultaneous positive. Therefore :
\[b_{p}.KCL_{q}^{t}=0\]
\end{enumerate}

}
\frame
{
\frametitle{An other KVL formulation}
\begin{figure}[h]
\centerline{
 \scalebox{0.5}{
    \input{branchb.pstex_t}
 }
}
\end{figure}
It consists in writing :
\[\forall p \in \lbrace 1,n_{b} \rbrace \qquad U_{p}=V_{j_{p}}-V_{k_{p}}\]
This system of equation implies BU=0.
\begin{block}{The matrix formulation is:}
\[U-A^{t}V=0\]
where V is the vector of nodes potential and U the vector of branches tensions.\\
\end{block}
$\Rightarrow$ In further consideration, BU=0 is eliminated.\\
Example:
\[U_{1}=V_{1} - V_{0} \qquad U_{2} = V_{0}-V_{1}\]
\[U_{3}=V_{2} - V_{1} \qquad U_{4} = V_{0}-V_{2}\]
}


\frame
{
  \begin{block}{The Sparse Tableau Approach STA leads to the following system:}
  \[AI=0 \qquad (KCL)\]
  \[U-A^{t}V=0 \qquad (KVL)\]
  \[\textrm{For all branches :} \qquad F(U_{a},I_{a},...)=0 \qquad(BCE) \]
  \end{block}  


Example:
  \begin{figure}[h]
   \centerline{
   \scalebox{0.5}{
    \input{LC.pstex_t}
  }
 } 
 \end{figure}
 The vector of unknowns: $(I_{L},I_{C},U_{L},U_{C},V_{1})^{t}$
 \[-I_{L} + I_{C} = 0 \qquad (KCL1)\]
 \[U_{L} + V_{1} = 0 \qquad U_{C} -V_{1}=0\qquad (KVL)\]
 \[CU_{C}'-I_{C}=0 \qquad LI_{L}'-U_{L}=0\qquad (BCE)\]
 
  }

%%% Local Variables: 
%%% mode: latex
%%% TeX-master: "main"
%%% End: 
