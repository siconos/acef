\frame
{
  \frametitle{Mixed Linear Complementarity Problem solvers}

}

\frame
{
\frametitle{MLCP to linear system}
\begin{block}{ Given a MLCP}
 \begin{equation}\label{eq:mlcp1}
 \begin{cases}
M \left(\begin{array}{c}
   u\\
   \lambda
   \end{array}\right) + q = \left(\begin{array}{c}
   0\\
   y
   \end{array}\right) \\
      0 \le \lambda \perp     y   \ge 0

      \end{cases}
\end{equation}


\end{block}

\begin{block}{A linear system}

Let $I_\lambda$ and $I_y$, tow complementary subsets of $\{1,..,m\}$ and
looking for a solution such that : \\
\[
\begin{cases}
\lambda_i=0 \textrm{ for } i \in I_\lambda \\
 y_i=0 \textrm{ for } i \in I_y
\end{cases}
 \]




\[\textrm{Solve the linear system: }N \left(\begin{array}{c}
   u\\
   \lambda y
   \end{array}\right) = -q   \textrm{   With   }
   \begin{cases}
   N_{.i}=M_{.i} \textrm{ for } i \in \{1..n\}\\
   N_{.i}=M_{.i} \textrm{ for } i-n \in I_y\\
   N_{.i}=-e_{i} \textrm{ for } i-n \in I_\lambda
\end{cases}
\]
\[ \textrm { Solution if the complementarity hold: }
\begin{cases}
y_i \ge 0 \textrm{ for } i \in I_\lambda  \\
\lambda_i \ge 0 \textrm{ for } i \in I_y
\end{cases}
\]
An enumeratif algorithm consists in solving the $2^m$ cases.

\end{block}

}
