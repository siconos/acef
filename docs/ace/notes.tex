
\newpage



\section{ACEF to Siconos}
\subsection{The MLCS formulation}
\subsubsection{The MLCS from the automatic formulation}
 \begin{eqnarray}
A_{x'}x'=A_{x}x +A_{z}Z_{s} +A_{\lambda} \lambda +a(t)&\label{eq2}\\
0=B_{x}x+B_{z}Z_{s} + B_{\lambda}\lambda + b(t)&\label{eq3}\\
Y=C_{x}x+C_{z}Z_{s}+C_{\lambda}\lambda + c(t) &\label{eq4}\\
0 \leq Y \, \perp \, \lambda \geq 0&\label{eqperp}
\end{eqnarray}
 With $A_{x'}$ regular.\\
 Let
 \begin{itemize}
  \item[--] Let n the x dimension.
  \item[--] Let m the Y dimension.
  \item[--] Let s the $Z_{s}$ dimension.
\end{itemize}
\subsubsection{The MLCS formulation for siconos}
It consists in concataining $Z_{s}$ and $ \lambda $ , it rises to the following system:
 \begin{eqnarray}
A_{x'}x'=A_{x}x +A_{\lambda} \lambda +a(t)&\label{eq2}\\
0=B_{x}x+ B_{\lambda}\lambda + b(t)&\label{eq3}\\
Y=C_{x}x+C_{\lambda}\lambda + c(t) &\label{eq4}\\
(\lambda_{1},..,\lambda_{s}) free\\
0 \leq Y \, \perp \, (\lambda_{s+1},..,\lambda_{s+m})^{t} \geq 0&\label{eqperp}
\end{eqnarray}


 \subsubsection{With the siconos convention}
 
The Dynamic system:
\[Mx'=Ax +b(t)+r\]
The First Order Linear Relation:
\[Y=CX +D \lambda\]
\[R=B \lambda\]
The complementary relation and interaction:
\[(Y_{1},..,Y_{s})^{t}=0\]
\[(\lambda_{1},..,\lambda_{s}) \qquad free\]
\[0 \leq (Y_{s+1},..,Y_{s+m})^{t} \, \perp \, (\lambda_{s+1},..,\lambda_{s+m})^{t} \geq 0\]

\subsubsection{Building a NonSmooth Dynamical System}
 
 aDS = new FirstOrderLinearDS$(int,A_{x'},A_{x},a(t))$;\\
 aTIR = new FirstOrderLinearTIR$(\left(\begin{array}{c}B_{x}\\C_{x}
 \end{array}\right),0,\left(\begin{array}{c}B_{\lambda}\\C_{\lambda}
 \end{array}\right),\left(\begin{array}{c}b(t)\\c(t) \end{array}\right),A_{\lambda})$\\

 aNSL= new complementarityConditionNSL(m);\\
 aI = new interaction(``MLCP'',allDS,1,m+s, aNSL, aTIR);\\
 aNSDS= new NonSmoothDynamicalSystem(aDS, aI);\\
 aM=new Model($t_{0}$,T);\\
 aM setNSDS(aNSDS);\\
\\
\subsubsection{Simulation}
aTD = new TimeDiscretisation(h,aM);\\
aS = new TimeStepping(aTD);\\
aMoreau = new Moreau(aDS,0.5,aS);\\
aMLCP = new MLCP(aM,mySolver,''MLCP'');\\
aS init();\\
aS run();\\

\section{Initial conditions}
\subsection{Initial conditions in NGSPICE}
In spice, the initial conditions are about the Voltage node. It is not possible to specify a
current. The DC analysis consists in computing the operating point of the circuit with inductors
shorted and capacitor opened. There are three ways to define the initial conditions(description from NGSPICE User Manual):

\subsubsection{.NODESET: Specify initial node voltage guesses}
It consists in doing a DC analysis. The initial node voltage are used to initialize the DC analysis,
but the iteration continues to the true solution.
\subsubsection{.IC: set initial condition }
It consists in doing a DC analysis. But the specify node voltage values are forced during the DC
iterations. It is the preferred method since it allows NGSPICE to compute a consistent DC solution.
\subsubsection{.IC and UIC: set initial condition }
In this case there are no DC analysis. The specified values are directly used for the trans
analysis.

\subsection{Initial conditions in ACEF}
In current version r84.\\
\subsubsection{.IC interpretation}
In ACEF there are no DC analysis. The initial node voltage values are used to initialize the
dynamical unknowns of type TENSION.
\subsubsection{It could be better to do}
First point, note that inductors shorted and capacitor opened means $\frac{dx}{dt}=0$. Because:\\
The capacitor constitutive law is : $C\frac{dU}{dt}=I$. If the capacitor branch is opened, it means
$I=0$. So $C\frac{dU}{dt}=0$.\\
The inductor constitutive law is : $L\frac{dI}{dt}=U$. If the inductor branch is shorted, it means
$U=0$. So $C\frac{dI}{dt}=0$.\\
Second point, all the node potential are unknowns included in the $Z_{s}$ vector with ACEF, or in
$\lambda$ vector with current version of SICONOS.\\

It leads to the following system:

 \begin{eqnarray}
0=A_{x}x_{0} +A_{\lambda} \lambda +a(t)&\label{eq2}\\
0=B_{x}x_{0}+ B_{\lambda}\lambda + b(t)&\label{eq3}\\
Y_{0}=C_{x}x_{0}+C_{\lambda}\lambda + c(t) &\label{eq4}\\
\lambda_{i} = \lambda_{i0} \qquad for \qquad i \in \{ user initial value \}\\
\lambda_{i} \qquad free \qquad i \notin \{ user initial value \}\\
0 \leq Y_0 \, \perp \, (\lambda_{s+1},..,\lambda_{s+m})^{t} \geq 0&\label{eqperp}
\end{eqnarray}
The solution $x_{0}$ and $\lambda_{1}...\lambda_{s}$ are the initial values for the trans analysis.

