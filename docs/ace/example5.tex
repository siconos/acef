 
\subsection{One  example with  a capacitor  loop}

The first paragraph of this section shows that standard MNA leads to a full implicit DAE. In the second
paragraph the previous algorithms are applied and leads to a semi-explicit system.
\begin{figure}[h]
\centerline{
 \scalebox{0.6}{
    \input{../ace/cir5.pstex_t}
 }
}\end{figure}\\


\paragraph{Standard MNA Algorithm}
It consists in writing the system with the following unknowns $x=^{t}(U_{12},U_{23},U_{34},U_{41})$$z=(V_{1},V_{2},V_{3},V_{4},V_{5},I_{50})$\\
Start to write $M \dot x=...$\
\[\left(\begin{array}{c}
  \\
KCL(1)\\KCL(2)\\KCL(3)\\KCL(4)
\end{array}\right)
\left(\begin{array}{cccc}
  \dot U_{12}&\dot U_{23}&\dot U_{34}&\dot U_{41}\\
  \hline
  C&0&0&-C\\
  -C&C&0&0\\
  0&-C&C&0\\
  0&0&-C&C\\  
\end{array}\right) \dot x= RHS
\]
The obtained matrix is not regular because of the cycle \{1-2,2-3,3-4,4-1\}, so it leads to a
implicit DAE. The Right-hand-side  $RHS$ is not here detailed. 



\paragraph{The proposed Algorithm~\ref{Algo:ACEF1}}
The previous algorithms are now performed. The chosen solution is to use a Spanning Tree \{1-2,2-3,3-4\} to write the KCL law. About the capacitive branch (41), $I_{41}$ is
added in the vector of unknowns z.
The algorithm \ref{Build a regular N} provides the following semi-explicit system: 
\[\left(\begin{array}{c}
  \\
KCL(1)\\KCL(2)\\KCL(3)\\I_{41}
\end{array}\right)
\left(\begin{array}{cccc}
  \dot U_{12}&\dot U_{23}&\dot U_{34}&\dot U_{41}\\
  \hline
  C&0&0&-C\\
  -C&C&0&0\\
  0&-C&C&0\\
  0&0&0&C\\  
\end{array}\right) \dot x=0x+
\left(\begin{array}{ccccccc}
  V_{1}&V_{2}&V_{3}&V_{4}&V_{5}&I_{50}&I_{41}\\
  \hline
  -\frac{1}{R}&0&0&0&0&0&0\\
  0&0&0&0&0&0&0\\
  0&0&0&\frac{1}{R}&-\frac{1}{R}&0&0\\
  0&0&0&0&0&0&1\\
\end{array}\right)z
\]
Multiply this line by $N^{-1}$ and all the currents are written like a linear combination of the
unknowns:
\begin{enumerate}
\item [--] $I_{41}$ is an unknown.
\item [--] $I_{12}=I_{41}-\frac{V_1}{R}$.
\item [--] $I_{23}=I_{41}-\frac{V_1}{R}$.
\item [--] $I_{43}=I_{41}-\frac{V_1}{R} - \frac{V_3-V_5}{R}$.
\end{enumerate}

The last step consists in writing the missing equations, using the current expression getting from the
 previous system:
 \[\left(\begin{array}{c}
  \\
KCL(4)\\KCL(5)\\U_{12}\\U_{23}\\U_{34}\\U_{41}\\VD_{50}
\end{array}\right)
 \left(\begin{array}{cccc}
  U_{12}&U_{23}&U_{34}&U_{41}\\
  \hline
  0&0&0&0\\
  0&0&0&0\\
  1&0&0&0\\
  0&1&0&0\\
  0&0&1&0\\
  0&0&0&1\\
  0&0&0&0\\

\end{array}\right)x+
 \left(\begin{array}{ccccccc}
  V_{1}&V_{2}&V_{3}&V_{4}&V_{5}&I_{50}&I_{41}\\
  \hline
  \frac{1}{R}&0&-\frac{1}{R}&0&\frac{1}{R}&0&\underline{-1+1}\\
  0&0&\frac{1}{R}&0&-\frac{1}{R}&-1&0\\
  -1&1&0&0&0&0&0\\
  0&-1&1&0&0&0&0\\
  0&0&-1&1&0&0&0\\
  0&0&0&-1&1&0&0\\
  0&0&0&0&1&0&0\\
\end{array}\right)z=
 \left(\begin{array}{c}
  \\0\\0\\0\\0\\0\\0\\E
\end{array}\right)
 \]
$N_{I}=N_{E}=11$\\
Note that the first line KCL(1) is the fact that the current that
go in the capacitor cycle is equals to the current that go out.
