 

\section{Examples}

This section illustrates the rules how get $x' = A_{1x}x +A_{1zs}Z_{s} + A_{1ns}Z_{ns}+A_{1s}$? To get this system, a good set of unknowns must be done. Only the necessary unknowns are added. The
following examples show how this choice could be done.
\subsection{Example 1}
\begin{figure}[h]
\centerline{
 \scalebox{0.5}{
    \input{../ace/cir1.pstex_t}
 }
}
\end{figure}
\paragraph{First matrices formulation}
X=$^{t}(V_{1},V_{2},V_{3})$\\
\[\left(\begin{array}{c}
  \\
  KCL(1)\\  KCL(2)\\  KCL(3)
  \end{array}\right)
\left(\begin{array}{ccc}
  V_{1}&V_{2}&V_{3}\\
  \hline
  \frac{1}{R}-\frac{1}{R}&  \frac{-1}{R}&0\\
  \frac{1}{R}&  \frac{-1}{R}&0\\
  0&0&\frac{1}{R}
\end{array}\right)X+
\underline{
\left(\begin{array}{ccc}
   V_{1}'&V_{2}'&V_{3}'\\
  \hline
0&0&0\\
  0&C&-C\\
  0&-C&C
\end{array}\right)}X'=
\left(\begin{array}{c}
  \\
  I\\
  0\\
  0
  \end{array}\right)
\]
But, we can't extract x from X to get x'=...\\
\newline
\paragraph{Add current and tension from the capacitor}
x=$(U_{32})$
$Z_{s}=^{t}(V_{1},V_{2},V_{3},I_{32})$\\
\[x'=CI_{32}\]
\[\left(\begin{array}{c}
  \\
  KCL(1)\\
  KCL(2)\\
  KCL(3)\\
  U_{32}
  \end{array}\right)
\left(\begin{array}{cccc}
  V_{1}&V_{2}&V_{3}&I_{32}\\
  \hline
  \frac{1}{R}-\frac{1}{R}&  \frac{-1}{R}&0&0\\
  \frac{1}{R}&  \frac{-1}{R}&0&1\\
  0&0&\frac{1}{R}&-1\\
  0&-1&1&0
\end{array}\right)Z_{s}+
\left(\begin{array}{c}
  U_{32}\\
  \hline
  0\\
  0\\
  0\\
  1
  \end{array}\right)x
=
\left(\begin{array}{c}
  \\
  I\\
  0\\
  0\\
  0
  \end{array}\right)
\]
We obtain the matrices system\\
\[x'=A1_{zs}Z_{s}\]
\[Ax+CZ_{s}=s\]
$N_{I}=N_{E}=5$\\
The following examples, show we don't need to add the capacitor current. Sometime, current in
capacitor can be replace with a KCL law.\\

\paragraph{Add only tension from the capacitor}
x=$(U_{32})$,$Z_{s}=^{t}(V_{1},V_{2},V_{3})$\\
First get x' with \underline{KCL(3)}:
\[x'=\frac{V_{3}}{RC}\]
Second, the current in capacitor is $\frac{V_{3}}{R}$, and use it to write other KCL law:
\[\left(\begin{array}{c}
  \\
  KCL(1)\\
  KCL(2)\\
  U_{32}
  \end{array}\right)
\left(\begin{array}{ccc}
V_{1}&V_{2}&V_{3}\\
  \hline
  \frac{1}{R}+\frac{1}{R}&  \frac{-1}{R}&0\\
  \frac{-1}{R}&  \frac{1}{R}&\frac{1}{R}\\
  0&-1&1
\end{array}\right)Z_{s}+
\left(\begin{array}{c}
U_{32}\\
  \hline
  0\\
  0\\
  0\\
  1
  \end{array}\right)x
=
\left(\begin{array}{c}
  \\
  I\\
  0\\
  0\\
  0
  \end{array}\right)
\]
$N_{I}=N_{E}=4$
\newpage

 
\subsubsection{Example 2}
\begin{figure}[h]
\centerline{
 \scalebox{0.9}{
    \input{cir2.pstex_t}
 }
}
\end{figure}
\paragraph{Add currents and tensions from the capacitor}
$x=^{t}(I_{42},U_{43},U_{31},U_{50})$,
$Z_{s}=^{t}(V_{1},V_{2},V_{3},V_{4},V_{5},I_{43},I_{31},I_{50})$
We obtain following equation:
\[\left(\begin{array}{cccc}
  I_{42}'&U_{43}'&U_{31}'&U_{50}'\\
  \hline
L&0&0&0\\
0&C&0&0\\
0&0&C&0\\
0&0&0&C
\end{array}\right)x'=
0x+
\left(\begin{array}{cccccccc}
  V_{1}&V_{2}&V_{3}&V_{4}&V_{5}&I_{43}&I_{31}&I_{50}\\
  \hline
  0&1&0&-1&0&0&0&0\\
  0&0&0&0&0&-1&0&0\\
  0&0&0&0&0&0&1&0\\
  0&0&0&0&0&0&0&1\\
\end{array}\right)Z_{s}
\]
\[\left(\begin{array}{c}
\\KCL(1)\\KCL(2)\\KCL(3)\\KCL(4)\\KCL(5)\\U_{43}\\U_{31}\\U_{50}
\end{array}\right)
\left(\begin{array}{cccccccc}
  V_{1}&V_{2}&V_{3}&V_{4}&V_{5}&I_{43}&I_{31}&I_{50}\\
  \hline
  -\frac{1}{R}&\frac{1}{R}&0&0&0&0&1&0\\
  \frac{1}{R}&-\frac{1}{R}&0&0&0&0&0&0\\
  0&0&0&0&0&-1&1&0\\
  0&0&0&-\frac{1}{R}&\frac{1}{R}&1&0&0\\
  0&0&0&\frac{1}{R}&-\frac{1}{R}&0&0&1\\
  0&0&-1&1&0&0&0&0\\
  -1&0&1&0&0&0&0&0\\
  0&0&0&0&1&0&0&0\\
\end{array}\right)Z_{s}+
\left(\begin{array}{cccc}
  I_{42}&U_{43}&U_{31}&U_{50}\\
  \hline
  0&0&0&0\\
  1&0&0&0\\
  0&0&0&0\\
  0&0&0&0\\
  1&0&0&0\\
  0&1&0&0\\
  0&0&1&0\\
  0&0&0&1\\
\end{array}\right)x=
\left(\begin{array}{c}
  \\ I\\  0\\  0\\  0\\  0\\  0\\  0\\  0
  \end{array}\right)
\]

\[x'=A1_{zs}Z_{s}\]
\[Cx+BZ_{s}=s\]
$N_{I}=N_{E}=12$
\paragraph{Add only tensions from the capacitor}
$x=^{t}(I_{42},U_{43},U_{31},U_{50})$,
$Z_{s}=^{t}(V_{1},V_{2},V_{3},V_{4},V_{5})$\\
We obtain following equation:
\[
\left(\begin{array}{c}
  \\  KCL(4)\\  KCL(3)\\  KCL(5)
\end{array}\right)
\underline{
\left(\begin{array}{cccc}
  I_{42}'&U_{43}'&U_{31}'&U_{50}'\\
  \hline
L&0&0&0\\
0&C&0&0\\
0&C&C&0\\
0&0&0&C
\end{array}\right)}x'=
\left(\begin{array}{cccc}
  I_{42}&U_{43}&U_{31}&U_{50}\\
  \hline
0&0&0&0\\
1&0&0&0\\
0&0&0&0\\
0&0&0&0\\ 
\end{array}\right)x+
\left(\begin{array}{ccccc}
V_{1}&V_{2}&V_{3}&V_{4}&V_{5}\\
  \hline
  0&1&0&-1&0\\
  0&0&0&\frac{1}{R}&-\frac{1}{R}\\
  0&0&0&0&0\\
  0&0&0&-\frac{1}{R}&\frac{1}{R}
\end{array}\right)Z_{s}
\]
So, we get $x'=A1_{x}x+A1_{zs}Z_{s}$. Therefore, all currents in the capacitor branch are known.\\
$I_{43}= I_{42}+\frac{V_{4}}{R}-\frac{V_{5}}{R}$,
$I_{31}= I_{42}+\frac{V_{4}}{R}-\frac{V_{5}}{R}$,
$I_{50}= \frac{V_{4}}{R}-\frac{V_{5}}{R}$\\
Use these equations to fill following matrices:
\[\left(\begin{array}{c}
KCL(1)\\KCL(2)\\U_{43}\\U_{31}\\U_{50}
\end{array}\right)
\left(\begin{array}{ccccc}
V_{1}&V_{2}&V_{3}&V_{4}&V_{5}\\
  \hline
  -\frac{1}{R}&\frac{1}{R}&0&\underline{\frac{1}{R}}&\underline{-\frac{1}{R}}\\
  \frac{1}{R}&-\frac{1}{R}&0&0&0\\
  0&0&-1&1&0\\
  -1&0&1&0&0\\
  0&0&0&0&1\\
\end{array}\right)Z_{s}+
\left(\begin{array}{cccc}
  I_{42}&U_{43}&U_{31}&U_{50}\\
  \hline
  \underline{1}&0&0&0\\
  1&0&0&0\\0&1&0&0\\0&0&1&0\\0&0&0&1\\
\end{array}\right)x=
\left(\begin{array}{c}
  I\\0\\0\\0\\0
  \end{array}\right)
\]
$N_{I}=N_{E}=9$


\subsection{Example 3}
\begin{figure}[h]
\centerline{
 \scalebox{0.8}{
    \input{cir3.pstex_t}
 }
}
\end{figure}

x=$(U_{12})$\\
$Z_{s}=^{t}(V_{1},V_{2},I_{10})$\\
\[(KCL(1))=>(C1+C2+C3)x'=I_{10}\]
So, all capacitor's currents are known:\\
$I_{c1}=\frac{C1}{C1+C2+C3}I_{10}$\\
$I_{c2}=\frac{C2}{C1+C2+C3}I_{10}$\\
$I_{c3}=\frac{C3}{C1+C2+C3}I_{10}$\\
Use these equations to fill following matrices:\\
\[\left(\begin{array}{c}
  \\
KCL(2)\\U_{21}\\VD
\end{array}\right)
\left(\begin{array}{c}
U_{21}\\
\hline
0\\
1\\
0
\end{array}\right)x+
\left(\begin{array}{ccc}
V_{1}&V_{2}&I_{10}\\
\hline
0&\frac{1}{R}&\frac{C1}{C1+C2+C3}+\frac{C2}{C1+C2+C3}+\frac{C3}{C1+C2+C3}\\
1&-1&0\\
1&0&0
\end{array}\right)Z_{s}=
\left(\begin{array}{c}
\\0\\0\\E
\end{array}\right)
\]
$N_{I}=N_{E}=4$
\newpage

\subsection{Example 4}
This example shows that is not always possible to use the KCL law to get x'=...\\
\begin{figure}[h]
\centerline{
 \scalebox{0.9}{
    \input{cir4.pstex_t}
 }
}\end{figure}\\
$x=^{t}(U_{21},U_{23},U_{34})$
$Z_{s}=^{t}(V_{1},V_{2},V_{3},V_{4})$
\paragraph{a problem:}
Start to fill the x' matrices. We use KCL(2) for $U_{21}$, and KCL(3) for $U_{34}$.
\[\left(\begin{array}{c}
  \\
KCL(2)\\KCL(3)\\??
\end{array}\right)
\left(\begin{array}{ccc}
  U_{21}'&U_{23}'&U_{34}'\\
  \hline
  C&-C&0\\
  0&C&C\\
  ?&?&?
\end{array}\right)x'=0x+
\left(\begin{array}{cccc}
  V_{1}&V_{2}&V_{3}&V_{4}\\
  \hline
  0&0&0&0\\
  0&0&0&0\\
  ?&?&?&?\\
\end{array}\right)Z_{s}
 \]
 We can't use a other KCL law to get $U_{23}$. A solution could be to add an unknown, \underline{$I_{23}$} in
 $Z_{s}$. The system becomes:\\
 \[\left(\begin{array}{c}
  \\
KCL(2)\\KCL(3)\\I_{23}
\end{array}\right)
\left(\begin{array}{ccc}
  U_{21}'&U_{23}'&U_{34}'\\
  \hline
  C&-C&0\\
  0&C&C\\
  0&C&0
\end{array}\right)x'=0x+
\left(\begin{array}{ccccc}
  V_{1}&V_{2}&V_{3}&V_{4}&\underline{I_{23}}\\
  \hline
  0&0&0&0&0\\
  0&0&0&0&0\\
  0&0&0&0&1\\
\end{array}\right)Z_{s}
 \]
 So, we obtain $x'=A1_{zs}Z_{s}$. All capacitor's currents are known:
 $I_{21}=\underline{I_{23}}=I_{34}$\\
\[\left(\begin{array}{c}
  \\
KCL(4)\\KCL(1)\\U_{21}\\U_{23}\\U_{34}
\end{array}\right)
\left(\begin{array}{ccc}
  U_{21}&U_{23}&U_{34}\\
  \hline
  0&0&0\\
  0&0&0\\
  1&0&0\\
  0&1&0\\
  0&0&1
\end{array}\right)x+
\left(\begin{array}{cccc}
  V_{1}&V_{2}&V_{3}&\underline{I_{23}}\\
  \hline
  0&0&0&1\\
  \frac{1}{R}&0&0&1\\
  -1&1&0&0\\
  0&-1&1&0\\
  0&0&-1&1\\
\end{array}\right)Z_{s}=
\left(\begin{array}{c}
\\I\\0\\0\\0\\0
\end{array}\right)
\]
 $N_{I}=N_{E}=8$\\
\paragraph{a good choice:}
\[\left(\begin{array}{c}
  \\
KCL(1)\\KCL(2)\\KCL(3)
\end{array}\right)
\left(\begin{array}{ccc}
  U_{21}'&U_{23}'&U_{34}'\\
  \hline
  C&0&0\\
  C&C&0\\
  0&-C&C
\end{array}\right)x'=0x+
\left(\begin{array}{cccc}
  V_{1}&V_{2}&V_{3}&V_{4}\\
  \hline
  \frac{1}{R}&-\frac{1}{R}&0&0\\
  0&0&0&0\\
  0&0&0&0\\
\end{array}\right)Z_{s}
 \]
 So, we obtain $x'=A1_{zs}Z_{s}$. Therefore all capacitor's currents are known: $I_{12}=I_{23}=I_{34}=\frac{V1}{R1}$\\
 \[\left(\begin{array}{c}
  \\
KCL(4)\\U_{21}\\U_{23}\\U_{34}
\end{array}\right)
 \left(\begin{array}{ccc}
  U_{21}&U_{23}&U_{24}\\
  \hline
  0&0&0\\
  1&0&0\\
  0&1&0\\
  0&0&1\\
\end{array}\right)x+
 \left(\begin{array}{cccc}
  V_{1}&V_{2}&V_{3}&V_{4}\\
  \hline
  \frac{1}{R}&0&0&0\\
  -1&1&0&0\\
  0&-1&1&0\\
  0&0&-1&1\\
\end{array}\right)Z_{s}=
 \left(\begin{array}{c}
  \\I\\0\\0\\0
\end{array}\right)
 \]
$N_{I}=N_{E}=7$\\






 
\subsection{One  example with  a capacitor  loop}

This example shows we have to manage the capacitor cycle.\\
\begin{figure}[h]
\centerline{
 \scalebox{0.6}{
    \input{../ace/cir5.pstex_t}
 }
}\end{figure}\\


\paragraph{Standard MNA Algorithm}
 \texttt{texte ::} $x=^{t}(U_{12},U_{23},U_{34},U_{41})$$z=(V_{1},V_{2},V_{3},V_{4},V_{5},I_{50})$\\
Start to write Mx'=...\
\[\left(\begin{array}{c}
  \\
KCL(1)\\KCL(2)\\KCL(3)\\KCL(4)
\end{array}\right)
\left(\begin{array}{cccc}
  U_{12}'&U_{23}'&U_{34}'&U_{41}'\\
  \hline
  C&0&0&-C\\
  -C&C&0&0\\
  0&-C&C&0\\
  0&0&-C&C\\  
\end{array}\right)x'= RHS
\]
where the Right-hand-side  $RHS$ is not here detailed.
This matrix is not regular because of the cycle \{1-2,2-3,3-4,4-1\}. 



\paragraph{The proposed Algorithm~\ref{Algo:ACEF1}}

A solution could be to use the Minimum
Spanning Tree \{1-2,2-3,3-4\} to write the KCL law. About the last tension, $U_{41}$, there are tow
ways:
\begin{enumerate}
\item add a unknown $I_{41} in z$ and write CU'=I
\item Find the linear relation $U_{41}'= \sum_{jk}^{}a_{jk}U_{kj}'$, and replace $U_{ki}'$.
\end{enumerate}
The matrices become:\\
Start to write Mx'=...\
\[\left(\begin{array}{c}
  \\
KCL(1)\\KCL(2)\\KCL(3)\\I_{41}
\end{array}\right)
\left(\begin{array}{cccc}
  U_{12}'&U_{23}'&U_{34}'&U_{41}'\\
  \hline
  C&0&0&-C\\
  -C&C&0&0\\
  0&-C&C&0\\
  0&0&0&C\\  
\end{array}\right)x'=0x+
\left(\begin{array}{ccccccc}
  V_{1}&V_{2}&V_{3}&V_{4}&V_{5}&I_{50}&I_{41}\\
  \hline
  -\frac{1}{R}&0&0&0&0&0&0\\
  0&0&0&0&0&0&0\\
  0&0&0&\frac{1}{R}&-\frac{1}{R}&0&0\\
  0&0&0&0&0&0&1\\
\end{array}\right)z
\]
So, we obtain $x'=A1_{z}z$. Therefore all capacitor's currents are
known:$I_{12}=I_{23}=I_{41}+\frac{V1}{R},I_{43}=I_{41}+\frac{V1}{R}+\frac{V5}{R}-\frac{V3}{R}$\\
The last step consists in writing the missing equations:
 \[\left(\begin{array}{c}
  \\
KCL(4)\\KCL(5)\\U_{12}\\U_{23}\\U_{34}\\U_{41}\\VD_{50}
\end{array}\right)
 \left(\begin{array}{cccc}
  U_{12}&U_{23}&U_{34}&U_{41}\\
  \hline
  0&0&0&0\\
  0&0&0&0\\
  1&0&0&0\\
  0&1&0&0\\
  0&0&1&0\\
  0&0&0&1\\
  0&0&0&0\\

\end{array}\right)x+
 \left(\begin{array}{ccccccc}
  V_{1}&V_{2}&V_{3}&V_{4}&V_{5}&I_{50}&I_{41}\\
  \hline
  \frac{1}{R}&0&-\frac{1}{R}&0&\frac{1}{R}&0&\underline{-1+1}\\
  0&0&\frac{1}{R}&0&-\frac{1}{R}&-1&0\\
  -1&1&0&0&0&0&0\\
  0&-1&1&0&0&0&0\\
  0&0&-1&1&0&0&0\\
  0&0&0&-1&1&0&0\\
  0&0&0&0&1&0&0\\
\end{array}\right)z=
 \left(\begin{array}{c}
  \\0\\0\\0\\0\\0\\0\\E
\end{array}\right)
 \]
$N_{I}=N_{E}=11$\\


\subsection{Automatic circuit equation formulation Algorithm}

\begin{algorithm}
\caption{fill the matrices : $x'=A_{1x}x+A_{1s}Z_{s}+A_{1ns}Z_{ns}$ }
\begin{algorithmic}
\REQUIRE Init\_I\_in\_x : initialize internal data structure to get all I from x.
\REQUIRE Next\_I\_in\_x : return the next available I from x. If there are not available x, return 0.\\
\REQUIRE Minimum Spanning Tree of the capacitor' tension graph
\REQUIRE Init\_MST : initialize internal data structure to get all u from x.
\REQUIRE Next\_u\_in\_MST : return a available U's neighbour from MST if possible. If there are not
available u in MST, return 0.\\
\REQUIRE Next\_U\_in\_x : return the next available U from x. If there are not available x, return 0.\\

0\\
\COMMENT{About current}\\
\COMMENT{get first current}
\STATE Init\_I\_in\_x()
\STATE $I_{kj}$ = Next\_I\_in\_x()
\WHILE{$I_{kj}$}
\STATE use $LI'=V_{j}-V_{k}$ to fill $I_{kj}$'s line.
\ENDWHILE\\
\COMMENT{About tension}\\
\COMMENT{get a first capacitor tension}
\STATE Init\_MST()
\STATE $U_{kj}$ = Next\_u\_in\_MST()
\WHILE{$U_{kj}$}
\STATE l=j or k with KCL(k) available.
\STATE Use CU'=I and KCL(k) to fill $U_{ki}$'s line.\\
\STATE enable KCL(k)

\STATE $U_{kj}$= Next\_u\_in\_MST ()
\ENDWHILE
\STATE $U_{kj}$ = Next\_U\_in\_x()
\WHILE {$U_{kj}$}
\STATE Add an unknown $I_{kj}$, and use it to fill the matrices. Write I=CU'.
\STATE $U_{kj}$ = Next\_U\_in\_x()
\ENDWHILE
\COMMENT{reverse A : $Ax'=Bx+CZ_{s}+DZ_{ns}$\\}
\STATE $A^{-1}$=Inv(A)
\STATE $x'=A_{1x}x+A_{1s}Z_{s}+A_{1ns}Z_{ns}$
\end{algorithmic}
\end{algorithm}



\subsection{conclusion}
The vector x contains inductor's currents and capacitor's tensions. Derivate inductor's current is equal to a
nodal voltage difference.\\
About the capacitor's tensions, we use the Minimum Spanning Tree of the capacitor' tension to avoid cycle.\\


%%% Local Variables: 
%%% mode: latex
%%% TeX-master: "ace"
%%% End: 

