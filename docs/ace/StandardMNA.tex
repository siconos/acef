
 
\section{Introduction, Modified Nodal Analysis}
This part is a short overview about the Modified Nodal Analysis(MNA). The MNA is the method used
in SPICE to obtain the circuit equation formulation. For more details, we refer the reader to [1][2].%~\cite{Ogrodzki1994,chua1991}.

\subsection{Basics on circuit topology}

Circuits are basically composed of branches, characterized by a current and a related voltage.
Moreover, branches are connected to circuit nodes. The physical behavior of the circuit obeys
Kirchhoff's law. Circuit analysis leads to a mathematical formulation which will be describe in this
document.

\subsubsection{Circuit branches}
\begin{figure}[h]
\centerline{
 \scalebox{0.6}{
    \input{Branch.pstex_t}
 }
}
\caption{Basic branch}
\label{fig-Basic-branch}
\end{figure}
In any case, the branch is described by a pair of branch variables: the tension $U_{a}$ and the current $I_{a}$.
Moreover, the Branch Constitutive Equation can be expressed in a general implicit form:
\begin{equation}\label{BCE}F(U_{a},I_{a},...)=0\end{equation}
We will see the different forms of this relation.
\subsubsection{The Kirchhoff Current Law (KCL)}
\newtheorem{kcl}{Kcl}
\begin{kcl}
At any node in an electrical circuit where charge density is not changing in time, the sum of
currents flowing towards that node is equal to the sum of currents flowing away from that node.
\end{kcl}
The KCL law gives rise to this type of equation:\\
\begin{equation}
 \sum_{i} I_{i}=0\label{eq:KCL}
\end{equation}

\begin{figure}[h]
\centerline{
 \scalebox{0.6}{
    \input{SimpleCircuit.pstex_t}
 }
}
\caption{Circuit and a graph representing its topology}
\label{fig-Circuit-example}
\end{figure}
With the example shown in figure \ref{fig-Circuit-example}, the KCL leads to the following equations:
\[-I_{1}+I_{2}+I_{3}=0\]
\[-I_{3}+I_{4}=0\]
\[I_{1}-I_{2}-I{4}=0\]
or the matrix formulation :
\[AI=0\]
where A is known as the incidence matrix and I is the vector of branch currents.
\subsubsection{The Kirchhoff Voltage Law (KVL)}
\newtheorem{kvl}{Kvl}
\begin{kvl}
The directed sum of the electrical differences around a closed circuit must be zero.
\end{kvl}
With the example shown in figure \ref{fig-Circuit-example}, the KVL leads to the following
equations:
\[U_{1}+U_{2}=0\]
\[-U_{2}+U_{3}+U_{4}=0\]
\[U_{1}+U_{3}+U_{4}=0\]
These equations contain redundancy and the system can be written:
\[\left(\begin{array}{cccc}
  1&1&0&0\\
  0&-1&1&1\\
  \end{array}\right)U=0
  \]
  where U is the vector of tensions. The matrix is known as the loop matrix B.
  \subsubsection{General equations formulation}
  The physical behavior of an electrical circuit can be describe with the following equations:
  \[AI=0\]
  \[BU=0\]
  \[\textrm{For all branches :} \qquad F(U_{a},I_{a},...)=0 \]
\subsection{Definition}
\newtheorem{mur}{Def}
\begin{mur}
The branch is current-defined if its currents is a function of its own voltage, controlling variable
or their time--derivatives:
\begin{equation}\label{CD}I_{a}=F_{i}(U_{a},U_{b},I_{c},\frac{dU_a}{dt},\frac{dU_b}{dt},\frac{dI_{c}}{dt})\end{equation}
\end{mur}
Examples : \\
A resistor is a current-defined branch because $I_{a}=\frac{U_{a}}{R}$.\\
A capacitor is a current-defined branch because $I_{a}=C\frac{dU_{a}}{dt}$.\\
\begin{mur}
The branch is voltage-defined if its voltage is a function of its own current, controlling variable
or their derivatives:
\begin{equation}\label{VD}U_{a}=F_{u}(I_{a},U_{b},I_{c},\frac{dU_a}{dt},\frac{dU_b}{dt},\frac{dI_{c}}{dt})\end{equation}
\end{mur}
Examples : \\
A resistor is a voltage-defined branch because $U_{a}=RI_{a}$.\\
A inductor is a voltage-defined branch because $U_{a}=L\frac{dI_{a}}{dt}$.\\
\subsection{Hypothesis}
The M.N.A. assumes smooth branches are explicit functions of current or voltage. It means each branch is either Voltage Defined (V.D.) or Current Defined (C.D.)\\
\subsection{Unknowns}
The M.N.A. use the following unknowns:
\begin{enumerate}
\item Nodal voltages
\item Currents in the V.D. branches
\item Capacitor's charges and currents
\item Inductor's flux and currents
\item Currents control
\end{enumerate}
These unknowns are the state variables which are assumed to be sufficient to describe the state of the circuit.
\subsection{Equations}
The M.N.A. use following equations:


Current from current-defined branch is replaced with relation~(\ref{CD}). The result is a linear relation between system's unknowns.
\subsubsection{Law in voltage-defined branches (LVD)}
It consists in replacing $U_{a}$ with $V_{i}-V_{j}$ in the relation \ref{VD} and we obtain a linear relation between system's unknowns.
\[V_{i}-V_{j}=F_{i}(I_{a},U_{b},I_{c},\frac{dU_a}{dt},\frac{dU_b}{dt},\frac{dI_{c}}{dt})\]
\subsubsection{Capacitor laws (CAP)}
In a capacitor branch, the voltage is defined by 
\begin{equation}
 q_{a}=CU_{a} 
\end{equation}

A first relation between capacitor charge and nodal tension can be written:\\
\begin{equation}
 q_{a}=C(V_{i}-V_{j})\label{eq:CAP1}\tag{CAP1}
\end{equation}

A second dynamic relation is obtained
\begin{equation}
I_{a}=\frac{dq_{a}}{dt} \label{eq:CAP2}\tag{CAP2}
\end{equation}


After a time discretisation, these equations give to linear relations between system's unknowns.
\subsubsection{Inductor laws (IND)}
A relation between inductor flux and current is defined by
\begin{equation}
\psi _{a}=LI_{a}\label{eq:IND1}\tag{IND1}
\end{equation}
and the  dynamic relation is given by
\begin{equation}
  \label{eq:IND2}
   V_{i}-V_{j}=\frac{d\psi _{a}}{dt} \tag{IND2}
\end{equation}
After a time discretization procedure, these equations give tow linear relations between system's unknowns.

\subsection{Stamps method}
The stamps method is an algorithmic method used to fill the table equation from the components. It
consists in writing a sub-table for each type of component, this sub-table is the contribution of the component in the tableau equation.\\
Following, there are stamp examples.
\subsubsection{Resistor stamp}
\[\left(\begin{array}{cccc}
&V_{i}&V_{j}&RSH\\
  \hline
  KCL(i)&\frac{-1}{R}&\frac{1}{R}&\\
  KCL(j)&\frac{1}{R}&\frac{-1}{R}&\\
  \end{array}\right)
\]
Where R is the branch's resistance.
\subsubsection{Conductance stamp}
\[\left(\begin{array}{ccccc}
&V_{i}&V_{j}&I_{a}&RSH\\
  \hline
  KCL(i)&&&1&\\
  KCL(j)&&&-1&\\
  LVD&G&-G&1&\\
  \end{array}\right)
\]
Where G is the branch's conductance.
\subsubsection{Voltage source stamp}
\[\left(\begin{array}{ccccc}
&V_{i}&V_{j}&I_{a}&RSH\\
  \hline
  KCL(i)&&&1\\
  KCL(j)&&&-1\\
  LVD&1&-1&&E
  \end{array}\right)
\]
\subsubsection{Current controlled voltage source stamp}
\[\left(\begin{array}{cccccc}
&V_{i}&V_{j}&I_{a}&I_{b}&RSH\\
  \hline
  KCL(i)&&&1&\\
  KCL(j)&&&-1&\\
  LVD&1&-1&&\gamma
  \end{array}\right)
\]
With $U_{a} = \gamma I_{b}$.

\subsection{Smooth equations and the DAE system.}
It is the same physical equations than in the MNA. This part describes the form and the unknowns
used in our formulation.

\begin{enumerate} 
 \item If a node is not connected to a capacitor then the Kirchhoff Current Law (KCL) is static. \\
  \item The V.D. equation (VDE) : \\
    A V.D. branch give an equation :
    \[V_{i}-V_{j} = \sum_{i}^{}a_{i}V_{i} + \sum_{j\in J}^{}b_{j}I_{j} +   source\]
    This equation comes from the branch constitution, not from the Kirchhoff Law. (ex : U = RI)
  \item If a node is connected to a capacitor then the Kirchhoff Current Law is dynamic (KCL).\\
    NB : The current in the capacitor branches is written with :
    \[I = C\frac{d(V_{i} - V_{j})}{dt}\]
  \item The inductor law (IL) : 
     \[V_{i} - V_{j} = L\frac{dI}{dt}\]
\end{enumerate}

These equations lead to a DAE of the form:
\[MX'+NX=s(t)\]

In general cases, the matrix $M$ is singular, it is a implicit DAE. But, with a good choice of the set of unknowns, it is
possible to write an explicit DAE of the form :
\[x' = A_{1x}x +A_{1zs}Z_{s} + A_{1ns}Z_{ns}+A_{1s}\]
\[0  = B_{1x}x+B_{1zs}Z_{s} + B_{1ns}Z_{ns}+B_{1s}\]



\subsection{General form of the MNA}
The result of the MNA is a system like :
\[MX'=AX+c\]
where X is the vector of unknowns, and M a matrix. Generally, M is not regular. Indeed, some variables
are not dynamic, for example the current in a resistor.
Moreover, If the circuit contains some non-linear components, for example non constant resistors or capacitors,
the matrices M and A will not be constant. In any case, the system can be written as an implicit DAE:
\begin{eqnarray}
F[X,X',t]=0&\label{eq-mna-dae}
\end{eqnarray}

\subsection {Discretization and resolution}

The discretization of the equation ~(\ref{eq-mna-dae}) consists in using the Differentiation Formula
or the Backward Differentiation Formula. It leads to a nonlinear algebraic equation:
\begin{eqnarray}
F[x_{n},z(x_{n},x_{n-1},x_{n-2},...),t_{n}]=0
\end{eqnarray}

This equation is linearized and solved using the Newton-Raphson iterations.




\newpage
\subsection{A simple MNA example}
\begin{figure}[h]
\centerline{
 \scalebox{0.6}{
    \documentclass[10pt]{article}


%% Symbole de fraction
\newcommand{\Frac}[2]{{\displaystyle \frac{\displaystyle #1}{\displaystyle #2}}}
\newcommand{\Prac}[2]{\displaystyle \genfrac{(}{)}{}{}{\displaystyle #1}{\displaystyle #2}}
\newcommand{\Crac}[2]{\displaystyle \genfrac{[}{]}{}{}{\displaystyle #1}{\displaystyle #2}}

\newcommand{\norme}[1]{\|#1\|}


\newcommand{\HRule}{\rule{\linewidth}{1mm}}

% Fonction math�matiques

\newcommand{\transposee}[1]{{\vphantom{#1}}^{\text{\tiny{\textsf T}}}{#1}}
\newcommand{\argmin}{\mathop{\mathrm{argmin}}}
\newcommand{\argminn}{\mathop{\mathrm{argmin}}}
\newcommand{\lexicomin}{\mathop{\mathrm{lexicomin}}}
%\newcommand{\arg}{\mathop{\mathrm{arg}}}



\DeclareMathOperator{\rot}{rot}
\DeclareMathOperator{\sh}{sh}
\DeclareMathOperator{\ch}{ch}
%\DeclareMathOperator{\th}{th}
\DeclareMathOperator{\arcsh}{arcsh}
\DeclareMathOperator{\argth}{argth}
\DeclareMathOperator{\sign}{sign}


%%The Principal Value Integral symbol
\def\Xint#1{\mathchoice
   {\XXint\displaystyle\textstyle{#1}}%
   {\XXint\textstyle\scriptstyle{#1}}%
   {\XXint\scriptstyle\scriptscriptstyle{#1}}%
   {\XXint\scriptscriptstyle\scriptscriptstyle{#1}}%
   \!\int}
\def\XXint#1#2#3{{\setbox0=\hbox{$#1{#2#3}{\int}$}
     \vcenter{\hbox{$#2#3$}}\kern-.5\wd0}}
\def\ddashint{\Xint=}
\def\dashint{\Xint-}



% macro pour les symbols d'ensemble
%\nbOne
\def\nbOne{{\mathchoice{\rm 1\mskip-4mu l}{\rm 1\mskip-4mu l} {\rm 1 \mskip-4.5mu l}{\rm 1\mskip-5mu l}}}
%
%%  Les ensembles de nombres  C. Fiorio (fiorio�at�math.tu-berlin.de) 
%
\def\nbR{\ensuremath{\mathrm{I\!R}}} % IR
\def\nbN{\ensuremath{\mathrm{I\!N}}} % IN
\def\nbF{\ensuremath{\mathrm{I\!F}}} % IF
\def\nbH{\ensuremath{\mathrm{I\!H}}} % IH
\def\nbK{\ensuremath{\mathrm{I\!K}}} % IK
\def\nbL{\ensuremath{\mathrm{I\!L}}} % IL
\def\nbM{\ensuremath{\mathrm{I\!M}}} % IM
\def\nbP{\ensuremath{\mathrm{I\!P}}} % IP
%
% \nbOne : 1I : symbol one
\def\nbOne{{\mathchoice {\rm 1\mskip-4mu l} {\rm 1\mskip-4mu l}
{\rm 1\mskip-4.5mu l} {\rm 1\mskip-5mu l}}}
%
% \nbC   :  Nombres Complexes
\def\nbC{{\mathchoice {\setbox0=\hbox{$\displaystyle\rm C$}%
\hbox{\hbox to0pt{\kern0.4\wd0\vrule height0.9\ht0\hss}\box0}}
{\setbox0=\hbox{$\textstyle\rm C$}\hbox{\hbox
to0pt{\kern0.4\wd0\vrule height0.9\ht0\hss}\box0}}
{\setbox0=\hbox{$\scriptstyle\rm C$}\hbox{\hbox
to0pt{\kern0.4\wd0\vrule height0.9\ht0\hss}\box0}}
{\setbox0=\hbox{$\scriptscriptstyle\rm C$}\hbox{\hbox
to0pt{\kern0.4\wd0\vrule height0.9\ht0\hss}\box0}}}}
%
% \nbQ   : Nombres Rationnels Q
\def\nbQ{{\mathchoice {\setbox0=\hbox{$\displaystyle\rm
Q$}\hbox{\raise
0.15\ht0\hbox to0pt{\kern0.4\wd0\vrule height0.8\ht0\hss}\box0}}
{\setbox0=\hbox{$\textstyle\rm Q$}\hbox{\raise
0.15\ht0\hbox to0pt{\kern0.4\wd0\vrule height0.8\ht0\hss}\box0}}
{\setbox0=\hbox{$\scriptstyle\rm Q$}\hbox{\raise
0.15\ht0\hbox to0pt{\kern0.4\wd0\vrule height0.7\ht0\hss}\box0}}
{\setbox0=\hbox{$\scriptscriptstyle\rm Q$}\hbox{\raise
0.15\ht0\hbox to0pt{\kern0.4\wd0\vrule height0.7\ht0\hss}\box0}}}}
%
% \nbT   : T
\def\nbT{{\mathchoice {\setbox0=\hbox{$\displaystyle\rm
T$}\hbox{\hbox to0pt{\kern0.3\wd0\vrule height0.9\ht0\hss}\box0}}
{\setbox0=\hbox{$\textstyle\rm T$}\hbox{\hbox
to0pt{\kern0.3\wd0\vrule height0.9\ht0\hss}\box0}}
{\setbox0=\hbox{$\scriptstyle\rm T$}\hbox{\hbox
to0pt{\kern0.3\wd0\vrule height0.9\ht0\hss}\box0}}
{\setbox0=\hbox{$\scriptscriptstyle\rm T$}\hbox{\hbox
to0pt{\kern0.3\wd0\vrule height0.9\ht0\hss}\box0}}}}
%
% \nbS   : S
\def\nbS{{\mathchoice
{\setbox0=\hbox{$\displaystyle     \rm S$}\hbox{\raise0.5\ht0%
\hbox to0pt{\kern0.35\wd0\vrule height0.45\ht0\hss}\hbox
to0pt{\kern0.55\wd0\vrule height0.5\ht0\hss}\box0}}
{\setbox0=\hbox{$\textstyle        \rm S$}\hbox{\raise0.5\ht0%
\hbox to0pt{\kern0.35\wd0\vrule height0.45\ht0\hss}\hbox
to0pt{\kern0.55\wd0\vrule height0.5\ht0\hss}\box0}}
{\setbox0=\hbox{$\scriptstyle      \rm S$}\hbox{\raise0.5\ht0%
\hboxto0pt{\kern0.35\wd0\vrule height0.45\ht0\hss}\raise0.05\ht0%
\hbox to0pt{\kern0.5\wd0\vrule height0.45\ht0\hss}\box0}}
{\setbox0=\hbox{$\scriptscriptstyle\rm S$}\hbox{\raise0.5\ht0%
\hboxto0pt{\kern0.4\wd0\vrule height0.45\ht0\hss}\raise0.05\ht0%
\hbox to0pt{\kern0.55\wd0\vrule height0.45\ht0\hss}\box0}}}}
%
% \nbZ   : Entiers Relatifs Z
\def\nbZ{{\mathchoice {\hbox{$\sf\textstyle Z\kern-0.4em Z$}}
{\hbox{$\sf\textstyle Z\kern-0.4em Z$}}
{\hbox{$\sf\scriptstyle Z\kern-0.3em Z$}}
{\hbox{$\sf\scriptscriptstyle Z\kern-0.2em Z$}}}}
%%%% fin macro %%%%



\newcommand{\putidx}[1]{\index{#1}\textit{#1}}
% macro pour r�f�rencer les �quations

\newcommand{\refeq}[1]{(\ref{#1})}
\newcommand{\reffig}[1]{({\it cf} figure : \ref{#1})}
\newcommand{\refann}[1]{({\it cf} Annexe : \ref{#1})}


%\definecolor{darkgray}{gray}{.25}
%\definecolor{gray}{gray}{.5}
%\definecolor{lightgray}{gray}{.75}
%\definecolor{gradbegin}{rgb}{0,1,1}
%\definecolor{gradend}{rgb}{0,.1,.95}
%\newcommand{\newtexte}[1]{\textcolor{darkgray} {#1}}
\newcommand{\newtexte}[1]{{#1}}% macro pour les varibales favorites
% normal tangent
\def\n{{\hbox{\tiny{N}}}}
\def\t{{\hbox{\tiny{T}}}}
\def\ss{{\hbox{\tiny{S}}}}
\def\nt{\hbox{\tiny{NT}}}
\def\nsf{\hbox{\tiny{\textsf N}}}
\def\tsf{\hbox{\tiny{\textsf T}}}
\def\sigman{\sigma_{\n}}
\def\sigmat{\sigma_{\t}}
\def\sigmant{\sigma_{\nt}}
\def\epsn{\epsilon_{\n}}
\def\epst{\epsilon_{\t}}
\def\epsnt{\epsilon_{\nt}}
\def\eps{\epsilon}
\def\veps{\varepsilon}
\def\sig{\sigma}
\def\Rn{R_{\n}}
\def\Rt{R_{\t}}
\def\cn{c_{\n}}
\def\Cn{C_{\n}}
\def\ct{c_{\t}}
\def\Ct{C_{\t}}
\def\un{u_{\n}}
\def\ut{\buu_{\t}}
\def\uut{u_{\t}}
\def\unc{u_{\n}^c}
\def\utc{\buu_{\t}^c}
\def\vn{v_{\n}}
\def\vt{v_{\t}}
\def\rr{\hbox{\tiny{\textsf R}}}
\def\irr{\hbox{\tiny{\textsf{IR}}}}
\def\rn{r_{\n}}
\def\rt{\brr_{\t}}
\def\rnc{r_{\n}^c}
\def\rtc{\brr_{\t}^c}
\def\trn{\Tilde{r}_{\n}}
\def\trt{\Tilde{\brr}_{\t}}
\def\tr{\Tilde{\brr}}
\def\tv{\Tilde{\bvv}}
\def\vn{v_{\n}}
\def\vt{\bvv_{\t}}
\def\adh{\mathsf{adh}}
\def\adj{\hbox{\tiny{\textsf{adj}}}}
\def\adjc{\hbox{\tiny{\textsf{adjC}}}}
\def\adja{\hbox{\tiny{\textsf{adjA}}}}
\def\cc{\hbox{\tiny{\textsf C}}}
\def\ca{\hbox{\tiny{\textsf A}}}
%%    Unit�e
\def\mm{\,\mathsf{mm}}
\def\cm{\,\mathsf{cm}}
\def\m{\,\mathsf{m}}
\def\ms{\,\mathsf{m.s^{-1}}}
\def\mms{\,\mathsf{mm.s^{-1}}}
\def\Mpa{\,\mathsf{MPa}}
\def\Gpa{\,\mathsf{GPa}}
\def\Kg{\,\mathsf{Kg}}
\def\Hz{\,\mathsf{Hz}}
\def\kHz{\,\mathsf{kHz}}
\def\N{\,\mathsf{N}}
\def\kN{\,\mathsf{kN}}
\def\Nmmm{\,\mathsf{N.m^{-3}}}
\def\ds{d_{\hbox{\tiny{S}}}}
% domaines et frontieres
\def\om{\Omega}
\def\oma{\Omega^{\alpha}}
\def\omu{\Omega^1\cup \Omega^2}
\def\gc{\Gamma_c}
\def\omt{\omu \cup \gc}
% derivee partielle et gradient et divergence
\def\p{\partial}
\def\grad{\nabla}
\def\div{\mathop{\rm div}\nolimits}
%

%\DeclareTextSymbol{\deg}{T1}{6}
%\def\degre{\mathdegree}
%\newcommand{\degre}{\mathdegree}

\def\etc{\textit{etc}\ldots}
\newcommand{\mdegre}{\hbox{\text{\degre}}}

%\def\nscd{\textsf{\bfseries NSCD}}
%\def\nscd{\textsf{NSCD}}
\newcommand{\nscd}{\textsf{NSCD}}
%\Pisymbol{psy}{212} ou encore \Pisymbol{psy}{228}




%----------------------------------------------------------------------
%             Des chiffres avec des ronds autour
%----------------------------------------------------------------------
\def\nombrecercle#1{\def\taille{0.3}
                \put(0,0){#1}
                \put(0.08,0.08){\circle{\taille}}}



\def\ae#1{\stackrel{\mbox{\scriptsize a.e.}}{#1}}
\def\argmin{\mathop{\rm argmin}}
\def\eqref#1{{\rm (\ref{#1})\/}}
\def\indicfon{\mathord{\rm i}}       %indicator function
\def\p{\mathord{\rm proj}}
\def\N{\mathord{\rm N}}
% \def\prosca#1#2{#1\cdot#2}
\def\prosca#1#2{\langle #1,#2\rangle}
\def\qedtext{\mbox{}\hfill$\Box$}
\def\qedmath{\eqno\Box}

\def\s{{$\mathcal{S}$}}
\def\somme{\mathop{\textstyle\sum}}
\def\somme{\mathop{\textstyle\sum}}
\def\submoins{_{\scriptscriptstyle-}}
\def\subplus{_{\scriptscriptstyle+}}
\def\T{\mathord{\rm T}}

%----------------------------------------------------------------------
%             Macro M Jean 
%----------------------------------------------------------------------

\def\Real{\mbox{I\hspace{-.15em}R}}
\def\Integer{\mbox{I\hspace{-.15em}N}}
\def\Bunit{\mbox{I\hspace{-.15em}B}}
\def\real{\mbox{\scriptsize I\hspace{-.15em}R}}
\def\bunit{\mbox{\scriptsize I\hspace{-.15em}B}}
\def\IL{\mbox{\scriptsize I\hspace{-.15em}L}}
\def\Indic{\mbox{\large $\psi$}}
\def\bfxi{\mbox{$\xi$ \hspace{-1.1em} $\xi$}}
%\def\bfXi{\mbox{$\Xi$ \hspace{-1.1em} $\Xi$}}
\def\RunR{\mathcal R}
\def\RunRN{\mathcal R_{N}}
\def\RunRT{\mathcal R_{T}}
\def\RunS{\mathcal S}
\def\RunSN{\mathcal S_{N}}
\def\RunST{\mathcal S_{T}}
\def\RunU{\mathcal U}
\def\RunUN{\mathcal U_{N}}
\def\RunUT{\mathcal U_{T}}
\def\RunUP{\mathcal U'}
\def\RunUPN{\mathcal U'_{N}}
\def\RunUPT{\mathcal U'_{T}}
\def\RunJ{\mathcal J}
\def\RunW{\mathcal W}
\def\RunF{f}
\def\RunFa{f_{1}}
\def\RunFb{f_{2}}
\def\RunFP{f'}
\def\RunV{v}
\def\RunVP{v'}
\def\EspF{\mathcal F}
\def\EspV{\mathcal V}
%%%%
\catcode`\�=13
\def�{\'e}
\catcode`\�=13
\def�{\`e}
\catcode`\�=13
\def�{\`a}
\catcode`\�=13
\def�{\c c}
\def\N{\mbox{I\hspace{ -.15em}N}}
\def\Z{\mbox{Z\hspace{ -.3em}Z}}
\def\Q{\mbox{l\hspace{ -.47em}Q}}
\def\R{\mbox{l\hspace{ -.15em}R}}
\def\F{\mbox{l\hspace{ -.15em}F}}
\def\E{\mbox{l\hspace{ -.15em}E}}
\def\LMGC90{{\small \it LMGC90 }}
\def\NSCD{{\small \it NSCD }}
\def\CHIC{{\small \it CHIC }}
\def\half{{\frac{_{1}}{^{2}}}}
\def\12T{{\frac{_{1}}{^{2T}}}}

\def\geq{\geqslant}
\def\leq{\leqslant}
\def\ge{\geqslant}
\def\le{\leqslant}


\begingroup
\count0=\time \divide\count0by60 % Hour
\count2=\count0 \multiply\count2by-60 \advance\count2by\time
% Min
\def\2#1{\ifnum#1<10 0\fi\the#1}
\xdef\isodayandtime{\the\year-\2\month-\2\day\space\2{\count0}:%
\2{\count2}}
\endgroup

%---------------------------------------------------------------------
%             Redaction note environnement B. Brogliato
%----------------------------------------------------------------------
\makeatletter

{\newtheorem{ndr1bb}{\textbf{\textsc{Redaction note B.B.}}}[section]}

\newenvironment{ndrbb}%
{%
\noindent\begin{ndr1bb}\hrule\vspace{1em}%
\ttfamily\small
}%
{%
\begin{flushright}%
%\vspace{-1.5em}\ding{111}
\end{flushright}%
\vspace{-1.5em}\hrule
\end{ndr1bb}%
}

%---------------------------------------------------------------------
%             Redaction note environnement O. Bonnefon
%----------------------------------------------------------------------

{\newtheorem{ndr1ob}{\textbf{\textsc{Redaction note O.B.}}}[section]}

\newenvironment{ndrob}%
{%
\noindent\begin{ndr1ob}\hrule\vspace{1em}%
\ttfamily\small
}%
{%
\begin{flushright}%
%\vspace{-1.5em}\ding{111}
\end{flushright}%
\vspace{-1.5em}\hrule
\end{ndr1ob}%
}

%----------------------------------------------------------------------
%             Redaction note environnement V.ACARY
%----------------------------------------------------------------------
% Faut etre fou pour s'amuser a pondre des notes pareilles

{\newtheorem{ndr1va}{\textbf{\textsc{Redaction note V.A.}}}[section]}

\newenvironment{ndrva}%
{%
\noindent\begin{ndr1va}\hrule\vspace{1em}%
\ttfamily\small \  \\
\indent}%
{%
\begin{flushright}%
\  \\
%\vspace{-1.5em}\ding{111}
\end{flushright}%
\vspace{-1.5em}\hrule
\end{ndr1va}%
}
\makeatother






% ----------------DEFINITIONS-----------------
% 

 \def\II{\mathop{{\rm I}\mskip-3.0mu{\rm I}}\nolimits}




% -----------------------------------
 \def\c{\mathop{{\rm 1}\mskip-10.0mu{\rm C}}\nolimits}
 \def\C{\mathop{{\rm 1}\mskip-10.0mu{\rm C}}\nolimits}
 \def\ZZ{\mathaccent23Z}
% 
 \def\abstract{
 \footnotesize\quotation \noindent {\bf Abstract.}}
% 

\newcommand{\ie}{{\textit{i.e.}}}


%\def\sgn{\mbox{\rm sgn}}
\DeclareMathOperator{\sgn}{sgn}
\DeclareMathOperator{\proj}{proj}

\newcommand{\RR}{\mbox{\rm $I\!\!R$}}
\newcommand{\NN}{\mbox{\rm $I\!\!N$}}



% ---------------- MMC -----------------
% 

\newcommand{\contract}{{\,:\,}}

\newcommand{\scontract}{{\,{\Bar\otimes}\,}}
\newcommand{\tcontract}{{\,{\Bar{\Bar{\Bar\otimes}}}\,}}


\newcommand{\DP}[2]{\displaystyle \frac{\partial {#1}}{\partial {#2}}}


\usepackage{pifont}
\makeatletter
\def\cqfd{\ifmmode\sqw\else{\ifhmode\unskip\fi\nobreak\hfil
\penalty50\hskip1em\null\nobreak\hfil\ding{111}
\parfillskip=0pt\finalhyphendemerits=0\endgraf}\fi}
\makeatother
\def\off{{\hbox{\tiny{\textsf{off}}}}}
\def\on{{\hbox{\tiny{\textsf{on}}}}}
\def\pwl{{\hbox{\tiny{\textsf{pwl}}}}}

%%% Local Variables: 
%%% mode: latex
%%% TeX-master: "book"
%%% x-symbol-coding: iso-8859-2
%%% End: 

\usepackage{psfrag}
\usepackage{fancyhdr}
\usepackage{subfigure}
%\renewcommand{\baselinestretch}{1.2}
\textheight 23cm
\textwidth 16cm
\topmargin 0cm
%\evensidemargin 0cm
\oddsidemargin 0cm
\evensidemargin 0cm
\usepackage{layout}
\usepackage{mathpple}
\makeatletter
\renewcommand\bibsection{\paragraph{References
     \@mkboth{\MakeUppercase{\bibname}}{\MakeUppercase{\bibname}}}}
\makeatother
%% style des entetes et des pieds de page
\fancyhf{} % nettoie le entetes et les pieds
\fancyhead[L]{Template 6 : Electrical oscillator with 4 diodes bridge full-wave rectifier - Pascal Denoyelle}
%\fancyhead[C]{}%
\fancyhead[R]{\thepage}
%\fancyfoot[L]{\resizebox{!}{0.7cm}{\includegraphics[clip]{logoesm2.eps}}}%
\fancyfoot[C]{}%
 \begin{document}
 
\section{Introduction\\}
This document is a short overview about the Modified Nodal Analysis. The M.N.A. is the method used
in SPICE to obtain the circuit equation formulation. For more details, read the book <<Circuit Simulation
Methods and Algorithms by Jan Ogrodzki>>.

\subsection{Notations}
\begin{enumerate}
  \item U is a tension, I is a current.
  \item V denotes a node's potential.
  \item q denotes a capacitor's charge.
  \item $\psi$ denotes a inductor's flux.
  \item Indice $_{a}$ denotes the current branch.
  \item Indice $_{b}$ denotes the other branch whose voltage is a controlling variable.
  \item Indice $_{c}$ denotes the other branch whose current is a controlling variable.
  \end{enumerate}

\newtheorem{mur}{Def}
\begin{mur}
The branch is current-defined if its currents is a function of its own voltage, controlling variable
or their derivatives:
\begin{equation}\label{CD}I_{a}=F_{i}(U_{a},U_{b},I_{c},\frac{dU_a}{dt},\frac{dU_b}{dt},\frac{dI_{c}}{dt})\end{equation}
\end{mur}
Examples : \\
A resistor is a current-defined branch because $I_{a}=\frac{U_{a}}{R}$.\\
A capacitor is a current-defined branch because $I_{a}=C\frac{dU_{a}}{dt}$.\\
\begin{mur}
The branch is voltage-defined if its voltage is a function of its own current, controlling variable
or their derivatives:
\begin{equation}\label{VD}U_{a}=F_{i}(I_{a},U_{b},I_{c},\frac{dU_a}{dt},\frac{dU_b}{dt},\frac{dI_{c}}{dt})\end{equation}
\end{mur}
Examples : \\
A resistor is a voltage-defined branch because $U_{a}=RI_{a}$.\\
A inductor is a voltage-defined branch because $U_{a}=L\frac{dI_{a}}{dt}$.\\
\section{Hypothesis\\}
The M.N.A. assumes smooth branches are explicit functions of current or voltage. It means each smooth
branch is Voltage Defined (V.D.) or Current Defined (C.D.)\\
\section{Unknowns}
The M.N.A. use the following unknowns:
\begin{enumerate}
\item Nodal voltages\\
\item Currents in the V.D. branches\\
\item Capacitor's charges and currents
\item Inductor's flux and currents
\item Currents control
\end{enumerate}
These unknowns are sufficient to describe the circuit.
\section{Equations}
The M.N.A. use following equations:
\subsection{The Kirchhoff Current Law (KCL)}
\newtheorem{kcl}{Kcl}
\begin{kcl}
At any node in an electrical circuit where charge density is not changing in time, the sum of
currents flowing towards that node is equal to the sum of currents flowing away from that node.
\end{kcl}
KCL law gives this type of equation:\\
$I_{1}+I_{2}+...+I_{n}=0$\\
Current from current-defined branch is replaced with relation \ref{CD}. The result is a linear relation between system's unknowns.
\subsection{Law in voltage-defined branches (LVD)}
It consists in replacing $U_{a}$ with $V_{i}-V_{j}$ in the relation \ref{VD} and we obtain a linear relation between system's unknowns.
\[V_{i}-V_{j}=F_{i}(I_{a},U_{b},I_{c},\frac{dU_a}{dt},\frac{dU_b}{dt},\frac{dI_{c}}{dt})\]
\subsection{Capacitor laws (CAP)}
A relation between capacitor charge and tension (CAP1):\\
\[ q_{a}=CU_{a} \]
Voltage definition (CAP2):
\[ U_{a}=V_{i}-V_{j} \]
A dynamic relation (CAP3):
\[ I_{a}=\frac{dq_{a}}{dt} \]

After a time discretisation, these equations give tow linear relations between system's unknowns.
\subsection{Inductor laws (IND)}
A relation between inductor flux and current (IND1):\\
\[ \psi _{a}=LI_{a} \]
A dynamic relation (IND2):
\[ V_{i}-V_{j}=\frac{d\psi _{a}}{dt} \]
After a time discretisation, these equations give tow linear relations between system's unknowns.

\section{Example}
\begin{figure}[h]
\centerline{
 \scalebox{0.5}{
    \documentclass[10pt]{article}


%% Symbole de fraction
\newcommand{\Frac}[2]{{\displaystyle \frac{\displaystyle #1}{\displaystyle #2}}}
\newcommand{\Prac}[2]{\displaystyle \genfrac{(}{)}{}{}{\displaystyle #1}{\displaystyle #2}}
\newcommand{\Crac}[2]{\displaystyle \genfrac{[}{]}{}{}{\displaystyle #1}{\displaystyle #2}}

\newcommand{\norme}[1]{\|#1\|}


\newcommand{\HRule}{\rule{\linewidth}{1mm}}

% Fonction math�matiques

\newcommand{\transposee}[1]{{\vphantom{#1}}^{\text{\tiny{\textsf T}}}{#1}}
\newcommand{\argmin}{\mathop{\mathrm{argmin}}}
\newcommand{\argminn}{\mathop{\mathrm{argmin}}}
\newcommand{\lexicomin}{\mathop{\mathrm{lexicomin}}}
%\newcommand{\arg}{\mathop{\mathrm{arg}}}



\DeclareMathOperator{\rot}{rot}
\DeclareMathOperator{\sh}{sh}
\DeclareMathOperator{\ch}{ch}
%\DeclareMathOperator{\th}{th}
\DeclareMathOperator{\arcsh}{arcsh}
\DeclareMathOperator{\argth}{argth}
\DeclareMathOperator{\sign}{sign}


%%The Principal Value Integral symbol
\def\Xint#1{\mathchoice
   {\XXint\displaystyle\textstyle{#1}}%
   {\XXint\textstyle\scriptstyle{#1}}%
   {\XXint\scriptstyle\scriptscriptstyle{#1}}%
   {\XXint\scriptscriptstyle\scriptscriptstyle{#1}}%
   \!\int}
\def\XXint#1#2#3{{\setbox0=\hbox{$#1{#2#3}{\int}$}
     \vcenter{\hbox{$#2#3$}}\kern-.5\wd0}}
\def\ddashint{\Xint=}
\def\dashint{\Xint-}



% macro pour les symbols d'ensemble
%\nbOne
\def\nbOne{{\mathchoice{\rm 1\mskip-4mu l}{\rm 1\mskip-4mu l} {\rm 1 \mskip-4.5mu l}{\rm 1\mskip-5mu l}}}
%
%%  Les ensembles de nombres  C. Fiorio (fiorio�at�math.tu-berlin.de) 
%
\def\nbR{\ensuremath{\mathrm{I\!R}}} % IR
\def\nbN{\ensuremath{\mathrm{I\!N}}} % IN
\def\nbF{\ensuremath{\mathrm{I\!F}}} % IF
\def\nbH{\ensuremath{\mathrm{I\!H}}} % IH
\def\nbK{\ensuremath{\mathrm{I\!K}}} % IK
\def\nbL{\ensuremath{\mathrm{I\!L}}} % IL
\def\nbM{\ensuremath{\mathrm{I\!M}}} % IM
\def\nbP{\ensuremath{\mathrm{I\!P}}} % IP
%
% \nbOne : 1I : symbol one
\def\nbOne{{\mathchoice {\rm 1\mskip-4mu l} {\rm 1\mskip-4mu l}
{\rm 1\mskip-4.5mu l} {\rm 1\mskip-5mu l}}}
%
% \nbC   :  Nombres Complexes
\def\nbC{{\mathchoice {\setbox0=\hbox{$\displaystyle\rm C$}%
\hbox{\hbox to0pt{\kern0.4\wd0\vrule height0.9\ht0\hss}\box0}}
{\setbox0=\hbox{$\textstyle\rm C$}\hbox{\hbox
to0pt{\kern0.4\wd0\vrule height0.9\ht0\hss}\box0}}
{\setbox0=\hbox{$\scriptstyle\rm C$}\hbox{\hbox
to0pt{\kern0.4\wd0\vrule height0.9\ht0\hss}\box0}}
{\setbox0=\hbox{$\scriptscriptstyle\rm C$}\hbox{\hbox
to0pt{\kern0.4\wd0\vrule height0.9\ht0\hss}\box0}}}}
%
% \nbQ   : Nombres Rationnels Q
\def\nbQ{{\mathchoice {\setbox0=\hbox{$\displaystyle\rm
Q$}\hbox{\raise
0.15\ht0\hbox to0pt{\kern0.4\wd0\vrule height0.8\ht0\hss}\box0}}
{\setbox0=\hbox{$\textstyle\rm Q$}\hbox{\raise
0.15\ht0\hbox to0pt{\kern0.4\wd0\vrule height0.8\ht0\hss}\box0}}
{\setbox0=\hbox{$\scriptstyle\rm Q$}\hbox{\raise
0.15\ht0\hbox to0pt{\kern0.4\wd0\vrule height0.7\ht0\hss}\box0}}
{\setbox0=\hbox{$\scriptscriptstyle\rm Q$}\hbox{\raise
0.15\ht0\hbox to0pt{\kern0.4\wd0\vrule height0.7\ht0\hss}\box0}}}}
%
% \nbT   : T
\def\nbT{{\mathchoice {\setbox0=\hbox{$\displaystyle\rm
T$}\hbox{\hbox to0pt{\kern0.3\wd0\vrule height0.9\ht0\hss}\box0}}
{\setbox0=\hbox{$\textstyle\rm T$}\hbox{\hbox
to0pt{\kern0.3\wd0\vrule height0.9\ht0\hss}\box0}}
{\setbox0=\hbox{$\scriptstyle\rm T$}\hbox{\hbox
to0pt{\kern0.3\wd0\vrule height0.9\ht0\hss}\box0}}
{\setbox0=\hbox{$\scriptscriptstyle\rm T$}\hbox{\hbox
to0pt{\kern0.3\wd0\vrule height0.9\ht0\hss}\box0}}}}
%
% \nbS   : S
\def\nbS{{\mathchoice
{\setbox0=\hbox{$\displaystyle     \rm S$}\hbox{\raise0.5\ht0%
\hbox to0pt{\kern0.35\wd0\vrule height0.45\ht0\hss}\hbox
to0pt{\kern0.55\wd0\vrule height0.5\ht0\hss}\box0}}
{\setbox0=\hbox{$\textstyle        \rm S$}\hbox{\raise0.5\ht0%
\hbox to0pt{\kern0.35\wd0\vrule height0.45\ht0\hss}\hbox
to0pt{\kern0.55\wd0\vrule height0.5\ht0\hss}\box0}}
{\setbox0=\hbox{$\scriptstyle      \rm S$}\hbox{\raise0.5\ht0%
\hboxto0pt{\kern0.35\wd0\vrule height0.45\ht0\hss}\raise0.05\ht0%
\hbox to0pt{\kern0.5\wd0\vrule height0.45\ht0\hss}\box0}}
{\setbox0=\hbox{$\scriptscriptstyle\rm S$}\hbox{\raise0.5\ht0%
\hboxto0pt{\kern0.4\wd0\vrule height0.45\ht0\hss}\raise0.05\ht0%
\hbox to0pt{\kern0.55\wd0\vrule height0.45\ht0\hss}\box0}}}}
%
% \nbZ   : Entiers Relatifs Z
\def\nbZ{{\mathchoice {\hbox{$\sf\textstyle Z\kern-0.4em Z$}}
{\hbox{$\sf\textstyle Z\kern-0.4em Z$}}
{\hbox{$\sf\scriptstyle Z\kern-0.3em Z$}}
{\hbox{$\sf\scriptscriptstyle Z\kern-0.2em Z$}}}}
%%%% fin macro %%%%



\newcommand{\putidx}[1]{\index{#1}\textit{#1}}
% macro pour r�f�rencer les �quations

\newcommand{\refeq}[1]{(\ref{#1})}
\newcommand{\reffig}[1]{({\it cf} figure : \ref{#1})}
\newcommand{\refann}[1]{({\it cf} Annexe : \ref{#1})}


%\definecolor{darkgray}{gray}{.25}
%\definecolor{gray}{gray}{.5}
%\definecolor{lightgray}{gray}{.75}
%\definecolor{gradbegin}{rgb}{0,1,1}
%\definecolor{gradend}{rgb}{0,.1,.95}
%\newcommand{\newtexte}[1]{\textcolor{darkgray} {#1}}
\newcommand{\newtexte}[1]{{#1}}% macro pour les varibales favorites
% normal tangent
\def\n{{\hbox{\tiny{N}}}}
\def\t{{\hbox{\tiny{T}}}}
\def\ss{{\hbox{\tiny{S}}}}
\def\nt{\hbox{\tiny{NT}}}
\def\nsf{\hbox{\tiny{\textsf N}}}
\def\tsf{\hbox{\tiny{\textsf T}}}
\def\sigman{\sigma_{\n}}
\def\sigmat{\sigma_{\t}}
\def\sigmant{\sigma_{\nt}}
\def\epsn{\epsilon_{\n}}
\def\epst{\epsilon_{\t}}
\def\epsnt{\epsilon_{\nt}}
\def\eps{\epsilon}
\def\veps{\varepsilon}
\def\sig{\sigma}
\def\Rn{R_{\n}}
\def\Rt{R_{\t}}
\def\cn{c_{\n}}
\def\Cn{C_{\n}}
\def\ct{c_{\t}}
\def\Ct{C_{\t}}
\def\un{u_{\n}}
\def\ut{\buu_{\t}}
\def\uut{u_{\t}}
\def\unc{u_{\n}^c}
\def\utc{\buu_{\t}^c}
\def\vn{v_{\n}}
\def\vt{v_{\t}}
\def\rr{\hbox{\tiny{\textsf R}}}
\def\irr{\hbox{\tiny{\textsf{IR}}}}
\def\rn{r_{\n}}
\def\rt{\brr_{\t}}
\def\rnc{r_{\n}^c}
\def\rtc{\brr_{\t}^c}
\def\trn{\Tilde{r}_{\n}}
\def\trt{\Tilde{\brr}_{\t}}
\def\tr{\Tilde{\brr}}
\def\tv{\Tilde{\bvv}}
\def\vn{v_{\n}}
\def\vt{\bvv_{\t}}
\def\adh{\mathsf{adh}}
\def\adj{\hbox{\tiny{\textsf{adj}}}}
\def\adjc{\hbox{\tiny{\textsf{adjC}}}}
\def\adja{\hbox{\tiny{\textsf{adjA}}}}
\def\cc{\hbox{\tiny{\textsf C}}}
\def\ca{\hbox{\tiny{\textsf A}}}
%%    Unit�e
\def\mm{\,\mathsf{mm}}
\def\cm{\,\mathsf{cm}}
\def\m{\,\mathsf{m}}
\def\ms{\,\mathsf{m.s^{-1}}}
\def\mms{\,\mathsf{mm.s^{-1}}}
\def\Mpa{\,\mathsf{MPa}}
\def\Gpa{\,\mathsf{GPa}}
\def\Kg{\,\mathsf{Kg}}
\def\Hz{\,\mathsf{Hz}}
\def\kHz{\,\mathsf{kHz}}
\def\N{\,\mathsf{N}}
\def\kN{\,\mathsf{kN}}
\def\Nmmm{\,\mathsf{N.m^{-3}}}
\def\ds{d_{\hbox{\tiny{S}}}}
% domaines et frontieres
\def\om{\Omega}
\def\oma{\Omega^{\alpha}}
\def\omu{\Omega^1\cup \Omega^2}
\def\gc{\Gamma_c}
\def\omt{\omu \cup \gc}
% derivee partielle et gradient et divergence
\def\p{\partial}
\def\grad{\nabla}
\def\div{\mathop{\rm div}\nolimits}
%

%\DeclareTextSymbol{\deg}{T1}{6}
%\def\degre{\mathdegree}
%\newcommand{\degre}{\mathdegree}

\def\etc{\textit{etc}\ldots}
\newcommand{\mdegre}{\hbox{\text{\degre}}}

%\def\nscd{\textsf{\bfseries NSCD}}
%\def\nscd{\textsf{NSCD}}
\newcommand{\nscd}{\textsf{NSCD}}
%\Pisymbol{psy}{212} ou encore \Pisymbol{psy}{228}




%----------------------------------------------------------------------
%             Des chiffres avec des ronds autour
%----------------------------------------------------------------------
\def\nombrecercle#1{\def\taille{0.3}
                \put(0,0){#1}
                \put(0.08,0.08){\circle{\taille}}}



\def\ae#1{\stackrel{\mbox{\scriptsize a.e.}}{#1}}
\def\argmin{\mathop{\rm argmin}}
\def\eqref#1{{\rm (\ref{#1})\/}}
\def\indicfon{\mathord{\rm i}}       %indicator function
\def\p{\mathord{\rm proj}}
\def\N{\mathord{\rm N}}
% \def\prosca#1#2{#1\cdot#2}
\def\prosca#1#2{\langle #1,#2\rangle}
\def\qedtext{\mbox{}\hfill$\Box$}
\def\qedmath{\eqno\Box}

\def\s{{$\mathcal{S}$}}
\def\somme{\mathop{\textstyle\sum}}
\def\somme{\mathop{\textstyle\sum}}
\def\submoins{_{\scriptscriptstyle-}}
\def\subplus{_{\scriptscriptstyle+}}
\def\T{\mathord{\rm T}}

%----------------------------------------------------------------------
%             Macro M Jean 
%----------------------------------------------------------------------

\def\Real{\mbox{I\hspace{-.15em}R}}
\def\Integer{\mbox{I\hspace{-.15em}N}}
\def\Bunit{\mbox{I\hspace{-.15em}B}}
\def\real{\mbox{\scriptsize I\hspace{-.15em}R}}
\def\bunit{\mbox{\scriptsize I\hspace{-.15em}B}}
\def\IL{\mbox{\scriptsize I\hspace{-.15em}L}}
\def\Indic{\mbox{\large $\psi$}}
\def\bfxi{\mbox{$\xi$ \hspace{-1.1em} $\xi$}}
%\def\bfXi{\mbox{$\Xi$ \hspace{-1.1em} $\Xi$}}
\def\RunR{\mathcal R}
\def\RunRN{\mathcal R_{N}}
\def\RunRT{\mathcal R_{T}}
\def\RunS{\mathcal S}
\def\RunSN{\mathcal S_{N}}
\def\RunST{\mathcal S_{T}}
\def\RunU{\mathcal U}
\def\RunUN{\mathcal U_{N}}
\def\RunUT{\mathcal U_{T}}
\def\RunUP{\mathcal U'}
\def\RunUPN{\mathcal U'_{N}}
\def\RunUPT{\mathcal U'_{T}}
\def\RunJ{\mathcal J}
\def\RunW{\mathcal W}
\def\RunF{f}
\def\RunFa{f_{1}}
\def\RunFb{f_{2}}
\def\RunFP{f'}
\def\RunV{v}
\def\RunVP{v'}
\def\EspF{\mathcal F}
\def\EspV{\mathcal V}
%%%%
\catcode`\�=13
\def�{\'e}
\catcode`\�=13
\def�{\`e}
\catcode`\�=13
\def�{\`a}
\catcode`\�=13
\def�{\c c}
\def\N{\mbox{I\hspace{ -.15em}N}}
\def\Z{\mbox{Z\hspace{ -.3em}Z}}
\def\Q{\mbox{l\hspace{ -.47em}Q}}
\def\R{\mbox{l\hspace{ -.15em}R}}
\def\F{\mbox{l\hspace{ -.15em}F}}
\def\E{\mbox{l\hspace{ -.15em}E}}
\def\LMGC90{{\small \it LMGC90 }}
\def\NSCD{{\small \it NSCD }}
\def\CHIC{{\small \it CHIC }}
\def\half{{\frac{_{1}}{^{2}}}}
\def\12T{{\frac{_{1}}{^{2T}}}}

\def\geq{\geqslant}
\def\leq{\leqslant}
\def\ge{\geqslant}
\def\le{\leqslant}


\begingroup
\count0=\time \divide\count0by60 % Hour
\count2=\count0 \multiply\count2by-60 \advance\count2by\time
% Min
\def\2#1{\ifnum#1<10 0\fi\the#1}
\xdef\isodayandtime{\the\year-\2\month-\2\day\space\2{\count0}:%
\2{\count2}}
\endgroup

%---------------------------------------------------------------------
%             Redaction note environnement B. Brogliato
%----------------------------------------------------------------------
\makeatletter

{\newtheorem{ndr1bb}{\textbf{\textsc{Redaction note B.B.}}}[section]}

\newenvironment{ndrbb}%
{%
\noindent\begin{ndr1bb}\hrule\vspace{1em}%
\ttfamily\small
}%
{%
\begin{flushright}%
%\vspace{-1.5em}\ding{111}
\end{flushright}%
\vspace{-1.5em}\hrule
\end{ndr1bb}%
}

%---------------------------------------------------------------------
%             Redaction note environnement O. Bonnefon
%----------------------------------------------------------------------

{\newtheorem{ndr1ob}{\textbf{\textsc{Redaction note O.B.}}}[section]}

\newenvironment{ndrob}%
{%
\noindent\begin{ndr1ob}\hrule\vspace{1em}%
\ttfamily\small
}%
{%
\begin{flushright}%
%\vspace{-1.5em}\ding{111}
\end{flushright}%
\vspace{-1.5em}\hrule
\end{ndr1ob}%
}

%----------------------------------------------------------------------
%             Redaction note environnement V.ACARY
%----------------------------------------------------------------------
% Faut etre fou pour s'amuser a pondre des notes pareilles

{\newtheorem{ndr1va}{\textbf{\textsc{Redaction note V.A.}}}[section]}

\newenvironment{ndrva}%
{%
\noindent\begin{ndr1va}\hrule\vspace{1em}%
\ttfamily\small \  \\
\indent}%
{%
\begin{flushright}%
\  \\
%\vspace{-1.5em}\ding{111}
\end{flushright}%
\vspace{-1.5em}\hrule
\end{ndr1va}%
}
\makeatother






% ----------------DEFINITIONS-----------------
% 

 \def\II{\mathop{{\rm I}\mskip-3.0mu{\rm I}}\nolimits}




% -----------------------------------
 \def\c{\mathop{{\rm 1}\mskip-10.0mu{\rm C}}\nolimits}
 \def\C{\mathop{{\rm 1}\mskip-10.0mu{\rm C}}\nolimits}
 \def\ZZ{\mathaccent23Z}
% 
 \def\abstract{
 \footnotesize\quotation \noindent {\bf Abstract.}}
% 

\newcommand{\ie}{{\textit{i.e.}}}


%\def\sgn{\mbox{\rm sgn}}
\DeclareMathOperator{\sgn}{sgn}
\DeclareMathOperator{\proj}{proj}

\newcommand{\RR}{\mbox{\rm $I\!\!R$}}
\newcommand{\NN}{\mbox{\rm $I\!\!N$}}



% ---------------- MMC -----------------
% 

\newcommand{\contract}{{\,:\,}}

\newcommand{\scontract}{{\,{\Bar\otimes}\,}}
\newcommand{\tcontract}{{\,{\Bar{\Bar{\Bar\otimes}}}\,}}


\newcommand{\DP}[2]{\displaystyle \frac{\partial {#1}}{\partial {#2}}}


\usepackage{pifont}
\makeatletter
\def\cqfd{\ifmmode\sqw\else{\ifhmode\unskip\fi\nobreak\hfil
\penalty50\hskip1em\null\nobreak\hfil\ding{111}
\parfillskip=0pt\finalhyphendemerits=0\endgraf}\fi}
\makeatother
\def\off{{\hbox{\tiny{\textsf{off}}}}}
\def\on{{\hbox{\tiny{\textsf{on}}}}}
\def\pwl{{\hbox{\tiny{\textsf{pwl}}}}}

%%% Local Variables: 
%%% mode: latex
%%% TeX-master: "book"
%%% x-symbol-coding: iso-8859-2
%%% End: 

\usepackage{psfrag}
\usepackage{fancyhdr}
\usepackage{subfigure}
%\renewcommand{\baselinestretch}{1.2}
\textheight 23cm
\textwidth 16cm
\topmargin 0cm
%\evensidemargin 0cm
\oddsidemargin 0cm
\evensidemargin 0cm
\usepackage{layout}
\usepackage{mathpple}
\makeatletter
\renewcommand\bibsection{\paragraph{References
     \@mkboth{\MakeUppercase{\bibname}}{\MakeUppercase{\bibname}}}}
\makeatother
%% style des entetes et des pieds de page
\fancyhf{} % nettoie le entetes et les pieds
\fancyhead[L]{Template 6 : Electrical oscillator with 4 diodes bridge full-wave rectifier - Pascal Denoyelle}
%\fancyhead[C]{}%
\fancyhead[R]{\thepage}
%\fancyfoot[L]{\resizebox{!}{0.7cm}{\includegraphics[clip]{logoesm2.eps}}}%
\fancyfoot[C]{}%
 \begin{document}
 
\section{Introduction\\}
This document is a short overview about the Modified Nodal Analysis. The M.N.A. is the method used
in SPICE to obtain the circuit equation formulation. For more details, read the book <<Circuit Simulation
Methods and Algorithms by Jan Ogrodzki>>.

\subsection{Notations}
\begin{enumerate}
  \item U is a tension, I is a current.
  \item V denotes a node's potential.
  \item q denotes a capacitor's charge.
  \item $\psi$ denotes a inductor's flux.
  \item Indice $_{a}$ denotes the current branch.
  \item Indice $_{b}$ denotes the other branch whose voltage is a controlling variable.
  \item Indice $_{c}$ denotes the other branch whose current is a controlling variable.
  \end{enumerate}

\newtheorem{mur}{Def}
\begin{mur}
The branch is current-defined if its currents is a function of its own voltage, controlling variable
or their derivatives:
\begin{equation}\label{CD}I_{a}=F_{i}(U_{a},U_{b},I_{c},\frac{dU_a}{dt},\frac{dU_b}{dt},\frac{dI_{c}}{dt})\end{equation}
\end{mur}
Examples : \\
A resistor is a current-defined branch because $I_{a}=\frac{U_{a}}{R}$.\\
A capacitor is a current-defined branch because $I_{a}=C\frac{dU_{a}}{dt}$.\\
\begin{mur}
The branch is voltage-defined if its voltage is a function of its own current, controlling variable
or their derivatives:
\begin{equation}\label{VD}U_{a}=F_{i}(I_{a},U_{b},I_{c},\frac{dU_a}{dt},\frac{dU_b}{dt},\frac{dI_{c}}{dt})\end{equation}
\end{mur}
Examples : \\
A resistor is a voltage-defined branch because $U_{a}=RI_{a}$.\\
A inductor is a voltage-defined branch because $U_{a}=L\frac{dI_{a}}{dt}$.\\
\section{Hypothesis\\}
The M.N.A. assumes smooth branches are explicit functions of current or voltage. It means each smooth
branch is Voltage Defined (V.D.) or Current Defined (C.D.)\\
\section{Unknowns}
The M.N.A. use the following unknowns:
\begin{enumerate}
\item Nodal voltages\\
\item Currents in the V.D. branches\\
\item Capacitor's charges and currents
\item Inductor's flux and currents
\item Currents control
\end{enumerate}
These unknowns are sufficient to describe the circuit.
\section{Equations}
The M.N.A. use following equations:
\subsection{The Kirchhoff Current Law (KCL)}
\newtheorem{kcl}{Kcl}
\begin{kcl}
At any node in an electrical circuit where charge density is not changing in time, the sum of
currents flowing towards that node is equal to the sum of currents flowing away from that node.
\end{kcl}
KCL law gives this type of equation:\\
$I_{1}+I_{2}+...+I_{n}=0$\\
Current from current-defined branch is replaced with relation \ref{CD}. The result is a linear relation between system's unknowns.
\subsection{Law in voltage-defined branches (LVD)}
It consists in replacing $U_{a}$ with $V_{i}-V_{j}$ in the relation \ref{VD} and we obtain a linear relation between system's unknowns.
\[V_{i}-V_{j}=F_{i}(I_{a},U_{b},I_{c},\frac{dU_a}{dt},\frac{dU_b}{dt},\frac{dI_{c}}{dt})\]
\subsection{Capacitor laws (CAP)}
A relation between capacitor charge and tension (CAP1):\\
\[ q_{a}=CU_{a} \]
Voltage definition (CAP2):
\[ U_{a}=V_{i}-V_{j} \]
A dynamic relation (CAP3):
\[ I_{a}=\frac{dq_{a}}{dt} \]

After a time discretisation, these equations give tow linear relations between system's unknowns.
\subsection{Inductor laws (IND)}
A relation between inductor flux and current (IND1):\\
\[ \psi _{a}=LI_{a} \]
A dynamic relation (IND2):
\[ V_{i}-V_{j}=\frac{d\psi _{a}}{dt} \]
After a time discretisation, these equations give tow linear relations between system's unknowns.

\section{Example}
\begin{figure}[h]
\centerline{
 \scalebox{0.5}{
    \documentclass[10pt]{article}
\input{macro.tex}
\usepackage{psfrag}
\usepackage{fancyhdr}
\usepackage{subfigure}
%\renewcommand{\baselinestretch}{1.2}
\textheight 23cm
\textwidth 16cm
\topmargin 0cm
%\evensidemargin 0cm
\oddsidemargin 0cm
\evensidemargin 0cm
\usepackage{layout}
\usepackage{mathpple}
\makeatletter
\renewcommand\bibsection{\paragraph{References
     \@mkboth{\MakeUppercase{\bibname}}{\MakeUppercase{\bibname}}}}
\makeatother
%% style des entetes et des pieds de page
\fancyhf{} % nettoie le entetes et les pieds
\fancyhead[L]{Template 6 : Electrical oscillator with 4 diodes bridge full-wave rectifier - Pascal Denoyelle}
%\fancyhead[C]{}%
\fancyhead[R]{\thepage}
%\fancyfoot[L]{\resizebox{!}{0.7cm}{\includegraphics[clip]{logoesm2.eps}}}%
\fancyfoot[C]{}%
 \begin{document}
 
\section{Introduction\\}
This document is a short overview about the Modified Nodal Analysis. The M.N.A. is the method used
in SPICE to obtain the circuit equation formulation. For more details, read the book <<Circuit Simulation
Methods and Algorithms by Jan Ogrodzki>>.

\subsection{Notations}
\begin{enumerate}
  \item U is a tension, I is a current.
  \item V denotes a node's potential.
  \item q denotes a capacitor's charge.
  \item $\psi$ denotes a inductor's flux.
  \item Indice $_{a}$ denotes the current branch.
  \item Indice $_{b}$ denotes the other branch whose voltage is a controlling variable.
  \item Indice $_{c}$ denotes the other branch whose current is a controlling variable.
  \end{enumerate}

\newtheorem{mur}{Def}
\begin{mur}
The branch is current-defined if its currents is a function of its own voltage, controlling variable
or their derivatives:
\begin{equation}\label{CD}I_{a}=F_{i}(U_{a},U_{b},I_{c},\frac{dU_a}{dt},\frac{dU_b}{dt},\frac{dI_{c}}{dt})\end{equation}
\end{mur}
Examples : \\
A resistor is a current-defined branch because $I_{a}=\frac{U_{a}}{R}$.\\
A capacitor is a current-defined branch because $I_{a}=C\frac{dU_{a}}{dt}$.\\
\begin{mur}
The branch is voltage-defined if its voltage is a function of its own current, controlling variable
or their derivatives:
\begin{equation}\label{VD}U_{a}=F_{i}(I_{a},U_{b},I_{c},\frac{dU_a}{dt},\frac{dU_b}{dt},\frac{dI_{c}}{dt})\end{equation}
\end{mur}
Examples : \\
A resistor is a voltage-defined branch because $U_{a}=RI_{a}$.\\
A inductor is a voltage-defined branch because $U_{a}=L\frac{dI_{a}}{dt}$.\\
\section{Hypothesis\\}
The M.N.A. assumes smooth branches are explicit functions of current or voltage. It means each smooth
branch is Voltage Defined (V.D.) or Current Defined (C.D.)\\
\section{Unknowns}
The M.N.A. use the following unknowns:
\begin{enumerate}
\item Nodal voltages\\
\item Currents in the V.D. branches\\
\item Capacitor's charges and currents
\item Inductor's flux and currents
\item Currents control
\end{enumerate}
These unknowns are sufficient to describe the circuit.
\section{Equations}
The M.N.A. use following equations:
\subsection{The Kirchhoff Current Law (KCL)}
\newtheorem{kcl}{Kcl}
\begin{kcl}
At any node in an electrical circuit where charge density is not changing in time, the sum of
currents flowing towards that node is equal to the sum of currents flowing away from that node.
\end{kcl}
KCL law gives this type of equation:\\
$I_{1}+I_{2}+...+I_{n}=0$\\
Current from current-defined branch is replaced with relation \ref{CD}. The result is a linear relation between system's unknowns.
\subsection{Law in voltage-defined branches (LVD)}
It consists in replacing $U_{a}$ with $V_{i}-V_{j}$ in the relation \ref{VD} and we obtain a linear relation between system's unknowns.
\[V_{i}-V_{j}=F_{i}(I_{a},U_{b},I_{c},\frac{dU_a}{dt},\frac{dU_b}{dt},\frac{dI_{c}}{dt})\]
\subsection{Capacitor laws (CAP)}
A relation between capacitor charge and tension (CAP1):\\
\[ q_{a}=CU_{a} \]
Voltage definition (CAP2):
\[ U_{a}=V_{i}-V_{j} \]
A dynamic relation (CAP3):
\[ I_{a}=\frac{dq_{a}}{dt} \]

After a time discretisation, these equations give tow linear relations between system's unknowns.
\subsection{Inductor laws (IND)}
A relation between inductor flux and current (IND1):\\
\[ \psi _{a}=LI_{a} \]
A dynamic relation (IND2):
\[ V_{i}-V_{j}=\frac{d\psi _{a}}{dt} \]
After a time discretisation, these equations give tow linear relations between system's unknowns.

\section{Example}
\begin{figure}[h]
\centerline{
 \scalebox{0.5}{
    \input{MNA.pstex_t}
 }
}
\end{figure}

\subsection{Unknowns}
\begin{enumerate}
\item Branch 1 is voltage defined.
\item Branch 2 is looked as current defined ($I_{2}=\frac{U_{2}}{R}$).
\item Branch 3 is current defined.
\item Branch 4 is voltage defined.
\item Branch 5 is looked as current defined ($U_{5}=RIU_{5}$).
\end{enumerate}
Therefore the unknowns vector is:\\
(V1,V2,V3,I1,I3,I4,I5,U3,q3,$\psi 4$)
\subsection{Table Equation}
\[\left(\begin{array}{cccccccccccc}
  &V1&V2&V3&I1&I3&I4&I5&U3&q3&\psi 4\\
  \hline
  \left(\begin{array}{c} KCL1 \end{array}\right)&\frac{-1}{R_{2}}&\frac{1}{R_{2}}&0&-1&0&0&0&0&0\\
  \left(\begin{array}{c} KCL2 \end{array}\right)&\frac{1}{R_{2}}&\frac{-1}{R_{2}}&0&0&-1&0&0&0&0&0\\
  \left(\begin{array}{c} KCL3 \end{array}\right)&0&0&0&0&0&1&-1&0&0&0\\
  \left(\begin{array}{c} CAP1 \end{array}\right)&0&0&0&0&0&0&0&C_{3}&-1&0\\
  \left(\begin{array}{c} IND1 \end{array}\right)&0&0&0&0&0&L_{4}&0&0&0&-1\\
  \left(\begin{array}{c} VDL5 \end{array}\right)&0&0&\frac{-1}{R_{5}}&0&0&0&1&0&0&0\\
  \left(\begin{array}{c} VDL1 \end{array}\right)&1&0&0&0&0&0&0&0&0&0&0\\
  \left(\begin{array}{c} CAP2 \end{array}\right)&0&1&0&0&0&0&0&-1&0&0&0\\
  \left(\begin{array}{c} CAP3 \end{array}\right)&0&0&0&0&1&0&0&0&0&\frac{-1}{h}&0\\
  \left(\begin{array}{c} IND2 \end{array}\right)&0&1&0&0&0&0&0&0&0&0&\frac{-1}{h}\\
\end{array}\right) =
\left(\begin{array}{c}
  RSH\\
  \hline
  0\\
  0\\
  0\\
  0\\
  0\\
  0\\
  0\\
  U_{1}(t)\\
  0\\
  \frac{-q_{3}(t-h)}{h}\\
  \frac{-\psi_{4}(t-h)}{h}\\
\end{array}\right)\]



\section{Stamp method}
The stamp method is an algorithmic method used to fill the table equation from the components. It
consists in writing a sub-table for each type of component, this sub-table is the contribution of the component in the tableau equation.\\
Following, this is stamp examples.
\subsection{Resistor stamp}
\[\left(\begin{array}{cccc}
&V_{i}&V_{j}&RSH\\
  \hline
  KCL(i)&\frac{-1}{R}&\frac{1}{R}&\\
  KCL(j)&\frac{1}{R}&\frac{-1}{R}&\\
  \end{array}\right)
\]
Where R is the branch's resistance.
\subsection{Conductance stamp}
\[\left(\begin{array}{ccccc}
&V_{i}&V_{j}&I_{a}&RSH\\
  \hline
  KCL(i)&&&1&\\
  KCL(j)&&&-1&\\
  LVD&G&-G&1&\\
  \end{array}\right)
\]
Where G is the branch's conductance.
\subsection{Voltage source stamp}
\[\left(\begin{array}{ccccc}
&V_{i}&V_{j}&I_{a}&RSH\\
  \hline
  KCL(i)&&&1\\
  KCL(j)&&&-1\\
  LVD&1&-1&&E
  \end{array}\right)
\]
\subsection{Current controlled voltage source stamp}
\[\left(\begin{array}{cccccc}
&V_{i}&V_{j}&I_{a}&I_{b}&RSH\\
  \hline
  KCL(i)&&&1&\\
  KCL(j)&&&-1&\\
  LVD&1&-1&&\gamma
  \end{array}\right)
\]
With $U_{a} = \gamma I_{b}$.

 \end{document}

 }
}
\end{figure}

\subsection{Unknowns}
\begin{enumerate}
\item Branch 1 is voltage defined.
\item Branch 2 is looked as current defined ($I_{2}=\frac{U_{2}}{R}$).
\item Branch 3 is current defined.
\item Branch 4 is voltage defined.
\item Branch 5 is looked as current defined ($U_{5}=RIU_{5}$).
\end{enumerate}
Therefore the unknowns vector is:\\
(V1,V2,V3,I1,I3,I4,I5,U3,q3,$\psi 4$)
\subsection{Table Equation}
\[\left(\begin{array}{cccccccccccc}
  &V1&V2&V3&I1&I3&I4&I5&U3&q3&\psi 4\\
  \hline
  \left(\begin{array}{c} KCL1 \end{array}\right)&\frac{-1}{R_{2}}&\frac{1}{R_{2}}&0&-1&0&0&0&0&0\\
  \left(\begin{array}{c} KCL2 \end{array}\right)&\frac{1}{R_{2}}&\frac{-1}{R_{2}}&0&0&-1&0&0&0&0&0\\
  \left(\begin{array}{c} KCL3 \end{array}\right)&0&0&0&0&0&1&-1&0&0&0\\
  \left(\begin{array}{c} CAP1 \end{array}\right)&0&0&0&0&0&0&0&C_{3}&-1&0\\
  \left(\begin{array}{c} IND1 \end{array}\right)&0&0&0&0&0&L_{4}&0&0&0&-1\\
  \left(\begin{array}{c} VDL5 \end{array}\right)&0&0&\frac{-1}{R_{5}}&0&0&0&1&0&0&0\\
  \left(\begin{array}{c} VDL1 \end{array}\right)&1&0&0&0&0&0&0&0&0&0&0\\
  \left(\begin{array}{c} CAP2 \end{array}\right)&0&1&0&0&0&0&0&-1&0&0&0\\
  \left(\begin{array}{c} CAP3 \end{array}\right)&0&0&0&0&1&0&0&0&0&\frac{-1}{h}&0\\
  \left(\begin{array}{c} IND2 \end{array}\right)&0&1&0&0&0&0&0&0&0&0&\frac{-1}{h}\\
\end{array}\right) =
\left(\begin{array}{c}
  RSH\\
  \hline
  0\\
  0\\
  0\\
  0\\
  0\\
  0\\
  0\\
  U_{1}(t)\\
  0\\
  \frac{-q_{3}(t-h)}{h}\\
  \frac{-\psi_{4}(t-h)}{h}\\
\end{array}\right)\]



\section{Stamp method}
The stamp method is an algorithmic method used to fill the table equation from the components. It
consists in writing a sub-table for each type of component, this sub-table is the contribution of the component in the tableau equation.\\
Following, this is stamp examples.
\subsection{Resistor stamp}
\[\left(\begin{array}{cccc}
&V_{i}&V_{j}&RSH\\
  \hline
  KCL(i)&\frac{-1}{R}&\frac{1}{R}&\\
  KCL(j)&\frac{1}{R}&\frac{-1}{R}&\\
  \end{array}\right)
\]
Where R is the branch's resistance.
\subsection{Conductance stamp}
\[\left(\begin{array}{ccccc}
&V_{i}&V_{j}&I_{a}&RSH\\
  \hline
  KCL(i)&&&1&\\
  KCL(j)&&&-1&\\
  LVD&G&-G&1&\\
  \end{array}\right)
\]
Where G is the branch's conductance.
\subsection{Voltage source stamp}
\[\left(\begin{array}{ccccc}
&V_{i}&V_{j}&I_{a}&RSH\\
  \hline
  KCL(i)&&&1\\
  KCL(j)&&&-1\\
  LVD&1&-1&&E
  \end{array}\right)
\]
\subsection{Current controlled voltage source stamp}
\[\left(\begin{array}{cccccc}
&V_{i}&V_{j}&I_{a}&I_{b}&RSH\\
  \hline
  KCL(i)&&&1&\\
  KCL(j)&&&-1&\\
  LVD&1&-1&&\gamma
  \end{array}\right)
\]
With $U_{a} = \gamma I_{b}$.

 \end{document}

 }
}
\end{figure}

\subsection{Unknowns}
\begin{enumerate}
\item Branch 1 is voltage defined.
\item Branch 2 is looked as current defined ($I_{2}=\frac{U_{2}}{R}$).
\item Branch 3 is current defined.
\item Branch 4 is voltage defined.
\item Branch 5 is looked as current defined ($U_{5}=RIU_{5}$).
\end{enumerate}
Therefore the unknowns vector is:\\
(V1,V2,V3,I1,I3,I4,I5,U3,q3,$\psi 4$)
\subsection{Table Equation}
\[\left(\begin{array}{cccccccccccc}
  &V1&V2&V3&I1&I3&I4&I5&U3&q3&\psi 4\\
  \hline
  \left(\begin{array}{c} KCL1 \end{array}\right)&\frac{-1}{R_{2}}&\frac{1}{R_{2}}&0&-1&0&0&0&0&0\\
  \left(\begin{array}{c} KCL2 \end{array}\right)&\frac{1}{R_{2}}&\frac{-1}{R_{2}}&0&0&-1&0&0&0&0&0\\
  \left(\begin{array}{c} KCL3 \end{array}\right)&0&0&0&0&0&1&-1&0&0&0\\
  \left(\begin{array}{c} CAP1 \end{array}\right)&0&0&0&0&0&0&0&C_{3}&-1&0\\
  \left(\begin{array}{c} IND1 \end{array}\right)&0&0&0&0&0&L_{4}&0&0&0&-1\\
  \left(\begin{array}{c} VDL5 \end{array}\right)&0&0&\frac{-1}{R_{5}}&0&0&0&1&0&0&0\\
  \left(\begin{array}{c} VDL1 \end{array}\right)&1&0&0&0&0&0&0&0&0&0&0\\
  \left(\begin{array}{c} CAP2 \end{array}\right)&0&1&0&0&0&0&0&-1&0&0&0\\
  \left(\begin{array}{c} CAP3 \end{array}\right)&0&0&0&0&1&0&0&0&0&\frac{-1}{h}&0\\
  \left(\begin{array}{c} IND2 \end{array}\right)&0&1&0&0&0&0&0&0&0&0&\frac{-1}{h}\\
\end{array}\right) =
\left(\begin{array}{c}
  RSH\\
  \hline
  0\\
  0\\
  0\\
  0\\
  0\\
  0\\
  0\\
  U_{1}(t)\\
  0\\
  \frac{-q_{3}(t-h)}{h}\\
  \frac{-\psi_{4}(t-h)}{h}\\
\end{array}\right)\]



\section{Stamp method}
The stamp method is an algorithmic method used to fill the table equation from the components. It
consists in writing a sub-table for each type of component, this sub-table is the contribution of the component in the tableau equation.\\
Following, this is stamp examples.
\subsection{Resistor stamp}
\[\left(\begin{array}{cccc}
&V_{i}&V_{j}&RSH\\
  \hline
  KCL(i)&\frac{-1}{R}&\frac{1}{R}&\\
  KCL(j)&\frac{1}{R}&\frac{-1}{R}&\\
  \end{array}\right)
\]
Where R is the branch's resistance.
\subsection{Conductance stamp}
\[\left(\begin{array}{ccccc}
&V_{i}&V_{j}&I_{a}&RSH\\
  \hline
  KCL(i)&&&1&\\
  KCL(j)&&&-1&\\
  LVD&G&-G&1&\\
  \end{array}\right)
\]
Where G is the branch's conductance.
\subsection{Voltage source stamp}
\[\left(\begin{array}{ccccc}
&V_{i}&V_{j}&I_{a}&RSH\\
  \hline
  KCL(i)&&&1\\
  KCL(j)&&&-1\\
  LVD&1&-1&&E
  \end{array}\right)
\]
\subsection{Current controlled voltage source stamp}
\[\left(\begin{array}{cccccc}
&V_{i}&V_{j}&I_{a}&I_{b}&RSH\\
  \hline
  KCL(i)&&&1&\\
  KCL(j)&&&-1&\\
  LVD&1&-1&&\gamma
  \end{array}\right)
\]
With $U_{a} = \gamma I_{b}$.

 \end{document}

 }
}
\caption{Circuit for the MNA example}
\label{fig-MNA-example}
\end{figure}
\subsubsection{Topology: Branches and nodes definition}
The first step consists in deciding witch branches are voltage defined and witch are current
defined. For example, a resistor is a simple branch that can be solved either for the current I=U/R or for the
voltage U=RI. \\
After, the list of unknowns can be done. \\
Finally, the table equations is filled with the physical equations. 

\subsubsection{Branches analysis}

\begin{enumerate}
\item Branch 1 is voltage defined.
\item Branch 2 is looked as current defined ($I_{2}=\frac{U_{2}}{R}$).
\item Branch 3 is current defined.
\item Branch 4 is voltage defined.
\item Branch 5 is looked as voltage defined ($U_{5}=RI_{5}$).
\end{enumerate}
\subsubsection{Unknowns}
\begin{enumerate}
\item The voltage nodes : $V_{1}$,$V_{2}$,$V_{3}$.
\item $I_{1}$,$I_{4}$,$I_{5}$, because the branches 1,4 and 5 are voltage defined.
\item $q_{3}$ and $I_{3}$, because the branch 3 is a capacitor.
\item $\psi _{4}$ from the inductor branch
\end{enumerate}
Therefore the unknowns vector is:\\
($V_{1}$,$V_{2}$,$V_{3}$,$I_{1}$,$I_{3}$,$I_{4}$,$I_{5}$,$q_{3}$,$\psi _{4}$)
\subsubsection{Table Equations}
\[\left(\begin{array}{ccccccccccc}
  V_{1}&V_{2}&V_{3}&I_{1}&I_{3}&I_{4}&I_{5}&q_{3}&\psi _{4}\\
  \hline
  \left(\begin{array}{c} KCL1 \end{array}\right)&\frac{-1}{R_{2}}&\frac{1}{R_{2}}&0&-1&0&0&0&0&0\\
  \left(\begin{array}{c} KCL2 \end{array}\right)&\frac{1}{R_{2}}&\frac{-1}{R_{2}}&0&0&-1&0&0&0&0\\
  \left(\begin{array}{c} KCL3 \end{array}\right)&0&0&0&0&0&1&-1&0&0\\
  \left(\begin{array}{c} CAP1 \end{array}\right)&0&C_{3}&0&0&0&0&0&-1&0\\
  \left(\begin{array}{c} IND1 \end{array}\right)&0&0&0&0&0&L_{4}&0&0&-1\\
  \left(\begin{array}{c} VDL5 \end{array}\right)&0&0&\frac{-1}{R_{5}}&0&0&0&1&0&0\\
  \left(\begin{array}{c} VDL1 \end{array}\right)&1&0&0&0&0&0&0&0&0\\
  \left(\begin{array}{c} CAP2 \end{array}\right)&0&0&0&0&1&0&0&\frac{-1}{h}&0\\
  \left(\begin{array}{c} IND2 \end{array}\right)&0&1&0&0&0&0&0&0&\frac{-1}{h}\\
\end{array}\right) \left(\begin{array}{c}
 V_{1}(t+h)\\
 V_{2}(t+h)\\
 V_{3}(t+h)\\
 I_{1}(t+h)\\
 I_{3}(t+h)\\
 I_{4}(t+h)\\
 I_{5}(t+h)\\
 q_{3}(t+h)\\
 \psi _{4}(t+h)
  \end{array}\right)=
\left(\begin{array}{c}
  RSH\\
  \hline
  0\\
  0\\
  0\\
  0\\
  0\\
  0\\
  0\\
  U_{1}(t+h)\\
  0\\
  \frac{-q_{3}(t)}{h}\\
  \frac{-\psi_{4}(t)}{h}\\
\end{array}\right)\]
Each time step consists in solving a system 9x9.




%%% Local Variables: 
%%% mode: latex
%%% TeX-master: "ace"
%%% End: 
